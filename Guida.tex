\documentclass[letterpaper,twocolumn,openany,nodeprecatedcode]{dndbook}

% Use babel or polyglossia to automatically redefine macros for terms
% Armor Class, Level, etc...
% Default output is in English; captions are located in lib/dndstrings.sty.
% If no captions exist for a language, English will be used.
%1. To load a language with babel:
%	\usepackage[<lang>]{babel}
%2. To load a language with polyglossia:
%	\usepackage{polyglossia}
%	\setdefaultlanguage{<lang>}
\usepackage[italian]{babel}
%\usepackage[italian]{babel}
% For further options (multilanguage documents, hypenations, language environments...)
% please refer to babel/polyglossia's documentation.

\usepackage[utf8]{inputenc}
\usepackage[singlelinecheck=false]{caption}
\usepackage{kantlipsum}
\usepackage{listings}
\usepackage{shortvrb}
\usepackage{stfloats}
\usepackage{hyperref}

\captionsetup[table]{labelformat=empty,font={sf,sc,bf,},skip=0pt}

\MakeShortVerb{|}

\lstset{%
  basicstyle=\ttfamily,
  language=[LaTeX]{TeX},
  breaklines=true,
}

\title{Compendio (Semi)Omnicomprensivo di Ciana\\
    \large {\textit{Alcuni potranno morire, ma è un sacrificio che sono pronto a fare.}}
    }
\author{Ciana, ovviamente}
\date{}

\begin{document}

\frontmatter

\maketitle

\tableofcontents

\mainmatter%

\part{Ipotesi necessarie}

\chapter{Introduzione}

\subsection{Che cos'è questo libro?}

\DndDropCapLine{M}{}a come siete frettolosi, le presentazioni innanzitutto! Il mio nome è Ciana, sono un giocatore ormai da almeno un annetto e mi sto buttando nel mondo del mastering. ora che ci conosciamo, posso rispondervi: questo libro raccoglie il materiale homebrew usato nella mia prima campagna homebrew e quindi generalmente ammesso per le prossime!

\subsection{Ah, quindi è rivolto solo ai tuoi giocatori?}

Niente affatto! Il bello dell'homebrew di D\&D è condividerlo con più persone possibile. \\ Questo libro è frutto del lavoro, degli spunti, delle idee e dell'entusiasmo di molte persone, è nato per coinvolgere e ampliare l'esperienza di D\&D e così deve essere utilizzato. Non pretende di essere il materiale più interessante o bilanciato, probabilmente altre persone nella comunità avranno avuto idee simili, ma va bene così, questa è la nostra versione!


\chapter{Qualche regola}

\section{Condotta generale}

\DndDropCapLine{U}na campagna di D\&D è per sua natura un'esperienza da condividere con diverse persone, e sebbene esistano delle regole ufficiali o semiufficiali, è sempre una buona idea tracciare alcune linee guida per garantire lo svolgimento ottimale (in termini di gradevolezza del gioco) della campagna che ci si appresta a giocare.

\subsection{D\&D è un gioco di squadra}

Qualunque cosa succeda, è sempre il caso di ricordarsi che D\&D nasce come un gioco e quello deve restare, non ha senso prendersela o arrabbiarsi per cose che succedono nel gioco dandogli più importanza di quella che hanno. \\ La precondizione fondamentale di ogni gioiosa sessione o campagna di D\&D è la disposizione di ognuno alla comprensione e la tolleranza. \\ Tranne se si parla con un mago gnomo di nome Enefeles.

\subsection{Le regole vanno interpretate}

Spesso durante una sessione di D\&D capita che sorgano dubbi sull'interpretazione delle regole, che in quinta edizione sono spesso scritte in modo abbastanza vago. \\ In questi casi, i giocatori sono incoraggiati a proporre la loro interpretazione e accordarsi democraticamente, ma il giudizio ultimo è sempre quello del DM. \\ Se la discussione di una regola inizia a diventare troppo lunga, il DM può decidere immediatamente e chiudere la discussione.

\subsection{Segreti tra i personaggi, non tra i giocatori}

Un certo livello di metagaming è ammesso e necessario almeno secondo me, rende tutti i giocatori più partecipi e coinvolti nel roleplay. A meno che non ci sia un buon motivo di trama per tenere delle informazioni segrete tra i giocatori, è sempre una buona idea condividerle, purchè poi i personaggi non sfruttino o agiscano in virtù di informazioni che non hanno motivo di avere.

\subsection{I segreti del DM}

Lo giuro su tutti i miei dadi, se andate a cercare gli statblock delle creature che state affrontando vi mangio il cranio. Quella è la forma di metagaming più sbagliata di tutte, anche se conoscete già quelle creature se i vostri personaggi non le conoscono non avete alcun diritto di agire come se le conoscessero.\\ I segreti del DM sono assolutamente inviolabili, lo schermo è una barriera infrangibile. \\ I tiri del DM sono coperti per un buon motivo.

\subsection{Azioni e conseguenze}

Ogni giocatore è responsabile delle scelte compiute dal proprio personaggio. \\ \textit{L'unico e solo modo di conoscere gli effetti di un'azione è compierla, non chiederlo al DM}

\section{I personaggi e il party}

\DndDropCapLine{Q}{}ueste regole sono più per questioni di bilanciamento personali piuttosto che imperativi kantiani.

\subsection{Creazione del personaggio}

\paragraph{Statistiche al 1° livello}

Tirare 20d6 e distribuire i punteggi ottenuti come meglio si crede. Nessun punteggio (prima dei bonus razziali) può essere più alto di 15 o più basso di 5 per garantire una gestione tutto sommato pacifica del roleplay.

\paragraph{Talenti al 1° livello}

No, giocate l'umano variante se volete talenti al livello 1.

\paragraph{Personalità e background}

Cercate di evitare "lupi solitari", va bene un personaggio con delle difficoltà a livello relazionale ma provate a impersonare personaggi che vi diano comunque un sufficiente spazio di roleplay e relazione con gli altri membri del party, vi assicuro che rende il gioco più divertente per tutti.

\paragraph{Materiale sorgente}

Il \textit{Manuale del Giocatore}, la \textit{Guida Omnicomprensiva di Xanathar} e il \textit{Calderone Omnicomprensivo di Tasha} sono ammessi insieme ovviamente a questo libro(!), classi homebrew non sono ammesse in generale (salvo rarissime eccezioni), razze e sottoclassi homebrew sono generalmente bene accolte previa approvazione del DM. 

\paragraph{Equipaggiamento al 1° livello}

Oltre al normale equipaggiamento garantito dalle classi e dai background, ogni personaggio al 1° livello inizia con un oggetto magico che diventerà progressivamente più potente (ogni incremento del punteggio di caratteristica). Questo oggetto deve essere creato dai giocatori insieme al DM e deve essere parte della lore del personaggio. Se un personaggio subentra ad un livello più alto, sarà come se avesse avuto il suo oggetto fin dal 1° livello. Attenzione, perdere la sintonia con l'oggetto comporta la perdita di tutti i power up! Riacquistare la sintonia significa partire dal 1° livello dell'oggetto! Al 12° livello l'oggetto diventa senziente.

\paragraph{Roleplay sopra al powerplay}

Avere un personaggio forte è sicuramente divertente e incoraggiabile, ma piuttosto che ottimizzare l'utilità in combattimento a tutti i costi, ricordate sempre di non tralasciare il roleplay per ottenere quel d8 di danno in più.

\paragraph{Punti ferita ai livelli superiori}

Per quanto riguarda l'aumento dei punti ferita chiedo di usare sempre il valore atteso del dado vita e tratto come retroattivi gli aumenti del punteggio di Costituzione per il calcolo dei punti ferita.

\subsection{Il party}

\paragraph{Varietà}

Nella composizione del party chiedo di sforzarsi e mettersi d'accordo con gli altri giocatori per garantire una certa varietà in termini di classi, ruoli, razze e pesonalità, in quanto rende più interessante sia il combattimento che il roleplay.

\paragraph{Vincoli di composizione}

In particolare, chiederei di limitare i membri del party appartenenti ad una classe ad un singolo membro contemporaneamente, a meno che non si tratti di due sottoclassi che cambiano radicalmente il playstyle (ad esempio un Warlock melee e un Warlock cecchino). \\ Anche un singolo livello di multiclasse occupa lo "slot" della classe (sto guardando voi, che puntavate ad un singolo livello in \textit{Lama del Sortilegio}).

\paragraph{PVP}

Il PVP è ammesso purchè sia estremamente ben giustificato a livello di roleplay, è ammesso che i personaggi eventualmente si stacchino dal party e diventino NPC, non sono ammessi personaggi antagonisti per partito preso.

\subsection{Addio di un personaggio}

\paragraph{Morte}

Vi avverto fin dall'inizio: non sarò clemente con danni non letali e resurrezioni in generale, se un personaggio muore e non viene resuscitato prima che subentri il nuovo personaggio del giocatore, sarà da considerarsi morto. Se dovesse venire resuscitato, esisterà nella campagna come NPC controllato in condivisione dal DM e dal suo giocatore iniziale.

\paragraph{Allontanamento}

Può capitare che un giocatore perda il \textit{feeling} per un personaggio, che l'arco narrativo del personaggio si chiuda naturalmente o che per una serie di scelte gli obiettivi e i valori del personaggio si allontanino irreversibilmente da quelli del party. In questi casi, in accordo col DM, si può negoziare un'uscita di scena di questo personaggio e l'ingresso del nuovo personaggio del giocatore.\\ Si consiglia vivamente di non abusare di questa opzione.

\paragraph{Perdita del giocatore}

Infine, alle volte accade che non sia il personaggio ma il giocatore stesso ad abbandonare il party, per una serie di motivi. In base alla natura dell'assenza, il DM potrà concordare con il giocatore (se questi sarà disponibile a negoziare) il destino del suo personaggio, ma in qualsiasi caso è consigliabile optare per un allontanamento temporaneo dal party.

\subsection{Riunione col party dopo una morte o un'assenza}

\paragraph{Morte}

Ogni giocatore deve aver già pronto un personaggio da far subentrare in caso di morte del proprio per la sessione successiva. Alternativamente, in base al momento all'interno della trama, è straordinariamente possibile impersonare un personaggio "filler" per una sessione di divertimento e gaia trivialità.

\paragraph{Assenza prolungata}

Questo tipo di assenza andrebbe discussa con più attenzione e soprattutto anche con gli altri membri del party, soprattutto se dovessero essere subentrati nuovi giocatori. \\ In seguito alla riunione di un giocatore dopo un periodo di Erasmus, il suo personaggio perde temporaneamente la conoscenza della lingua Comune.

\section{Regole opzionali}

\DndDropCapLine{Q}{}ueste regole sono quelle che deviano o si aggiungono alle regole (o banalmente riporto da un manuale reinterpretate) di D\&D e che preferisco utilizzare nelle mie campagne.

\subsection{Realismo stronzo}

La quinta edizione di D\&D non è bilanciata intorno all'idea di una battaglia per riposo lungo, i personaggi hanno semplicemente troppe opzioni e si finisce per trascurare l'importante fattore di gestione delle risorse che giustifica la minore versatilità dei combattenti rispetto agli incantatori, ma allo stesso tempo giocare qualcosa come 5 o 6 incontri al giorno può risultare estenuante. Per questo io preferisco implementare la regola del \textit{Realismo Crudo}: un riposo breve richiede 8 ore e un riposo lungo richiede una settimana.

\subsection{Dadi esplosivi}

Un dado esplosivo è un dado che viene ritirato ogni volta che il suo risultato è pari al massimo del dado e alla fine del processo restituisce la somma delle iterazioni. \\ Ad esempio, un d4 esplosivo potrebbe restituire come serie di risultati 4, 4 e 2, quindi il risultato del tiro sarebbe 4+4+2=10.

\subsection{Endomulticlasse}

Come ammetto la multiclasse, ammetto anche che un personaggio multiclassi nella sua stessa classe, magari per ricevere i privilegi di diverse sottoclassi! Al fine di conteggiare i suoi privilegi limitati tuttavia questi vanno conteggiati separatamente e secondo il livello della classe che glieli garantisce. \\ Ad esempio un Monaco/Monaco di livello 10/10 potrà ridurre al massimo di 50 punti ferita il suo danno da caduta usando una reazione, mentre un Warlock/Warlock di livello 11/11 avrà due utilizzi dell'Arcanum Mistico ma entrambi limitati al 6° livello. \\ Ogni ambiguità va sottoposta al giudizio del DM, ovviamente.

\subsection{Multiclasse alternativa}

Non amo molto il sistema di multiclasse secondo il quale finchè prendi livelli in altri incantatori puri puoi avere la normale progressione degli slot. Per quanto mi riguarda, multiclassando in diverse classi da incantatore si ottengono gli stessi slot che si otterrebbero con le altre classi e si sommano ai propri. Ad esempio un Mago/Chierico 6/5 non avrà slot di 6° livello, ma ne avrà 3+2 slot di 3° livello. Allo stesso modo, un combattente che multiclassa in un incantatore non riceve metà degli slot che riceverebbe normalmente ma li riceve tutti.

\subsection{Danni non letali}

Non mi importa molto della logica, ogni personaggio può decidere di infliggere danni non letali con qualsiasi sorgente... a meno di spiacevoli incidenti... \textit{risata malvagia}. \\ Ogni volta che un personaggio decide di colpire con danni non letali deve tirare un d100. Se fa meno dei danni che ha causato con quell'attacco, i suoi danni sono comunque considerati letali. \\ Se un personaggio non è in grado di infliggere danni letali o non letali per via di un'altra regola (ad esempio il Monaco dell'Ombra Redenta), quella bypassa il tiro.

\subsection{Punti Follia}

Non sempre nelle campagne di D\&D si bada alla salute mentale dei personaggi, ma non è questo il caso nel mio mondo! Dopo aver vissuto esperienze disturbanti, il DM può premiare i vostri personaggi con dei fantastici punti Follia, in base all'esperienza!

\begin{DndTable}[header=Effetti della Follia]{XX}
    Punteggio & Effetto \\
    1-4 & Svantaggio alle prove basate su caratteristiche mentali e ai tiri salvezza contro gli incantesimi\\
    5-9 & Paura di creature, luoghi e oggetti casuali o determinati dal DM\\
    10-19 & Svantaggio a tutti i tiri salvezza, tiri per colpire e prove di caratteristica\\
    20-34 & Ogni ora va superato un tiro salvezza su Saggezza con CD 12 o si è colpiti da pazzia a breve termine per 1d10 minuti\\
    35-49 & Paralisi mentale, se non viene curata la Follia dopo 1d4 giorni arriva a 50 punti automaticamente\\
    50+ & Pazzia permanente, il personaggio cade sotto il controllo del DM e il giocatore deve usare un nuovo personaggio\\
\end{DndTable}

\paragraph{Curare la Follia}
Un riposo breve o mangiare una bacca di \textit{Bacche Benefiche} (al massimo una per riposo breve) cura 1 punto Follia, un riposo lungo ne cura 3, \textit{Ristorare Inferiore} ne cura 5 e \textit{Ristorare Superiore} porta al limite superiore del livello precedente, \textit{Parola del Potere Guarire} cura tutti i punti Follia a meno che non siano 50 o più e l'effetto di \textit{Desiderio} può riportare alla normalità qualunque livello di Follia.


\part{Opzioni per i personaggi}

\chapter{Razze}

% \section{Divorati}

% \textit{Questa razza è un estratto dalla Biblioteca Omnicomprensiva di Ker}

% \DndDropCapLine{Q}{}uando un incantatore supera i propri limiti e si spinge oltre, può accadere che il suo corpo non riesca a reggere la pressione e per non morire si aggrappi alla forza della magia. In questi casi l'incantesimo che lo salva può diventare parte di lui, e il suo corpo può divenire un canale per la magia stessa. La forma fisica dell'incantatore cambia, e il suo corpo diventa simile all'incantesimo che lo ha divorato. L'incantesimo tuttavia non agisce solo sull'aspetto del divorato, ma ne influenza anche la mente, e può portare a cambiamenti di personalità che spesso sfociano in comportamenti altrimenti inspiegabili e in certi casi può condurre persino alla follia.

% \subsection{La trasformazione}
% Un incantatore di livello inferiore al 12° non è abbastanza potente da lanciare incantesimi che possano ridurlo allo stato di divorato. Il loro corpo mantiene alcuni tratti in comune con la loro forma precedente, ad esempio la statura, ma diventa più slanciato e i suoi lineamenti divengono più affilati. La sua pelle assume un colore simile a quello dell'incantesimo che lo ha divorato, e i suoi occhi diventano vacui e profondi, a riflettere la corruzione che lo ha colpito.

% \subsection{Una razza molto rara}
% Sono pochi gli incantatori abbastanza potenti e avventati da diventare un divorato, e ancor meno sono quelli che sopravvivono alla trasformazione. Per questo motivo i divorati sono una razza molto rara, e la maggior parte di loro non ha mai incontrato un altro membro della propria specie. I divorati sono solitari per natura, e preferiscono vivere in luoghi isolati dove possono essere se stessi senza dover temere di essere giudicati.

% \subsection{Un nuovo potere...}
% La potente magia che pervade il divorato gli dona un nuovo e maggiore potere. La longevità di un divorato aumenta drasticamente, e un divorato sufficientemente potente può vivere anche per più di mille anni. Secondo alcune leggende, nascosti in luoghi remoti, ci sono alcuni divorati che lo sono diventati ancora prima della caduta del Netheril.  Un divorato ha un forte legame con il tipo di magia che lo ha divorato. I tipi dei danni inferti dal dvorato, detti da energia elementale, dipendono da questo legame.

% \subsection{...Ma a quale prezzo}
% Un divorato è un essere molto potente nelle arti magiche, ma questo potere ha un costo in termini di energie: egli può lanciare incantesimi come un qualsiasi incantatore oppure attingere all'energia che lo tiene in vita per lanciare speciali incantesimi senza consumare slot, ma facendo questo si indebolisce. La grande longevità di un divorato non gli impedisce tuttavia di morire in battaglia o di esaurire l'energia che lo alimenta. 

% \subsection{I nomi dei divorati}
% Un familiare che diviene un divorato è percepito tra molte razze come un grande
% disonore, per questo spesso i divorati si allontanano da chi era loro caro. Per evitare che il disonore cada anche sui loro familiari, molti divorati decidono di cambiare il proprio nome. Alcuni di loro scelgono un nome che rifletta la loro nuova natura, altri invece scelgono un nome che li aiuti a dimenticare il loro passato. Altri ancora preferiscono invece mantenere il proprio nome, o sceglierne uno simile, per ricordare chi erano e da dove vengono.

% \subsection{Tratti dei divorati}
% Venire divorato dalla magia è una vera e propria trascensione. Un divorato durante la trasformazione perde i suoi tratti razziali, ma acquisisce i seguenti.
% \paragraph{Vista cieca} Il divorato non può vedere come prima, ma ha un'altissima percezione della magia che lo circonda, il che gli consente di percepire l'ambiente circostante anche senza l'uso della vista.
% \paragraph{Nutrimento magico} Un divorato non si nutre come un normale essere vivente, ma attinge energia magica dall'ambiente circostante. Un divorato non ha bisogno di mangiare, bere o respirare. Il suo sonno è diverso da quello dei normali esseri viventi: il divorato entra in uno stato di trance in cui è semicosciente ed entra in contatto con la forza che lo ha divorato, attingendo alla magia per recuperare tutte le energie di cui ha bisogno. 4 ore di trance sono sufficienti per un divorato a effettuare un riposo lungo. Se un divorato rimane circa una settimana nel raggio di 1,5 m di un oggetto magico che non sia esplicitamente progettato per resistervi, esso perde permanentemente le sue proprietà magiche.
% \paragraph{Linguaggi} Il divorato mantiene la conoscenza delle lingue che aveva appreso prima di trasformarsi. Potrebbe accadere che, durante la trasformazione, un divorato perda la capacità di comprendere il comune.
% \paragraph{Velocità} La velocità base sul terreno di un divorato è di 9 metri.
% \paragraph{Difesa senza armatura} Un divorato non può indossare armature, ma la sua CA è pari a 10 + il suo modificatore di Intelligenza.
% \paragraph{Incremento del Punteggi di Caratteristica} Un punteggio a scelta tra Saggezza, Intelligenza e Carisma di un divorato aumenta di 2, fino a un massimo di 20. Un divorato perde eventuali incrementi dei punteggi di caratteristica dovuti alla sua razza prima della trasformazione.

% \subsection{Divorati elementali}
% Essi costituiscono la maggior parte dei divorati: quando un incantatore attinge all'energia elementale e diviene un divorato, il suo corpo diventa una manifestazione dell'elemento stesso Manifesta quindi una forte connessione con l'elemento di cui ora il suo corpo è fatto. Un divorato dell'acqua, ad esempio, potrebbe avere la pelle blu e gli occhi azzurri, mentre un divorato del fuoco potrebbe avere la pelle rossa e gli occhi gialli.
% \paragraph{Resistenza elementale} Un divorato elementale ha resistenza a un tipo di danno a scelta tra acido, freddo, fulmine, fuoco e tuono, che dipende dall'elemento cui è legato.
% \paragraph{Timore reverenziale} Un divorato elementale ha competenza nei tiri per intimidire.
% \paragraph{Velocità} La velocità base sul terreno di un divorato è di 7,5 metri.
% \paragraph{Affinità elementale}Un divorato elementale conosce l'incantesimo \textit{Scarica Elementale} e, un numero di volte pari al suo bonus di competenza per riposo lungo, può lanciarlo senza consumare slot. Il tipo di danni inferti dall'incantesimo dipende dall'elemento del divorato. Quando lo fa, subisce 2d4+2 danni puri.
% \paragraph{Morte violenta} Se un divorato muore per il contraccolpo di un proprio incantesimo, il suo corpo si dissolve nell'ambiente circostante, mentre se muore per altri motivi, il suo corpo esplode in una nube di energia elementale. In questo caso tutti coloro che si trovano entro 6 metri dal divorato devono effettuare un tiro salvezza su Destrezza. Se lo falliscono, subiscono 8d6 danni da energia elementale, mentre se lo superano, subiscono soltanto la metà di quei danni.

% \subsection{Divorati del Caos}
% Secondo antiche leggende, esisteva un altro tipo di magia, oggi quasi completamente andato perduto: la Magia del Caos. Solo pochi incantatori di straordinario potere erano in grado di padroneggiarla, e ancor meno erano abbastanza avventati da utilizzarla. La Magia del Caos era estremamente potente, ma anche molto pericolosa, e spesso gli incantatori che la utilizzavano perdevano il controllo e venivano divorati da essa. Molte persone credono che i Divorati del Caos, come la magia che li alimenta, siano solo un mito diffuso per spaventare i bambini. L'aspetto di un divorato del Caos è ingrado di incutere timore anche nei più coraggiosi: la sua pelle è del colore del vuoto più profondo e i suoi occhi brillano di una luce violastra.
% \paragraph{Resistenza magica} Un divorato del Caos ha resistenza ai danni da contundenti, perforanti, da taglio provocati da armi magiche e da forza.
% \paragraph{Velocità} La velocità base sul terreno di un divorato è di 10,5 metri.
% \paragraph{Aspetto del Caos} Un divorato del Caos ha maestria nei tiri per intimidire.
% \paragraph{Tutt'uno con la magia} Un divorato del Caos ha maestria nei tiri su arcano.
% \paragraph{Incantesimi del Caos}Un divorato del Caos conosce l'incantesimo \textit{Punizione del Caos} e, un numero di volte pari al suo bonus di competenza, può lanciarlo senza consumare slot. Quando lo fa, subisce 3d4+2 danni puri. Inoltre un divorato del caos conosce il trucchetto \textit{Deflagrazione Occulta}, ma ogni volta che lo lancia subisce 1 danno puro.
% \paragraph{Buco Nero} Quando un divorato del caos muore, genera una scarica di energia caotica che apre un portale verso il Piano Astrale nel punto in cui si trova. Qualsiasi creatura entro 3 metri dal portale viene risucchiata al suo interno e ricompare in un posto a caso sul Piano Astrale, poi il portale si richiude. Il portale è a senso unico e non può essere riaperto.

\section{Forgiati}

\DndDropCapLine{N}{}on c'è un'ambientazione fantasy che sia completa senza una vasta selezione di amiconi artificiali, del resto chi può dire di aver visto Guerre Stellari senza essersi innamorato di C3PO o di R2D2? Evidentemente poche persone, dato che internet è assolutamente pieno di varie versioni dei Forgiati. \\ Mancando tuttavia una versione ufficiale nei manuali di base di quinta edizione o in Tasha e Xanathar, ho deciso di sopperire con la mia versione, a voi!

\subsection{Tratti dei forgiati}
\paragraph{Aumento del punteggio di caratteristica} Il punteggio di Costituzione di un forgiato aumenta di 2.
\paragraph{Difesa senza Armatura} Un forgiato non può indossare armature, ma la sua CA è pari a 8 + il suo modificatore di Destrezza + il suo bonus di competenza.
\paragraph{Velocità} La velocità base sul terreno di un forgiato è di 9 metri.
\paragraph{Resistenza al Veleno} Un forgiato ha resistenza ai danni da veleno e ai tiri salvezza contro il veleno.
\paragraph{Radar} Un forgiato possiede scurovisione nel raggio di 18 m.

\subsection{Forgiati da combattimento}
\paragraph{Aumento del punteggio di caratteristica} Il punteggio di Destrezza di un forgiato da combattimento aumenta di 1.

\subsection{Forgiati da esplorazione}
\paragraph{Aumento del punteggio di caratteristica} Il punteggio di Intelligenza di un forgiato da esplorazione aumenta di 1.
\paragraph{Scanner magico} Un forgiato da esplorazione conosce l'incantesimo \textit{Individuazione del Magico} e può lanciarlo un numero di volte pari al suo bonus di competenza senza usare slot per ogni riposo lungo.
\paragraph{Velocità} La velocità base sul terreno di un forgiato da esplorazione è di 10,5 metri.

\subsection{Forgiati da costruzione}
\paragraph{Aumento del punteggio di caratteristica} Il punteggio di Forza di un forgiato da costruzione aumenta di 1.

\nchapter{Artefice}

\section{Privilegi di Classe Opzionali}



\chapter{Barbaro}

\begin{DndReadAloud}
    \it
    "GRRYAAAAAH!" \\ (Anonimo)
  \end{DndReadAloud}

\section{Privilegi di Classe Opzionali}

\section{Cammino Primordiale: Cammino del Caos}

\subsection{Tutto muscoli, tutto cervello}
A partire dal 3° livello, quando il Barbaro usa il suo privilegio \textit{Ira}, i suoi punteggi di Forza e Intelligenza vengono scambiati se quello di Intelligenza è più alto di quello di Forza.

\subsection{Incantesimi}
A partire dal 3° livello, il Barbaro acquisisce la capacità di lanciare degli incantesimi.
\paragraph{Caratteristica da incantatore} La caratteristica da incantatore del Barbaro è Intelligenza.
\paragraph{Trucchetti} Il Barbaro conosce due trucchetti a scelta dalla lista del Warlock. Al 10° livello ne apprende un terzo sempre dalla lista del Warlock.
\paragraph{Slot Incantesimo} Ha la stessa tavola di slot del Cavaliere Mistico.
\paragraph{Incantesimi conosciuti} Il Barbaro può imparare un numero di incantesimi da Warlock pari al suo livello da Barbaro più il suo modificatore di Intelligenza. Inoltre può imparare incantesimi da qualsiasi lista nel momento in cui questi gli siano insegnati.

\subsection{Corpo del Caos}
A partire dal 6° livello, quando il Barbaro usa il suo privilegio \textit{Ira} trasforma il suo corpo in un vortice di pura energia caotica.\\
La sua veloità di movimento aumenta di 1.5 metri e diventa una sorgente di luce fioca entro 9 metri.\\
Inoltre, quando effettua un tiro salvezza su Intelligenza durante la trasformazione, può trattarlo come un tiro salvezza su Forza.

\subsection{Orizzonte degli eventi}
A partire dal 10° livello, quando il Barbaro usa il suo privilegio \textit{Ira} proietta un campo gravitazionale intorno a sè. Quando una creatura alleata entro 1.5 metri da lui è bersaglio di un tiro a colpire a distanza, può usare la sua reazione per decidere di reindirizzarlo su di sè ma essere sicuramente colpito, riducendo i danni di 1d10 più il suo modificatore di Costituzione. Se i danni vengono ridotti a 0 da questo privilegio, come parte della stessa reazione può lanciare un trucchetto.\\
Se il Barbaro muore durante la trasformazione, apre un portale verso il Piano Astrale. Il portale origina dal punto in cui un oggetto è stato piazzato all'interno dell'altro. Qualsiasi creatura entro 3 metri dal portale viene risucchiata al suo interno e depositata in un luogo casuale del Piano Astrale. Poi il portale si chiude. Il portale è a senso unico e non può essere riaperto.

\subsection{}

\subsection{Ho detto tutto cervello}
Quando al 20° livello il Barbaro ottiene l'aumento sia del punteggio che del massimo punteggio di Forza, questo invece va su Intelligenza.

\chapter{Bardo}

\section{Privilegi di Classe Opzionali}

\section{Collegio Bardico: Collegio dei Ludopatici Anonimi}

\subsection{Antinormalità}

Ogni volta che il bardo effettua una prova di caratteristica, un tiro salvezza o un tiro per colpire, tira due volte e deve usare il risultato più lontano da 10; se il risultato del dado è minore di 10, sottrae il suo modificatore, mentre se è maggiore di 10 lo aggiunge normalmente.

\section{Collegio Bardico: Collegio del Primo Contatto}

\begin{DndReadAloud}

\end{DndReadAloud}

\subsection{}


\chapter{Chierico}

\section{Privilegi di Classe Opzionali}

\section{Dominio di Integrità}

\subsection{Incantesimi di dominio}

\begin{DndTable}{XX}
  Livello da Chierico  & Incantesimi \\
  1° & Armatura di Agathys, Comando\\
  3° & Frantumare, Immagine Speculare\\
  5° & Aura di Vitalità, Controincantesimo\\
  7° & Inaridire, Occhio Arcano\\
  9° & Mano di Bigby, Reincarnazione\\
\end{DndTable}

\subsection{Competenze bonus}

Al 1° livello, il Chierico ottiene competenza nelle seguenti abilità: Arcano e Intuizione. \\ Quando effettua una prova di caratteristica su queste abilità raddoppia il suo modificatore di competenza.

\subsection{Forza bruta}

A partire dal 1° livello, il Chierico può scegliere di usare il suo modificatore di Intelligenza invece che di Saggezza per ogni prova di caratteristica, tiro per colpire o tiro salvezza che lo richieda.\\ Allo stesso modo, la sua caratteristica da incantatore diventa Intelligenza.

\subsection{Incanalare Divinità: Divisone con resto}

A partire dal 2° livello il Chierico può spendere un suo utilizzo di \textit{incanalare divinità} per effettuare la divisione con resto su dei danni subiti da un suo alleato. \\ I danni vengono divisi per tre e inflitti equamente al Chierico, all'alleato e alla creatura che ha inflitto i danni. \\ I tre poi devono effettuare una prova contrapposta di Intelligenza pura, e chi ottiene il risultato più basso riceve il resto dei danni.

\subsection{Incanalare Divinità: Permutazione}

A partire dal 6° livello il Chierico può spendere un suo utilizzo di \textit{incanalare divinità} per scegliere un numero di creature pari o inferiore al suo livello da Chierico (almeno 2) di cui conosca le posizioni e permutarle a suo piacimento. \\ Se una creatura viene permutata in uno spazio che non può occupare, viene automaticamente teletrasportata nello spazio sicuro che possa occupare più vicino.

\subsection{Buon ordinamento}

A partire dall'8° livello, quando il Chierico tira per iniziativa, può scegliere l'ultima creatura alleata nell'ordine di iniziativa, sè stesso incluso, e decidere quando agirà all'inizio di ogni round.

\subsection{Fattorizzazione unica}

A partire dal 17° livello, una volta per riposo lungo, il Chierico può usare la sua azione per scegliere una creatura che sia in grado di vedere e conoscerne i punti ferita attuali. \\ Quando lo fa, deve scegliere uno dei fattori primi del numero di punti ferita attuali della creatura: questa viene divisa in quel numero di copie più piccole di sè stessa e i suoi punti ferita vengono divisi equamente tra di esse. \\ Le copie possiedono le stesse statistiche e abilità della creatura di partenza, ma non possono effettuare azioni o azioni bonus, soltanto reazioni e movimenti.\\ All'inizio del turno successivo del Chierico, le copie vengono rifuse nella creatura iniziale, la quale riappare nello spazio in cui si trovava prima di essere divisa o nel più vicino spazio libero e subisce i danni subiti da ognuna delle singole copie. \\ Se il numero di punti ferita della creatura è un numero primo, il Chierico ottiene un utilizzo aggiuntivo di \textit{incanalare divinità}. \\ Il DM sceglie la taglia delle copie (minore o uguale alla taglia della creatura originale) e la loro posizione.


\chapter{Druido}

\section{Privilegi di Classe Opzionali}

\section{Circolo Sperimentale}

\begin{DndReadAloud}
  \it
  "Siamo a Fisica, quantifichi!" \\ (R. Dolesi)
\end{DndReadAloud}

\subsection{Un caffè buonissimo, qualcosa di bello da mangiare}


\chapter{Guerriero}

\section{Privilegi di Classe Opzionali}


\chapter{Ladro}

\begin{DndReadAloud}
  \it
  "E lei che fa nella vita?" \\ L'halfling, con una naturalezza e un'innocenza disarmanti, sorrise ed esclamò: "Io rubo!"
\end{DndReadAloud}

\section{Privilegi di Classe Opzionali}

\subsection{Introvabile persino per il DM}

A partire dal 1° livello, se nella narrazione non è specificato che stia facendo qualcos'altro, il Ladro può effettuare una prova di Furtività con una CD di 35. \\ Se la supera (o ottiene un 20 naturale), si materializza nella scena attualmente narrata ed è considerato nascosto da tutte le creature coscienti. \\ Se lo ritiene opportuno e il giocatore non è presente, il DM può effettuare questa prova al posto del giocatore.

\section{Il Camminatore delle Ombre}

\DndDropCapLine{S}{}arà capitato a tutti nelle proprie avventure di ritrovarsi nel party un Ladro, generalmente un halfling, che paradossalmente si rivelasse inutile nel momento del bisogno (peraltro nel suo elemento naturale, l'oscurità) data la sua mancanza inspiegabile di Scurovisione. Ispirato da queste edificanti esperienze, ho deciso di darvi questo nuovo archetipo ladresco, il Camminatore delle Ombre!

\begin{DndReadAloud}
  \it
  "Beh, c'è chi dice che è morto. Baggianate, così penso io. No! Per me è ancora in circolazione..." \\ (Harry Potter e la Pietra Filosofale, 2001)
\end{DndReadAloud}

\subsection{Adottato dall'Ombra}

A partire dal 3° livello, il Ladro acquisisce \textit{Scurovisione} e la capacità di vedere nell'oscurità magica, se non le possiede già.

\subsection{Lampo nel Buio}

A partire dal 3° livello, quando il Ladro colpisce una creatura col suo attacco furtivo e lui stesso si trova in uno spazio in condizioni di luce fioca o oscurità, questa deve superare un TS su Saggezza. Se lo fallisce, subisce altri danni radiosi ed è accecata fino all'inizio del turno successivo del ladro. \\ I danni radiosi subiti sono 1d6 per ogni 2d6 lanciati dal Ladro per l'attacco furtivo.

\subsection{Passo nell'Ombra}

A partire dal 9° livello, il Ladro può usare la sua Azione Scaltra per teletrasportarsi nell'ombra di una creatura che sia in grado di vedere. \\ Se il Ladro era nascosto alla creatura lo rimane, altrimenti può tirare per una prova di Furtività per nascondersi nell'ombra della creatura.

\subsection{Cortina di Fumo!}

A partire dal 13° livello, il Ladro può usare la sua Azione Scaltra per lanciare il trucchetto \textit{Illusione Minore} e gli incantesimi \textit{Camuffare Sè Stesso}, \textit{Movimenti del Ragno} e \textit{Oscurità} a volontà. \\ Inoltre, quando si trova in condizioni di luce fioca o oscurità, può usare la sua Azione Scaltra per lanciare l'incantesimo \textit{Invisibilità Superiore} su sè stesso a volontà.
\paragraph{CD del tiro salvezza}La CD del tiro salvezza contro questi incantesimi è pari a 8 + il bonus di competenza del Ladro + il suo modificatore di Destrezza.

\subsection{Intoccabile come un'Ombra}

A partire dal 17° livello, il Ladro quando si trova in condizioni di luce fioca o oscurità supera automaticamente tutti i tiri salvezza su Destrezza. \\ Inoltre il suo punteggio di Destrezza aumenta di 4, e il valore massimo del suo punteggio di Destrezza è 24.

\subsection{Nemmeno Omatara potrebbe trovarlo}

A partire dal 17° livello, se i punti ferita del Ladro scendono a 0 mentre si trova in condizioni di luce fioca o oscurità, questi rimane a 1 punto ferita e viene considerato automaticamente nascosto da tutte le creature coscienti, alleati compresi. Questo privilegio può essere utilizzato una sola volta per riposo lungo.

\chapter{Mago}

\section{Privilegi di Classe Opzionali}

\subsection{Controcanto della Lama}

Dal 6° livello un Cantore della Lama, quando un nemico fallisce un attacco da mischia contro di lui, può usare la sua reazione per contrattaccare con la propria arma senza aggiungere il modificatore di caratteristica al danno. È comunque necessario il tiro per colpire.

\section{Il Criminale di Guerra}

\begin{DndReadAloud}
    \it
    Mi piace l'odore del napalm al mattino. Una volta abbiamo bombardato una collina, per dodici ore, e finita l'azione siamo andati a vedere. Non c'era più neanche l'ombra di quegli sporchi bastardi. Ma quell'odore... sai quell'odore di benzina? Tutto intorno. Profumava come... come di vittoria.\\ (Apocalypse Now, 1979)
\end{DndReadAloud}

\subsection{livello 2}
A partire dal 2° livello;

\subsection{livello 6}
A partire dal 6° livello;

\subsection{Bombardamento a Tappeto}
A partire dal 10° livello, prima di lanciare un incantesimo a distanza con un'area d'effetto circolare, il Mago può usare la sua azione bonus per chiamare un bombardamento a tappeto. Una volta che lo fa, lancia l'incantesimo e questo viene lanciato 2d6 volte in punti casuali determinati dal DM.\\ Una volta che questo privilegio è stato utilizzato, non può più essere utilizzato fino al prossimo riposo lungo.

\subsection{Cupola di Ferro}
A partire dal 14° livello, una volta per riposo lungo, il Mago può usare la sua azione per spendere uno slot di 5° livello o superiore ed entrare in concentrazione e azzerare la sua velocità di movimento. \\ Fino alla fine della sua concentrazione, ogni proiettile generato fuori da una sfera di raggio 18 m e centro nel Mago e diretto verso l'interno di questa sfera, viene intercettato da dei \textit{Dardi Incantati} partiti dal Mago non appena entra nella sfera. \\ La CA di ogni creatura all'interno della sfera contro attacchi a distanza provenienti dall'esterno della sfera riceve un bonus pari al livello dello slot utilizzato, come il tiro salvezza contro proiettili ad area d'effetto circolare provenienti dall'esterno della sfera.

\chapter{Monaco}

\begin{DndReadAloud}
  \it
  "Li vedi questi?" il Monaco cieco mostrò i suoi pugni, splendenti di energia verde, al povero borseggiatore colto in flagrante. \\ "Questo è Porta e questo è Fogli."
\end{DndReadAloud}

\section{Privilegi di Classe Opzionali}

\DndDropCapLine{A}{}h, il Monaco, una delle mie classi preferite da giocare, purtroppo a mio parere un po' troppo underpowered in termini di disponibilità di risorse e dipendenza dalle statistiche, quindi visto il mio grande amore per questa classe ho deciso di buffarla perchè sì.

\subsection{Riserva di ki aumentata}

Il numero massimo di punti ki di un Monaco è pari al suo livello + il suo bonus di competenza.

\subsection{Arti Marziali ampliate}

Un Monaco competente può usare un punto ki per aggiungere il suo modificatore di Destrezza alle prove di caratteristica di Atletica per afferrare una creatura.

\section{Via dell'Ombra Redenta}

\subsection{Pentimento e Redenzione}

\DndDropCapLine{S}{}ebbene tutti i monaci condividano la saggezza degli antichi maestri, ci sono tradizioni che si allontanano dalla loro illuminazione: è questo il caso della Via dell'Ombra. \\ Talvolta però accade che un Monaco, dopo aver sfiorato la morte ad esempio, un Monaco può decidere di allontanarsi da queste pratiche e riscoprire la dimensione più meditativa, pietosa e naturale delle arti del ki.

\subsection{Magia del ki}

Tra i molti privilegi concessi dalla via dell'Ombra, vi è quello di lanciare alcuni incantesimi, ma si tratta di un potere oppresso e incompleto. Una volta che un Monaco si libera dall'influenza dell'oscurità, può finalmente imparare a incanalare il suo ki per lanciare incantesimi più raffinati.

\paragraph{Trucchetti} Un Monaco Redento conosce due trucchetti a sua scelta tratti dalla lista degli incantesimi del druido. Apprende un trucchetto da druido aggiuntivo a sua scelta al 10° livello.

\paragraph{Slot incantesimo}La tabella indica quanti slot incantesimo possiede un Monaco Redento per lanciare i suoi incantesimi di 1° livello e di livello superiore. \\ Per lanciare uno di questi incantesimi, il Monaco deve spendere uno slot incantesimo di livello pari o superiore al livello dell'incantesimo. Per esempio, se un Monaco conosce l'incantesimo di 1° livello \textit{Scudo} e possiede uno slot incantesimo di 1° livello e uno slot incantesimo di 2° livello, può lanciare \textit{Scudo} usando uno qualsiasi dei due slot. \\ Il Monaco recupera tutti gli slot incantesimo spesi quando completa un riposo lungo.

\paragraph{Incantesimi conosciuti di 1° livello e superiore}Un Monaco Redento conosce tre incantesimi di 1° livello a sua scelta dalla lista di incantesimi del druido. La colonna "Incantesimi Conosciuti" nella tabella indica quando un Monacoo impara altri incantesimi di 1° livello o di livello superiore. Ognuno di questi incantesimi deve appartenere a un livello di cui il Monaco possiede degli slot incantesimo. Per esempio, quando un Monaco arriva al 7° livello in questa classe, può apprendere un nuovo incantesimo di 1° o 2° livello. \\ Ogni volta che il Monaco acquisisce un livello, può sostituire uno degli incantesimi da Monaco che conosce con un altro incantesimo a sua scelta della lista degli incantesimi del druido. Il nuovo incantesimo deve essere di un livello di cui il Monaco possiede almeno uno slot incantesimo.

\paragraph{Caratteristica da incantatore}Saggezza è la caratteristica da incantatore usata per gli incantesimi da Monaco Redento. Il potere dei suoi incantesimi deriva dalla sua sintonia con il Ki intorno a lui. Un Monaco usa Saggezza ogni volta che un incantesimo fa riferimento alla sua caratteristica da incantatore. Usa inoltre il suo modificatore di Saggezza per definire la CD del tiro salvezza di un incantesimo da Monaco da lui lanciato e quando effettua un tiro per colpire con un incantesimo.

\paragraph{CD del tiro salvezza degli incantesimi}= 8 + il bonus di competenza del Monaco + il modificatore di Saggezza del Monaco. 

\paragraph{Modificatore di attacco dell'incantesimo}= il bonus di competenza del Monaco + il modificatore di Saggezza del Monaco

\begin{DndTable}{XXXXXXX}
  Livello & Truc. con. & Inc. con. & Slot di 1° livello & Slot di 2° livello & Slot di 3° livello & Slot di 4° livello\\
  3° & 2 & 3 & 2 & - & - & - \\
  4° & 2 & 4 & 3 & - & - & - \\
  5° & 2 & 4 & 3 & - & - & - \\
  6° & 2 & 4 & 3 & - & - & - \\
  7° & 2 & 5 & 4 & 2 & - & - \\
  8° & 2 & 6 & 4 & 2 & - & - \\
  9° & 2 & 6 & 4 & 2 & - & - \\
  10° & 3 & 7 & 4 & 3 & - & - \\
  11° & 3 & 8 & 4 & 3 & - & - \\
  12° & 3 & 8 & 4 & 3 & - & - \\
  13° & 3 & 9 & 4 & 3 & 2 & - \\
  14° & 3 & 10 & 4 & 3 & 2 & - \\
  15° & 3 & 10 & 4 & 3 & 2 & - \\
  16° & 3 & 11 & 4 & 3 & 3 & - \\
  17° & 3 & 11 & 4 & 3 & 3 & - \\
  18° & 3 & 11 & 4 & 3 & 3 & - \\
  19° & 3 & 12 & 4 & 3 & 3 & 1 \\
  20° & 3 & 13 & 4 & 3 & 3 & 1 \\
\end{DndTable}

\subsection{Aura viva}

A partire dal 3° livello, il ki di un Monaco Redento si manifesta visualmente come un'aura intorno a lui, che sia cosciente o incosciente. Questa funge da fonte di luce fioca per tutte le creature non ostili. \\ Tutti i danni inferti dal Monaco a creature viventi sono danni non letali.

\subsection{Auree nell'oscurità}

A partire dal 6° livello, il Monaco perde la sua visione. Guadagna la capacità di vedere le auree delle creature viventi con la sua mente e può ricavarne diverse informazioni, come l'allineamento, i punti ferita, la capacità di lanciare incantesimi e la forza magica. Tuttavia grazie alla sua sintonia con il mondo circostante, guadagna vista cieca nel raggio di 18m. \\ A causa della sua cecità, quando usa il privilegio "Deviare i proiettili" il Monaco non può più rilanciare indietro i proiettili neutralizzati.
\paragraph{Consigli per il DM}È consigliabile comunicare al giocatore queste informazioni in modo implicito, tramite ad esempio una descrizione dell'aura della creatura, piuttosto che dargliele direttamente.

\subsection{Nulla si crea, nulla si distrugge, tutto si trasforma}

A partire dall'11° livello, una volta per riposo lungo, quando il Monaco vede l'aura di una creatura vivente scendere a 0 punti ferita, può spendere un turno in concentrazione per entrare in sintonia con il suo corpo e assorbire da essa un numero di punti ki pari al suo bonus di competenza, ma per ogni punto ki recuperato in questo modo perde lui stesso due punti ferita. \\ In qualunque momento, il Monaco può convertire i suoi punti ki in slot incantesimo temporanei e viceversa, con un tasso di un punto ki per livello (purchè possieda già uno slot di quel livello).

\subsection{Sacrificio assoluto}

A partire dal 16° livello, con un atto altruistico definitivo, il Monaco può attingere a tutto il suo ki per poi farlo esplodere con un'emanazione del raggio di 15 metri. Tutti gli alleati morti che siano all'interno dell'emanazione sono riportati in vita, come se fossero soggetti ad un incantesimo \textit{Resurrezione Pura}. \\ Il Monaco viene completamente distrutto. Un Monaco distrutto in tal modo non può più tornare in vita, nemmeno tramite un incantesimo \textit{Desiderio} o \textit{Miracolo} o grazie al potere di una divinità. \\ Inoltre, il nome del Monaco può essere pronunciato ma non potrà più essere scritto. Tutti i riferimenti scritti del suo nome divengono nient'altro che spazi bianchi e tutti gli oggetti magici con cui era in sintonia perdono ogni proprietà magica. \\ Tutto ciò che rimane del Monaco dopo aver usato questo privilegio è un sacchetto di semi di \textit{Principessa Serena} con su cucita l'unica testimonianza scritta del suo nome.



\chapter{Paladino}

\section{Privilegi di Classe Opzionali}

\section{Giuramento di Verità}


\chapter{Ranger}

\section{Privilegi di Classe Opzionali}



\chapter{Stregone}

\begin{DndReadAloud}
    \it
    "Uno stregone non è mai in ritardo, Frodo Baggins. Né in anticipo. Arriva precisamente quando intende farlo." \\ (Il Signore degli Anelli - La Compagnia dell'Anello, 2001)
  \end{DndReadAloud}

\section{Privilegi di Classe Opzionali}

\subsection{Improvvisazione}

La magia di uno Stregone è istintiva, deriva dal suo talento e dalle sue emozioni, quindi è sensato che gli giunga nel momento del bisogno. \\ Quando gli incantesimi conosciuti da uno Stregone dovrebbero aumentare salendo di livello, non li impara passivamente, ma ottiene degli slot temporanei aggiuntivi rispetto ai suoi, il cui livello è pari al livello massimo degli slot ordinari possieduti dallo Stregone e il cui numero è pari alla differenza tra il numero degli incantesimi che potrebbe conoscere e di quelli che conosce. \\ Questi slot sono cumulativi (il livello di ciascuno slot non sale tuttavia) e permangono finchè non vengono usati. Possono essere usati esclusivamente per lanciare incantesimi che lo Stregone non conosce ancora: quando un incantesimo viene lanciato in questo modo, lo Stregone lo aggiunge alla lista degli incantesimi che conosce.

\subsection{Riserva di punti stregoneria aumentata}

Il numero massimo di punti stregoneria di uno Stregone è pari al suo livello da Stregone + il suo bonus di competenza

\subsection{Opzioni di metamagia}
Innanzitutto, si possono usare più opzioni di metamagia nello stesso turno a meno che non specifichino esse stesse un costo in azioni et cetera.

\subsubsection{Incantesimo reattivo}
Lo Stregone può spendere 4 punti stregoneria per lanciare un incantesimo di 3° livello o inferiore (anche lanciato a livello più alto) come reazione ma durante il suo turno. Questa opzione può essere usata solo una volta per round.

\subsubsection*{Incantesimo leggendario}
Lo Stregone può spendere 4 punti stregoneria per lanciare un incantesimo di 3° livello o inferiore (anche lanciato a livello più alto) come azione leggendaria (ovvero in qualsiasi momento del round). Questa opzione può essere usata solo una volta per round.

\section{Dinastia di Analisti}

\begin{DndReadAloud}
  \it
  "\begin{math}f(x)\end{math} \ è suriettiva se whoop whoop, whooooooop!" \\ (A. Defranceschi)
\end{DndReadAloud}

\DndDropCapLine{L}{}'Analisi Matematica anche a livelli tutto sommato superficiali richiede la visualizzazione di concetti che i più potrebbero ritenere... psichedelici. \\ \begin{math} \mathbb{R}^7, \mathbb{C}^2\end{math}, spazi misurabili, aperti, chiusi, tutte strutture che una mente umana fatica a concepire. \\ Accadde che un giorno un'Analista particolarmente dedita alla sua disciplina fece ricorso ad un... \textit{aiutino druidico} per visualizzare certi concetti, ma purtroppo non sapeva di stare aspettando un figlio... Dopo qualche anno, fu estremamente sorpresa dalle doti di questo bambino, capace di oltrepassare senza problemi (... più o meno) i limiti della mente umana.

\subsection{Teorema prezzemolino}

A partire dal 1° livello, lo Stregone ottiene competenza in Natura, Sopravvivenza e nell'uso degli attrezzi da erborista.

\subsection{Connesso per poligonali}

A partire dal 1° livello, lo Stregone può seguire qualunque cammino lui voglia per arrivare in un punto qualsiasi nel raggio della sua velocità di movimento. Se esce dall'area minacciata di un nemico, subisce comunque gli attacchi di opportunità.

\subsection{Funzione Inversa}

A partire dal 6° livello, un numero di volte pari a metà del suo bonus di competenza per riposo lungo, lo Stregone quando lancia l'incantesimo \textit{Scudo} può spendere un punto stregoneria per riflettere l'incantesimo invece di pararlo semplicemente, usando come tiro per colpire il tiro originale dell'incantesimo. \\ Se lo Stregone non conosce già l'incantesimo \textit{Scudo}, lo impara automaticamente e questo non viene conteggiato tra i suoi incantesimi conosciuti.

\subsection{Superfici Complesse}

A partire dal 14° livello, l'Analista inizia ad interessarsi all'Analisi complessa e impara a vedere la realtà come si presenta davvero, reale e immaginaria. \\ Ottiene \textit{Vista Pura} nel raggio di 18 m.

\subsection{Massimo globale}

A partire dal 18° livello, all'inizio del suo turno lo Stregone può spendere 13 punti stregoneria per sovraccaricarsi di energia magica come azione gratuita. \\ Fino alla fine dello stesso turno, tutti i tiri per i danni da lui inflitti sono massimizzati (a meno di TS superati) e tutti gli incantesimi da lui lanciati che richiedono un tiro per colpire colpiscono automaticamente.

\section{Cavaliere della Trama}
\textit{Questa origine stregonesca non è di mia invenzione, è un'origine homebrew piuttosto popolare che includo e ribilancio qui al fine di includerla nelle mie campagne}

\begin{DndReadAloud}
  \it
  "L'abilità di distruggere un pianeta è insignificante in confronto alla potenza della Forza" \\ (Guerre Stellari, 1977)
\end{DndReadAloud}

\DndDropCapLine{A}{} prima vista l'Ordine dei Cavalieri della Trama può sembrare più simile ad un gruppo di monaci o ad un ordine di maghi, ma in realtà tutti i suoi membri sono stregoni.

\subsection{La Spada di Trama}

A partire dal 1° livello un Cavaliere della Trama può dedicare un riposo breve a concentrarsi per entrare in sintonia con un'arma a una mano in suo possesso e trasformarla in una Spada di Trama. \\ Usando un'azione bonus, l'arma può essere attivata, trasformando la sua lama in un fascio di energia magica pura. \\ I danni dell'arma diventano 1d8 + modificatore di Carisma danni da forza (mantenendo eventuali bonus dell'arma originale, ad esempio un +1 a colpire e ai danni). \\ L'arma guadagna le proprietà \textit{accurata} e \textit{a due mani} e il Cavaliere è considerato automaticamente competente con qualsiasi Spada di Trama da lui creata. \\ La Spada di Trama può essere manipolata usando l'incantesimo \textit{Mano Magica} (non per attaccare), ma mentre non si trova in mano al Cavaliere deve essere mantenuta attiva tramite la concentrazione dell'incantatore, altrimenti si disattiva. \\ In altri casi, la lama della Spada rimane attiva finchè non viene usata un'azione bonus per disattivarla, finchè il suo Cavaliere non diventa incosciente o fino alla sua morte. \\ Finchè un Cavaliere impugna la sua Spada della Trama, può usare una reazione per lanciare l'incantesimo \textit{Interdizione alle Lame} a volontà. \\ Il danno della Spada di Trama aumenta di 1d8 al 5°, 11° e 17° livello.

\subsection{Cavaliere della Trama}

A partire dal 1° livello un Cavaliere della Trama, come azione bonus, può spendere un punto stregoneria per ottenere uno dei seguenti effetti: \\
Ottenere vantaggio ad una prova di abilità, tiro per colpire con la sua Spada di Trama, tiro per colpire con un incantesimo o tiro salvezza. \\
Ottenere un bonus di +1 alla sua CA fino all'inizio del suo prossimo turno.

\subsection{Percorso della Trama}

Al 6° livello, il Cavaliere sceglie un Percorso da seguire:
\paragraph{Percorso del Grifone} La Spada perde la proprietà \textit{a due mani} e guadagna le proprietà \textit{leggera} e \textit{versatile (1d10)}
\paragraph{Percorso della Testuggine} Quando il Cavaliere impugna la sua Spada di Trama con una mano e non impugna nulla nell'altra, ottiene un bonus di +1 alla CA e viene considerato come se impugnasse uno scudo per ogni attività che ne richieda l'uso.
\paragraph{Percorso della Manticora} La Spada ottiene la proprietà \textit{lancio (6/12)}. Il Cavaliere può usare un'azione bonus per richiamare la Spada dopo averla lanciata.

\subsection{Stile della Trama}
Al 14° livello, il Cavaliere sceglie uno stile di combattimento tra i seguenti:
\paragraph{Stile del Fiume} Quando una Creatura effettua un attacco a distanza contro il Cavaliere, può usare la sua azione bonus per imporre svantaggio al tiro per colpire o usare anche un punto stregoneria per deviare l'attacco.
\paragraph{Stile della Tempesta} Il Cavaliere ottiene l'abilità di creare due Spade di Trama alla volta. Inoltre può usarle per combattere come se avesse lo stile di combattimento \textit{combattimento a due armi} e aggiungere il modificatore di Carisma anche al secondo attacco.
\paragraph{Stile del Vulcano} Il Cavaliere può usare un'azione bonus per spendere un punto stregoneria e uno slot incantesimo in suo possesso e ottenere una riserva di attacchi extra pari al livello dello slot utilizzato. Questa riserva dura 10 minuti. Quando il Cavaliere effettua quest'azione può subito usare uno degli attacchi. Nei turni successivi può spendere un'azione bonus per usare un altro degli attacchi rimasti.

\subsection{Tutt'uno con la Trama}
Al 18° livello il Cavaliere apprende un Percorso e uno Stile aggiuntivi tra quelli che non conosce.
\\ Inoltre, può spendere uno slot incantesimo per venire a conoscenza della presenza di ogni incantatore nel raggio di 1,5 km per livello dello slot e del loro slot incantesimo di livello più elevato.

\chapter{Warlock}

\begin{DndReadAloud}
  \it
  "Uh" il giovane Aiden osservò i tre Galeb Duhr che lo circondavano e sparì con un sonoro "Pop". \\ Prima che i tre elementali potessero capire cosa fosse successo, il braccio di uno dei Guardiani di Pietra esplose in un lampo di energia arancione brillante.
\end{DndReadAloud}

\section{Privilegi di Classe Opzionali}

\subsection{Supplica Occulta: Occhi Aperti}
\textit{Prerequisito: 18° livello} \\ Il potere del patrono, apre gli occhi al Warlock, che potrebbe non essere più in grado di richiuderli: ottiene \textit{vista pura}, non può essere limitata o disattivata se non rimuovendo questa supplica o accecandosi volontariamente. \\
La costante esposizione al Piano Etereo è estremamente impegnativa anche per la mente di incantatori molto esperti, quindi il Warlock ha bisogno di tempo ed esperienza per abituarsi.
\paragraph{Dopo nessun aumento di livello}Il Warlock ha costanti emicranee di intensità variabile in base alla presenza di auree molto potenti; in presenza di creature la cui forma nel Piano Etereo risulterebbe sconvolgente o disturbante (a discrezione del DM), il Warlock ha svantaggio ai TS per mantenere la concentrazione e ogni attacco inflitto da tali creature provoca 1d4-1 danni psichici aggiuntivi.
\paragraph{Dopo un aumento di livello}Il Warlock si sta abituando alla forma del mondo sul Piano Etereo. Non subisce più danni aggiuntivi e non ha più svantaggio ai TS per mantenere la concentrazione, a meno che questa non sia disturbata direttamente dalle creature dall'aspetto più disturbante.
\paragraph{Dopo due aumenti di livello o al 20° livello} Ormai la \textit{vista pura} è l'unica vista che il Warlock conosca, il Piano Etereo e il Piano Materiale per lui sono assolutamente inseparabili. La \textit{vista pura} non gli causa più nessun problema. Da questo momento in poi, rimuovere questa supplica gli causerebbe cecità permanente, incurabile se non da altre sorgenti di \textit{vista pura} (come \textit{Visione del Vero}). \\
Riottenere questa supplica dopo averla rimossa richiede di riabituarsi a essa.

\subsection{Supplica Occulta: Deflagrazione aleatoria}
\textit{Prerequisito: trucchetto Deflagrazione Occulta} \\
Un numero di volte pari al suo bonus di competenza per riposo breve, quando il Warlock colpisce un bersaglio con il trucchetto Deflagrazione Occulta, come azione bonus può imporre al bersaglio di tirare dalla tabella della Magia Selvaggia un numero di volte pari al numero di raggi da cui è stato colpito.

\section{Lo Spazio Proiettivo}

\begin{DndReadAloud}
  \it
  "E quando l'ho visto, ragazzi, ho esclamato 'Ma questa è Geometria Pura!'" \\ (M. Andreatta)
\end{DndReadAloud}

\DndDropCapLine{G}{}irano voci su una certa entità... Un oggetto legato alla stessa natura dei piani dove ci muoviamo... coloro che ne hanno ricevuto l'intuizione la descrivono come "una ganzata pazzesca" o "roba da Harry Potter", perdono l'uso della ragione e vanno in giro a importunare i passanti con frasi terrificanti come "Ragazzi, avete della Geometria?"... roba da brividi...

\subsection{Lista degli incantesimi ampliata}

\begin{DndTable}{XX}
  Livello dell'incantesimo  & Incantesimi \\
  1°  &  Caduta Morbida, Colpo Intrappolante\\
  2°  &  Levitazione, Zona di Verità\\
  3°  &  Glifo di Interdizione, Lentezza\\
  4°  &  Divinazione, Sfera Elastica di Otiluke\\
  5°  &  Cerchio di Teletrasporto, Legame Planare\\
\end{DndTable}

\subsection{}

\subsection{Omogeneizzazione}

A partire dal 6° livello, un numero di volte pari al suo bonus di competenza per riposo lungo, quando una creatura è a terra con 0 punti ferita, il Warlock può usare la sua azione e toccarla per trasferirla in un semipiano temporaneo. \\ Finchè la creatura si trova nel semipiano, il Warlock ha uno slot incantesimo aggiuntivo. Quando il Warlock inizia un riposo breve o usa questo privilegio su un'altra creatura, la creatura intrappolata viene liberata e diventa stabile.

\subsection{Proiezione}

A partire dal 10° livello, un numero di volte pari al suo bonus di competenza per riposo lungo, il Warlock può usare la sua azione per proiettare la sua immagine su una superficie solida che è in grado di vedere. \\ Per non più di 1 minuto, la posizione del Warlock diventa quella della sua immagine. Fintanto che si trova su una superficie, il Warlock può muoversi liberamentein ogni direzione lungo la stessa, al doppio della sua velocità di movimento, ma non può effettuare azioni che coinvolgano il mondo esterno ad eccezione del parlare. \\ Dopo 1 minuto, il Warlock esce dalla superficie e si trova nello spazio disponibile più vicino a dove si trovava sulla superficie. Questo effetto termina in anticipo se la superficie dove si trova il Warlock viene distrutta o se usa la sua azione per uscirne.

\subsection{Chiusura Proiettiva di Bézout}

A partire dal 14° livello, una volta per riposo breve, il Warlock può usare la sua azione bonus per generare un'aura di Geometria Pura nel raggio di 36m intorno a sè. \\ Fino alla fine del prossimo turno del Warlock, tutti gli attacchi a distanza che richiedono un tiro per colpire effettuati da creature dentro l'aura contro altre creature dentro l'aura vanno automaticamente a segno senza effettuare il tiro per colpire.

\section{Il Signore dell'Assurdo}

\DndDropCapLine{N}{}el multiverso sono poche le creature in grado di esistere contemporaneamente in ogni realtà. È questo il caso dell'entità conosciuta nel nostro mondo come il Signore dell'Assurdo. A voi potrebbe essere familiare sotto altri nomi, come Il Triangolo, Fancy Dorito, Alex, o il suo preferito... Bill. \\ Il suo vero obiettivo non è chiaro, ma una cosa è certa: ciò che pretende dai suoi Warlock è una sana dose di divertimento.

\subsection{Lista degli incantesimi ampliata}

\begin{DndTable}{XX}
  Livello dell'incantesimo  & Incantesimi \\
  1°  & Risata Incontenibile di Tasha, Dardo Tracciante\\
  2°  & Alterare Sè Stesso, Trucco della Corda \\
  3°  & Palla di Fuoco, Parola Guaritrice di Massa \\
  4° & Santuario Privato di Mordenkainen, Tentacoli Neri di Evard \\
  5° & Dominare Persone, Ristorare Superiore \\
\end{DndTable}

\subsection{Squarcio della fortuna}

A partire dal 1° livello, dopo aver lanciato un incantesimo il Warlock deve tirare il d20. Se il risultato del tiro è uguale o inferiore a 2 + il livello dello slot utilizzato, 2+0 nel caso di un trucchetto, il Warlock deve tirare il d100 e subire il relativo effetto dalla tabella della Magia Selvaggia. \\ Se il risultato dalla tabella è il lancio di un incantesimo, subisce anch'esso gli effetti di Squarcio della fortuna.\\ In compenso, il Warlock ha uno slot incantesimo per riposo breve in più, una supplica occulta in più e un utilizzo dell'arcanum mistico di ogni livello per riposo lungo in più.

\begin{DndTable}{XX}
  Livello da Warlock & Dado esplosivo \\
  1°-4°  & d4\\
  5°-10°  & d6 \\
  11°-16°  & d8 \\
  17°-20° & d10 \\
\end{DndTable}

\subsection{Deflagrazione deflagrante}

A partire dal 1° livello quando il Warlock colpisce un avversario con il trucchetto "Deflagrazione Occulta", come azione bonus può aggiungere il suo dado esplosivo ai danni inflitti da ogni raggio. \\ Se il Warlock è sotto un qualunque effetto di massimizzazione del risultato dei tiri per i danni, non può aggiungere il dado esplosivo. \\ Questo privilegio può essere usato un numero di volte pari al bonus di competenza del Warlock per riposo breve.

\subsection{(S)fortunato}

A partire dal 6° livello, il Warlock può appellarsi al suo patrono per alterare il fato in suo favore. \\ Ottiene il talento "Fortunato", ma ogni volta che lo usa deve tirare dalla tabella della Magia Selvaggia.

\subsection{Divertimento Extraplanare}

L'influenza del patrono si espande alle creature intorno al Warlock. \\ Al 10° livello il Warlock ottiene il privilegio "Ispirazione bardica", il dado di ispirazione è il suo dado esplosivo (che se usato in questo modo non esplode). \\ Se il dado di ispirazione risulta in un 1, l'utilizzatore deve tirare dalla tabella della Magia Selvaggia.

\subsection{Pandemonio}

Una volta per riposo lungo, il Warlock può imporre con un'azione a tutte le creature coscienti (compreso sè stesso) nel raggio di 18m di tirare dalla tabella della Magia Selvaggia e tirare il dado esplosivo del Warlock. Ogni creatura riceve il totale dei danni da lei tirati come danni psichici.



\chapter{Talenti}

\subsection{Negazione divina}

\paragraph{Prerequisiti}Essere un negoziante nell'esercizio della sua protezione.
\paragraph{Aumento dei punteggi di caratteristica} Fintanto che i requisiti di questo talento sono soddisfatti, i punteggi di caratteristica del personaggio diventano tutti 20.
\paragraph{Competenze bonus}Fintanto che i requisiti di questo talento sono soddisfatti, il personaggio diventa competente in tutte le abilità.
\paragraph{Parola del Potere: No}Fintanto che i requisiti di questo talento sono soddisfatti, il personaggio può lanciare l'incantesimo di 10° livello \textit{Parola del Potere: No} a volontà senza spendere componenti materiali o slot incantesimo.



\chapter{Incantesimi}

\section{Lista degli incantesimi espansa}

\section{Descrizione degli incantesimi}

\DndSpellHeader%
  {Limita Magie}
  {Abiurazione di 7° livello}
  {1 reazione, che l'incantatore effettua quando viene lanciato un incantesimo con bersaglio entro gittata}
  {9 metri}
  {S}
  {Concentrazione, finché non viene dissolto}
L'incantesimo crea una bolla di raggio a scelta dell'incantatore compreso tra 0,5 metri e 3 metri entro gittata. Ogni incantesimo che si trova all'interno della bolla viene bloccato nell'istante in cui compare la bolla. Quando la bolla scompare, gli incantesimi al suo interno riprendono il proprio corso.

\DndSpellHeader%
  {Parola del Potere: No}
  {Ammaliamento di 10° livello}
  {1 reazione}
  {Vista}
  {V}
  {Istantanea}
L'Incantatore pronuncia una parola del potere per negare le azioni e le intenzioni di un gruppo di creture in grado di sentirlo.
Le azioni compiute dalle creature influenzate dalla fine dell'ultimo turno dell'incantatore vengono annullate.
Tutte le creature influenzate sono spaventate fino alla fine del prossimo turno dell'Incantatore.

\DndSpellHeader%
  {Punizione del Caos}
  {Invocazione di 3° livello}
  {1 azione bonus}
  {Incantatore}
  {V}
  {Concentrazione, fino a 1 minuto}
La prossima volta che l'incantatore colpisce una creatura con un attacco con un'arma da mischia entro la durata di questo incantesimo, l'arma in questione viene pervasa da un'aura di \textit{energia caotica} e l'attacco infligge 3d8 danni necrotici extra al bersaglio. 
\subparagraph{Ai livelli superiori}Quando questo incantesimo viene lanciato con uno slot di livello superiore al 3°, i danni inflitti aumentano di 1d8 ogni due livelli.

\DndSpellHeader{Scarica Elementale}
{Invocazione di 3° livello}
{1 azione}
{Incantaore (linea di 30 metri)}
{V, S}
{Istantanea}
Un raggio di energia elementale parte dall'incantatore in una direzione a sua scelta, formando una linea lunga 30 metri a larga 1,5 metri. Ogni creatura situata entro una sfera del raggio di 6 metri centrata su quel punto deve effettuare un tiro salvezza su Destrezza. Se lo fallisce, subisce 8d6 danni da  energia elementale del tipo scelto dall'incantatore (acido, freddo, fulmine, fuoco o tuono), mentre se lo supera, subisce soltanto la metà di quei danni. 
L'esplosione si diffonde oltre gli angoli e danneggia ogni oggetto vulnerabile al tipo di danno scelto nell'area che non sia indossato o trasportato.




\chapter{Oggetti magici}

\section{Anello protettore di Sophos il Savio}

\textit{Artefatto meraviglioso leggendario, richiede sintonia con un Incantatore che stia venendo divorato dalla magia. \\ Questo oggetto è un estratto dalla Biblioteca Omnicomprensiva di Ker.}

\DndDropCapLine{D}{}urante le sue ricerche per scoprire quanto più possibile sulla magia, il mago Ælinor Könungru si è imbattuto in un incantesimo estremamente antico e potente, che appena lanciato ha cominciato a consumarne il corpo. Per fortuna è riuscito a contenerlo e ad isolarlo nel proprio braccio grazie al potere dell'Anello Protettore di Sophos il Savio. L'incantesimo non è tuttavia sparito, e i suoi segni sono rimasti impressi sulla pelle del mago.

\begin{DndReadAloud}
  \it
  L'Anello Protettore di Sophos il Savio permette a chi lo indossa di attingere ad una fonte di antica energia racchiusa nella sua gemma, che reagisce con alcuni tipi di incantesimi protettivi alimentandoli senza richiedere energia all'incantatore.
\end{DndReadAloud}

\subsection{Potere passivo}
L'anello può sostenere la concentrazione necessaria al mantenimento dell'incantesimo \textit{Limita Magie}. 

\subsection{Potere Attivo}
\textit{Requisito: 4 livelli in sintonia con l'anello.}\\
Un numero di volte pari al proprio bonus di competenza per riposo lungo, l'incantatore che usa l'anello può lanciare l'incantesimo \textit{scudo} senza consumare slot.

\subsection{Coscienza}
\textit{Requisito: 12 livelli in sintonia con l'anello.}\\
L'anello è considerato senziente, la sua personalità è simile a quella di Sophos il Savio ma non ne possiede i ricordi o le conoscenze.

\subsection{Corruzione mortale}
\textit{Requisito: 12 livelli in sintonia con l'anello.}\\
Una volta per riposo lungo, l'Incantatore può usare la sua reazione per spostare l'incantesimo \textit{Limita Magia} dal suo corpo ad un punto esterno. Fare questo tuttavia libera l'incantesimo nel suo corpo, il quale finché la bolla non viene ripristinata subisce 1d6+2 danni necrotici ogni volta che lancia un incantesimo e 1d4 danni necrotici ogni turno. Se l'incantatore mantiene un'altra concentrazione, ogni turno deve passare un TS su costituzione con CD pari alla propria CD del tiro salvezza contro gli incantesimi, altrimenti la perde. Se l'incantatore muore per questi danni, si trasforma in un divorato.

\section{STOCCO DI SARZEE}

\textit{Artefatto meraviglioso leggendario, richiede sintonia con un Bardo}

\section{Prigione Fluida di PATRONO DI SIMONE}

\textit{Artefatto meraviglioso leggendario, richiede sintonia con un Warlock}



\part{Il Mondo di Eovras}

\chapter{Foresta di Mythrenwald}

\section{Luoghi}

\section{Abitanti}

\subsection{Halimath Selevarum}

\subsubsection{Il cieco con gli occhi aperti}

Nessuno conosce davvero la storia di come il grande Guardiano di Mythrenwald abbia perso la vista, ma tutti coloro che ne abbiano mai sentito parlare sanno bene che non è saggio assumere che Halimath Selevarum non sia altro che un monaco cieco e indifeso. \\ La sua sconfinata saggezza è frutto dell'esperienza di quasi nove secoli, la sua è una storia di redenzione e di introspezione... ma non è detto che sia così propenso a raccontarvela.

\subsubsection{Un passato oscuro}

Sono in pochi coloro che sanno che in realtà Halimath è giunto a Eovras quando ormai era già un guerriero esperto. \\ Una notte d'estate fu trovato in una radura, nudo e privo di sensi, il suo corpo pieno di bruciature e cicatrici, ma con un sorriso sereno in volto. Quando i druidi di Mythrenwald riuscirono a fargli riprendere i sensi, si trovarono davanti un elfo completamente in pace con sè stesso. Non parlò mai a nessuno del suo passato.

\begin{DndMonster}[float*=b,width=\textwidth + 8pt]{Halimath Selevarum}
    \begin{multicols}{2}
      \DndMonsterType{Elfo dei boschi, buono neutrale}
  
      % If you want to use commas in the key values, enclose the values in braces.
      \DndMonsterBasics[
          armor-class = {20},
          hit-points  = {\DndDice{40d8 + 200}},
          speed       = {19.5 m},
        ]
  
      \DndMonsterAbilityScores[
          str = 12,
          dex = 20,
          con = 20,
          int = 20,
          wis = 20,
          cha = 16,
        ]
  
      \DndMonsterDetails[
          saving-throws = {Str +13, Dex +17, Con +5, Int +17, Wis +17, Cha +5},
          skills = {Animal Handling +17, Arcana +17, Athletics +13, Insight +17, Perception +17, Sleight of Hand +17, Stealth +17},
          %damage-vulnerabilities = {cold},
          %damage-resistances = {bludgeoning, piercing, and slashing from nonmagical attacks},
          %damage-immunities = {poison},
          condition-immunities = {Avvelenato,},
          senses = {Vista cieca 36 m, vista delle auree 72 m. Percezione Passiva 27},
          languages = {Comune, Elfico, Silvano, Gnomesco, Druidico, Primordiale, Draconico},
          challenge = 1,
        ]
      % Traits

      \DndMonsterAction{Tratti di classe}
      Halimath è un Monaco dell'Ombra Redenta di 20° livello e un Druido del Circolo delle Stelle di 20° livello. Possiede tutti i tratti garantitigli da queste due classi.

      \DndMonsterAction{Incantesimi}
      Halimath è un druido di 20° livello. Conosce tutti gli incantesimi da druido. Recupera i suoi slot dopo ogni riposo lungo.
      \begin{DndTable}[header=Slot per livello]{XXXXXXXXX}
        1° & 2° & 3° & 4° & 5° & 6° & 7° & 8° & 9°\\
        8  & 6  & 6  & 4  & 3  & 2  & 2  & 1  & 1 \\
      \end{DndTable}
  
      \DndMonsterSection{Azioni}
      \DndMonsterAction{Multiattacco}
      Halimath compie tre attacchi da mischia.
  
      %Default values are shown commented out
      \DndMonsterAttack[
        name=Pugni,
        distance=melee, % valid options are in the set {both,melee,ranged},
        %type=weapon, %valid options are in the set {weapon,spell}
        mod=+17,
        %reach=1.5,
        %range=20/60,
        %targets=bersaglio singolo,
        dmg=\DndDice{1d10+5},
        dmg-type=forza,
        %plus-dmg=,
        %plus-dmg-type=,
        %or-dmg=,
        %or-dmg-when=,
        %extra=,
      ]
  
      % Legendary Actions
      \DndMonsterSection{Punti ki}
      Halimath possiede 32 punti ki che può usare per i poteri di un Monaco dell'Ombra Redenta di 20° livello.
    \end{multicols}
\end{DndMonster}

\subsection{I Druidi di Mythrenwald}

\part{Nemici e Pericoli}

\end{document}