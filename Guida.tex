\documentclass[letterpaper,twocolumn,openany,nodeprecatedcode]{dndbook}

% Use babel or polyglossia to automatically redefine macros for terms
% Armor Class, Level, etc...
% Default output is in English; captions are located in lib/dndstrings.sty.
% If no captions exist for a language, English will be used.
%1. To load a language with babel:
%	\usepackage[<lang>]{babel}
%2. To load a language with polyglossia:
%	\usepackage{polyglossia}
%	\setdefaultlanguage{<lang>}
\usepackage[italian]{babel}
%\usepackage[italian]{babel}
% For further options (multilanguage documents, hypenations, language environments...)
% please refer to babel/polyglossia's documentation.

\usepackage[utf8]{inputenc}
\usepackage[singlelinecheck=false]{caption}
\usepackage{kantlipsum}
\usepackage{listings}
\usepackage{shortvrb}
\usepackage{stfloats}
%\usepackage{hyperref}

\captionsetup[table]{labelformat=empty,font={sf,sc,bf,},skip=0pt}

\MakeShortVerb{|}

\lstset{%
  basicstyle=\ttfamily,
  language=[LaTeX]{TeX},
  breaklines=true,
}

\title{Compendio (Semi)Omnicomprensivo di Ciana\\
    \large {\textit{Alcuni potranno morire, ma è un sacrificio che sono pronto a fare.}}
    }
\author{Ciana, ovviamente}
\date{}

\begin{document}

\frontmatter

\maketitle

\tableofcontents

\mainmatter%

\part{Ipotesi necessarie}

\nchapter{Introduzione}

\subsection{Che cos'è questo libro?}

\DndDropCapLine{M}{}a come siete frettolosi, le presentazioni innanzitutto! Il mio nome è Ciana, sono un giocatore ormai da almeno un annetto e mi sto buttando nel mondo del mastering. ora che ci conosciamo, posso rispondervi: questo libro raccoglie il materiale homebrew usato nella mia prima campagna homebrew e quindi generalmente ammesso per le prossime!

\subsection{Ah, quindi è rivolto solo ai tuoi giocatori?}

Niente affatto! Il bello dell'homebrew di D\&D è condividerlo con più persone possibile. \\ Questo libro è frutto del lavoro, degli spunti, delle idee e dell'entusiasmo di molte persone, è nato per coinvolgere e ampliare l'esperienza di D\&D e così deve essere utilizzato. Non pretende di essere il materiale più interessante o bilanciato, probabilmente altre persone nella comunità avranno avuto idee simili, ma va bene così, questa è la nostra versione!


\chapter{Qualche regola}

\section{Condotta generale}

\DndDropCapLine{U}na campagna di D\&D è per sua natura un'esperienza da condividere con diverse persone, e sebbene esistano delle regole ufficiali o semiufficiali, è sempre una buona idea tracciare alcune linee guida per garantire lo svolgimento ottimale (in termini di gradevolezza del gioco) della campagna che ci si appresta a giocare.

\subsection{D\&D è un gioco di squadra}

Qualunque cosa succeda, è sempre il caso di ricordarsi che D\&D nasce come un gioco e quello deve restare, non ha senso prendersela o arrabbiarsi per cose che succedono nel gioco dandogli più importanza di quella che hanno. \\ La precondizione fondamentale di ogni gioiosa sessione o campagna di D\&D è la disposizione di ognuno alla comprensione e la tolleranza. \\ Tranne se si parla con un mago gnomo di nome Enefeles.

\subsection{Le regole vanno interpretate}

Spesso durante una sessione di D\&D capita che sorgano dubbi sull'interpretazione delle regole, che in quinta edizione sono spesso scritte in modo abbastanza vago. \\ In questi casi, i giocatori sono incoraggiati a proporre la loro interpretazione e accordarsi democraticamente, ma il giudizio ultimo è sempre quello del DM. \\ Se la discussione di una regola inizia a diventare troppo lunga, il DM può decidere immediatamente e chiudere la discussione.

\subsection{Segreti tra i personaggi, non tra i giocatori}

Un certo livello di metagaming è ammesso e necessario almeno secondo me, rende tutti i giocatori più partecipi e coinvolti nel roleplay. A meno che non ci sia un buon motivo di trama per tenere delle informazioni segrete tra i giocatori, è sempre una buona idea condividerle, purchè poi i personaggi non sfruttino o agiscano in virtù di informazioni che non hanno motivo di avere.

\subsection{I segreti del DM}

Lo giuro su tutti i miei dadi, se andate a cercare gli statblock delle creature che state affrontando vi mangio il cranio. Quella è la forma di metagaming più sbagliata di tutte, anche se conoscete già quelle creature se i vostri personaggi non le conoscono non avete alcun diritto di agire come se le conoscessero.\\ I segreti del DM sono assolutamente inviolabili, lo schermo è una barriera infrangibile. \\ I tiri del DM sono coperti per un buon motivo.

\subsection{Azioni e conseguenze}

Ogni giocatore è responsabile delle scelte compiute dal proprio personaggio. \\ \textit{L'unico e solo modo di conoscere gli effetti di un'azione è compierla, non chiederlo al DM}

\section{I personaggi e il party}

\DndDropCapLine{Q}{}ueste regole sono più per questioni di bilanciamento personali piuttosto che imperativi kantiani.

\subsection{Creazione del personaggio}

\paragraph{Statistiche al 1° livello}

Tirare 20d6 e distribuire i punteggi ottenuti come meglio si crede. Nessun punteggio (prima dei bonus razziali) può essere più alto di 15 o più basso di 5 per garantire una gestione tutto sommato pacifica del roleplay.

\paragraph{Talenti al 1° livello}

No, giocate l'umano variante se volete talenti al livello 1.

\paragraph{Personalità e background}

Cercate di evitare "lupi solitari", va bene un personaggio con delle difficoltà a livello relazionale ma provate a impersonare personaggi che vi diano comunque un sufficiente spazio di roleplay e relazione con gli altri membri del party, vi assicuro che rende il gioco più divertente per tutti.

\paragraph{Materiale sorgente}

Il \textit{Manuale del Giocatore}, la \textit{Guida Omnicomprensiva di Xanathar} e il \textit{Calderone Omnicomprensivo di Tasha} sono ammessi insieme ovviamente a questo libro(!), classi homebrew non sono ammesse in generale (salvo rarissime eccezioni), razze e sottoclassi homebrew sono generalmente bene accolte previa approvazione del DM. 

\paragraph{Equipaggiamento al 1° livello}

Oltre al normale equipaggiamento garantito dalle classi e dai background, ogni personaggio al 1° livello inizia con un oggetto magico che diventerà progressivamente più potente (ogni incremento del punteggio di caratteristica). Questo oggetto deve essere creato dai giocatori insieme al DM e deve essere parte della lore del personaggio. Se un personaggio subentra ad un livello più alto, sarà come se avesse avuto il suo oggetto fin dal 1° livello. Attenzione, perdere la sintonia con l'oggetto comporta la perdita di tutti i power up! Riacquistare la sintonia significa partire dal 1° livello dell'oggetto! Al 12° livello l'oggetto diventa senziente.

\paragraph{Roleplay sopra al powerplay}

Avere un personaggio forte è sicuramente divertente e incoraggiabile, ma piuttosto che ottimizzare l'utilità in combattimento a tutti i costi, ricordate sempre di non tralasciare il roleplay per ottenere quel d8 di danno in più.

\paragraph{Punti ferita ai livelli superiori}

Per quanto riguarda l'aumento dei punti ferita chiedo di usare sempre il valore atteso del dado vita e tratto come retroattivi gli aumenti del punteggio di Costituzione per il calcolo dei punti ferita.

\subsection{Il party}

\paragraph{Varietà}

Nella composizione del party chiedo di sforzarsi e mettersi d'accordo con gli altri giocatori per garantire una certa varietà in termini di classi, ruoli, razze e pesonalità, in quanto rende più interessante sia il combattimento che il roleplay.

\paragraph{Vincoli di composizione}

In particolare, chiederei di limitare i membri del party appartenenti ad una classe ad un singolo membro contemporaneamente, a meno che non si tratti di due sottoclassi che cambiano radicalmente il playstyle (ad esempio un Warlock melee e un Warlock cecchino). \\ Anche un singolo livello di multiclasse occupa lo "slot" della classe (sto guardando voi, che puntavate ad un singolo livello in \textit{Lama del Sortilegio}).

\paragraph{PVP}

Il PVP è ammesso purchè sia estremamente ben giustificato a livello di roleplay, è ammesso che i personaggi eventualmente si stacchino dal party e diventino NPC, non sono ammessi personaggi antagonisti per partito preso.

\subsection{Addio di un personaggio}

\paragraph{Morte}

Vi avverto fin dall'inizio: non sarò clemente con danni non letali e resurrezioni in generale, se un personaggio muore e non viene resuscitato prima che subentri il nuovo personaggio del giocatore, sarà da considerarsi morto. Se dovesse venire resuscitato, esisterà nella campagna come NPC controllato in condivisione dal DM e dal suo giocatore iniziale.

\paragraph{Allontanamento}

Può capitare che un giocatore perda il \textit{feeling} per un personaggio, che l'arco narrativo del personaggio si chiuda naturalmente o che per una serie di scelte gli obiettivi e i valori del personaggio si allontanino irreversibilmente da quelli del party. In questi casi, in accordo col DM, si può negoziare un'uscita di scena di questo personaggio e l'ingresso del nuovo personaggio del giocatore.\\ Si consiglia vivamente di non abusare di questa opzione.

\paragraph{Perdita del giocatore}

Infine, alle volte accade che non sia il personaggio ma il giocatore stesso ad abbandonare il party, per una serie di motivi. In base alla natura dell'assenza, il DM potrà concordare con il giocatore (se questi sarà disponibile a negoziare) il destino del suo personaggio, ma in qualsiasi caso è consigliabile optare per un allontanamento temporaneo dal party.

\subsection{Riunione col party dopo una morte o un'assenza}

\paragraph{Morte}

Ogni giocatore deve aver già pronto un personaggio da far subentrare in caso di morte del proprio per la sessione successiva. Alternativamente, in base al momento all'interno della trama, è straordinariamente possibile impersonare un personaggio "filler" per una sessione di divertimento e gaia trivialità.

\paragraph{Assenza prolungata}

Questo tipo di assenza andrebbe discussa con più attenzione e soprattutto anche con gli altri membri del party, soprattutto se dovessero essere subentrati nuovi giocatori. \\ In seguito alla riunione di un giocatore dopo un periodo di Erasmus, il suo personaggio perde temporaneamente la conoscenza della lingua Comune.

\section{Regole opzionali}

\DndDropCapLine{Q}{}ueste regole...

\subsection{Dadi esplosivi}

Un dado esplosivo è un dado che viene ritirato ogni volta che il suo risultato è pari al massimo del dado e alla fine del processo restituisce la somma delle iterazioni. \\ Ad esempio, un d4 esplosivo potrebbe restituire come serie di risultati 4, 4 e 2, quindi il risultato del tiro sarebbe 4+4+2=10.

\subsection{Automulticlasse}

Come ammetto la multiclasse, ammetto anche che un personaggio multiclassi nella sua stessa classe, magari per ricevere i privilegi di diverse sottoclassi! Al fine di conteggiare i suoi privilegi limitati tuttavia questi vanno conteggiati separatamente e secondo il livello della classe che glieli garantisce. \\ Ad esempio un Monaco di livello 10/10 potrà ridurre al massimo di 50 punti ferita il suo danno da caduta usando una reazione, mentre un Warlock di livello 11/11 avrà due utilizzi dell'Arcanum Mistico ma entrambi limitati al 6° livello. \\ Ogni ambiguità va sottoposta al giudizio del DM, ovviamente.



\part{Opzioni per i personaggi}

\nchapter{Razze}

% \section{Divorati}

% \textit{Questa razza è un estratto dalla Biblioteca Omnicomprensiva di Ker}

% \DndDropCapLine{Q}{}uando un incantatore supera i propri limiti e si spinge oltre, può accadere che il suo corpo non riesca a reggere la pressione e per non morire si aggrappi alla forza della magia. In questi casi l'incantesimo che lo salva può diventare parte di lui, e il suo corpo può divenire un canale per la magia stessa. La forma fisica dell'incantatore cambia, e il suo corpo diventa simile all'incantesimo che lo ha divorato. L'incantesimo tuttavia non agisce solo sull'aspetto del divorato, ma ne influenza anche la mente, e può portare a cambiamenti di personalità che spesso sfociano in comportamenti altrimenti inspiegabili e in certi casi può condurre persino alla follia.

% \subsection{La trasformazione}
% Un incantatore di livello inferiore al 12° non è abbastanza potente da lanciare incantesimi che possano ridurlo allo stato di divorato. Il loro corpo mantiene alcuni tratti in comune con la loro forma precedente, ad esempio la statura, ma diventa più slanciato e i suoi lineamenti divengono più affilati. La sua pelle assume un colore simile a quello dell'incantesimo che lo ha divorato, e i suoi occhi diventano vacui e profondi, a riflettere la corruzione che lo ha colpito.

% \subsection{Una razza molto rara}
% Sono pochi gli incantatori abbastanza potenti e avventati da diventare un divorato, e ancor meno sono quelli che sopravvivono alla trasformazione. Per questo motivo i divorati sono una razza molto rara, e la maggior parte di loro non ha mai incontrato un altro membro della propria specie. I divorati sono solitari per natura, e preferiscono vivere in luoghi isolati dove possono essere se stessi senza dover temere di essere giudicati.

% \subsection{Un nuovo potere...}
% La potente magia che pervade il divorato gli dona un nuovo e maggiore potere. La longevità di un divorato aumenta drasticamente, e un divorato sufficientemente potente può vivere anche per più di mille anni. Secondo alcune leggende, nascosti in luoghi remoti, ci sono alcuni divorati che lo sono diventati ancora prima della caduta del Netheril.  Un divorato ha un forte legame con il tipo di magia che lo ha divorato. I tipi dei danni inferti dal dvorato, detti da energia elementale, dipendono da questo legame.

% \subsection{...Ma a quale prezzo}
% Un divorato è un essere molto potente nelle arti magiche, ma questo potere ha un costo in termini di energie: egli può lanciare incantesimi come un qualsiasi incantatore oppure attingere all'energia che lo tiene in vita per lanciare speciali incantesimi senza consumare slot, ma facendo questo si indebolisce. La grande longevità di un divorato non gli impedisce tuttavia di morire in battaglia o di esaurire l'energia che lo alimenta. 

% \subsection{I nomi dei divorati}
% Un familiare che diviene un divorato è percepito tra molte razze come un grande
% disonore, per questo spesso i divorati si allontanano da chi era loro caro. Per evitare che il disonore cada anche sui loro familiari, molti divorati decidono di cambiare il proprio nome. Alcuni di loro scelgono un nome che rifletta la loro nuova natura, altri invece scelgono un nome che li aiuti a dimenticare il loro passato. Altri ancora preferiscono invece mantenere il proprio nome, o sceglierne uno simile, per ricordare chi erano e da dove vengono.

% \subsection{Tratti dei divorati}
% Venire divorato dalla magia è una vera e propria trascensione. Un divorato durante la trasformazione perde i suoi tratti razziali, ma acquisisce i seguenti.
% \paragraph{Vista cieca} Il divorato non può vedere come prima, ma ha un'altissima percezione della magia che lo circonda, il che gli consente di percepire l'ambiente circostante anche senza l'uso della vista.
% \paragraph{Nutrimento magico} Un divorato non si nutre come un normale essere vivente, ma attinge energia magica dall'ambiente circostante. Un divorato non ha bisogno di mangiare, bere o respirare. Il suo sonno è diverso da quello dei normali esseri viventi: il divorato entra in uno stato di trance in cui è semicosciente ed entra in contatto con la forza che lo ha divorato, attingendo alla magia per recuperare tutte le energie di cui ha bisogno. 4 ore di trance sono sufficienti per un divorato a effettuare un riposo lungo. Se un divorato rimane circa una settimana nel raggio di 1,5 m di un oggetto magico che non sia esplicitamente progettato per resistervi, esso perde permanentemente le sue proprietà magiche.
% \paragraph{Linguaggi} Il divorato mantiene la conoscenza delle lingue che aveva appreso prima di trasformarsi. Potrebbe accadere che, durante la trasformazione, un divorato perda la capacità di comprendere il comune.
% \paragraph{Velocità} La velocità base sul terreno di un divorato è di 9 metri.
% \paragraph{Difesa senza armatura} Un divorato non può indossare armature, ma la sua CA è pari a 10 + il suo modificatore di Intelligenza.
% \paragraph{Incremento del Punteggi di Caratteristica} Un punteggio a scelta tra Saggezza, Intelligenza e Carisma di un divorato aumenta di 2, fino a un massimo di 20. Un divorato perde eventuali incrementi dei punteggi di caratteristica dovuti alla sua razza prima della trasformazione.

% \subsection{Divorati elementali}
% Essi costituiscono la maggior parte dei divorati: quando un incantatore attinge all'energia elementale e diviene un divorato, il suo corpo diventa una manifestazione dell'elemento stesso Manifesta quindi una forte connessione con l'elemento di cui ora il suo corpo è fatto. Un divorato dell'acqua, ad esempio, potrebbe avere la pelle blu e gli occhi azzurri, mentre un divorato del fuoco potrebbe avere la pelle rossa e gli occhi gialli.
% \paragraph{Resistenza elementale} Un divorato elementale ha resistenza a un tipo di danno a scelta tra acido, freddo, fulmine, fuoco e tuono, che dipende dall'elemento cui è legato.
% \paragraph{Timore reverenziale} Un divorato elementale ha competenza nei tiri per intimidire.
% \paragraph{Velocità} La velocità base sul terreno di un divorato è di 7,5 metri.
% \paragraph{Affinità elementale}Un divorato elementale conosce l'incantesimo \textit{Scarica Elementale} e, un numero di volte pari al suo bonus di competenza per riposo lungo, può lanciarlo senza consumare slot. Il tipo di danni inferti dall'incantesimo dipende dall'elemento del divorato. Quando lo fa, subisce 2d4+2 danni puri.
% \paragraph{Morte violenta} Se un divorato muore per il contraccolpo di un proprio incantesimo, il suo corpo si dissolve nell'ambiente circostante, mentre se muore per altri motivi, il suo corpo esplode in una nube di energia elementale. In questo caso tutti coloro che si trovano entro 6 metri dal divorato devono effettuare un tiro salvezza su Destrezza. Se lo falliscono, subiscono 8d6 danni da energia elementale, mentre se lo superano, subiscono soltanto la metà di quei danni.

% \subsection{Divorati del Caos}
% Secondo antiche leggende, esisteva un altro tipo di magia, oggi quasi completamente andato perduto: la Magia del Caos. Solo pochi incantatori di straordinario potere erano in grado di padroneggiarla, e ancor meno erano abbastanza avventati da utilizzarla. La Magia del Caos era estremamente potente, ma anche molto pericolosa, e spesso gli incantatori che la utilizzavano perdevano il controllo e venivano divorati da essa. Molte persone credono che i Divorati del Caos, come la magia che li alimenta, siano solo un mito diffuso per spaventare i bambini. L'aspetto di un divorato del Caos è ingrado di incutere timore anche nei più coraggiosi: la sua pelle è del colore del vuoto più profondo e i suoi occhi brillano di una luce violastra.
% \paragraph{Resistenza magica} Un divorato del Caos ha resistenza ai danni da contundenti, perforanti, da taglio provocati da armi magiche e da forza.
% \paragraph{Velocità} La velocità base sul terreno di un divorato è di 10,5 metri.
% \paragraph{Aspetto del Caos} Un divorato del Caos ha maestria nei tiri per intimidire.
% \paragraph{Tutt'uno con la magia} Un divorato del Caos ha maestria nei tiri su arcano.
% \paragraph{Incantesimi del Caos}Un divorato del Caos conosce l'incantesimo \textit{Punizione del Caos} e, un numero di volte pari al suo bonus di competenza, può lanciarlo senza consumare slot. Quando lo fa, subisce 3d4+2 danni puri. Inoltre un divorato del caos conosce il trucchetto \textit{Deflagrazione Occulta}, ma ogni volta che lo lancia subisce 1 danno puro.
% \paragraph{Buco Nero} Quando un divorato del caos muore, genera una scarica di energia caotica che apre un portale verso il Piano Astrale nel punto in cui si trova. Qualsiasi creatura entro 3 metri dal portale viene risucchiata al suo interno e ricompare in un posto a caso sul Piano Astrale, poi il portale si richiude. Il portale è a senso unico e non può essere riaperto.

\section{Forgiati}

\DndDropCapLine{N}{}on c'è un'ambientazione fantasy che sia completa senza una vasta selezione di amiconi artificiali, del resto chi può dire di aver visto Guerre Stellari senza essersi innamorato di C3PO o di R2D2? Evidentemente poche persone, dato che internet è assolutamente pieno di varie versioni dei Forgiati. \\ Mancando tuttavia una versione ufficiale nei manuali di base di quinta edizione o in Tasha e Xanathar, ho deciso di sopperire con la mia versione, a voi!

\subsection{Tratti dei forgiati}
\paragraph{Aumento del punteggio di caratteristica} Il punteggio di Costituzione di un forgiato aumenta di 2.
\paragraph{Difesa senza Armatura} Un forgiato non può indossare armature, ma la sua CA è pari a 8 + il suo modificatore di Destrezza + il suo bonus di competenza.
\paragraph{Velocità} La velocità base sul terreno di un forgiato è di 9 metri.
\paragraph{Resistenza al Veleno} Un forgiato ha resistenza ai danni da veleno e ai tiri salvezza contro il veleno.
\paragraph{Radar} Un forgiato possiede scurovisione nel raggio di 18 m.

\subsection{Forgiati da combattimento}
\paragraph{Aumento del punteggio di caratteristica} Il punteggio di Destrezza di un forgiato da combattimento aumenta di 1.

\subsection{Forgiati da esplorazione}
\paragraph{Aumento del punteggio di caratteristica} Il punteggio di Intelligenza di un forgiato da esplorazione aumenta di 1.
\paragraph{Scanner magico} Un forgiato da esplorazione conosce l'incantesimo \textit{Individuazione del Magico} e può lanciarlo un numero di volte pari al suo bonus di competenza senza usare slot per ogni riposo lungo.
\paragraph{Velocità} La velocità base sul terreno di un forgiato da esplorazione è di 10,5 metri.

\subsection{Forgiati da costruzione}
\paragraph{Aumento del punteggio di caratteristica} Il punteggio di Forza di un forgiato da costruzione aumenta di 1.

\nchapter{Artefice}

\section{Privilegi di Classe Opzionali}



\chapter{Barbaro}

\section{Privilegi di Classe Opzionali}


\nchapter{Bardo}

\section{Privilegi di Classe Opzionali}

\section{Collegio Bardico: Collegio dei Ludopatici Anonimi}

\subsection{Antinormalità}

Ogni volta che il bardo effettua una prova di caratteristica, un tiro salvezza o un tiro per colpire, tira due volte e deve usare il risultato più lontano da 10; se il risultato del dado è minore di 10, sottrae il suo modificatore, mentre se è maggiore di 10 lo aggiunge normalmente.

\section{Collegio Bardico: Collegio del Primo Contatto}

\begin{DndReadAloud}

\end{DndReadAloud}

\subsection{}


\nchapter{Chierico}

\section{Privilegi di Classe Opzionali}

\section{Dominio Divino: Dominio di Integrità}

\begin{DndReadAloud}
  \it
  "E questo è interessante. Perchè è interessante? Perchè l'ho scritto, e io scrivo solo cose interessanti." \\ (W.A. De Graaf)
\end{DndReadAloud}

\subsection{Incantesimi di dominio}

\begin{DndTable}{XX}
  Livello da Chierico  & Incantesimi \\
  1° & Armatura di Agathys, Comando\\
  3° & Frantumare, Immagine Speculare\\
  5° & Aura di Vitalità, Controincantesimo\\
  7° & Inaridire, Occhio Arcano\\
  9° & Mano di Bigby, Reincarnazione\\
\end{DndTable}

\subsection{Competenze bonus}

Al 1° livello, il Chierico ottiene competenza nelle seguenti abilità: Arcano e Intuizione. \\ Quando effettua una prova di caratteristica su queste abilità raddoppia il suo bonus di competenza.

\subsection{Forza brutta}

A partire dal 1° livello, il Chierico può scegliere di usare il suo modificatore di Intelligenza invece che di Saggezza per ogni prova di caratteristica, tiro per colpire o tiro salvezza che lo richieda.\\ Allo stesso modo, la sua caratteristica da incantatore diventa Intelligenza.

\subsection{Incanalare Divinità: Divisone con resto}

A partire dal 2° livello il Chierico può spendere un suo utilizzo di \textit{incanalare divinità} per effettuare la divisione con resto su dei danni subiti da un suo alleato. \\ I danni vengono divisi per tre e inflitti equamente al Chierico, all'alleato e alla creatura che ha inflitto i danni. \\ I tre poi devono effettuare una prova contrapposta di Intelligenza pura, e chi ottiene il risultato più basso riceve il resto dei danni.

\subsection{Incanalare Divinità: Permutazione}

A partire dal 6° livello il Chierico può spendere un suo utilizzo di \textit{incanalare divinità} per scegliere un numero di creature pari o inferiore al suo livello da Chierico (almeno 2) entro 9 metri di cui conosca le posizioni e permutarle a suo piacimento. \\ Se una creatura viene permutata in uno spazio che non può occupare, viene automaticamente teletrasportata nello spazio sicuro che possa occupare più vicino.

\subsection{Buon ordinamento}

A partire dall'8° livello, quando il Chierico tira per iniziativa, può scegliere l'ultima creatura alleata nell'ordine di iniziativa, sè stesso incluso, e decidere quando agirà all'inizio di ogni round.

\subsection{Fattorizzazione unica}

A partire dal 17° livello, una volta per riposo lungo, il Chierico può usare la sua azione per scegliere una creatura che sia in grado di vedere e conoscerne i punti ferita attuali. \\ Quando lo fa, deve scegliere uno dei fattori primi del numero di punti ferita attuali della creatura: questa viene divisa in quel numero di copie più piccole di sè stessa e i suoi punti ferita vengono divisi equamente tra di esse. \\ Le copie possiedono le stesse statistiche e abilità della creatura di partenza, ma non possono effettuare azioni o azioni bonus, soltanto reazioni e movimenti.\\ All'inizio del turno successivo del Chierico, le copie vengono rifuse nella creatura iniziale, la quale riappare nello spazio in cui si trovava prima di essere divisa o nel più vicino spazio libero e subisce i danni subiti da ognuna delle singole copie. \\ Se il numero di punti ferita della creatura è un numero primo, il Chierico ottiene un utilizzo aggiuntivo di \textit{incanalare divinità}. \\ Il DM sceglie la taglia delle copie (minore o uguale alla taglia della creatura originale) e la loro posizione.


\nchapter{Druido}

\section{Privilegi di Classe Opzionali}

\section{Circolo Druidico: Circolo Sperimentale}

\begin{DndReadAloud}
  \it
  "Siamo a Fisica, quantifichi!" \\ (R. Dolesi)
\end{DndReadAloud}

\subsection{Un caffè buonissimo, qualcosa di bello da mangiare}


\chapter{Guerriero}

\section{Privilegi di Classe Opzionali}

\section{Re dei Cavalieri}

\subsection{Acciaio e Vento}

A partire dal 3° livello, il Guerriero ottiene un'arma leggendaria, \textit{Acciaio e Vento}. \\ Si tratta di una spada lunga (versatile), il cui danno aumenta nel tempo in base al livello da Guerriero del suo possessore. Si presenta come una spada invisibile avvolta da delle forti correnti d'aria che condensando il vapore acqueo nell'atmosfera permettono di distinguerla. \\ Come azione gratuita, il Guerriero può congedare \textit{Acciaio e Vento} in un semipiano accessibile solo ad essa e richiamarla a sè, ovunque si trovi. \\ Una volta per riposo lungo, il Guerriero può decidere di usare la sua azione bonus per rivelare la vera forma di \textit{Acciaio e Vento} per 1 minuto, aggiungendo dei danni radiosi ai colpi effettuati con essa.
\paragraph{CD del tiro salvezza} Alcuni effetti di \textit{Acciaio e Vento} impongono ai bersagli un tiro salvezza. La CD di questo tiro è pari a 8 + il modificatore di Forza del Guerriero + il bonus di competenza del Guerriero.

\begin{DndTable}[header=Acciaio e vento]{XXXX}
    Livello da Guerriero & Danni & Danni radiosi extra & Velocità bonus \\
    3°-6° & 1d8 (1d10) & 1d6 & --\\
    7°-9° & 1d10 (1d12) & 2d6 & 1,5 m\\
    10°-14° & 2d6 (2d8) & 3d6 & 3 m\\
    15°-17° & 2d8 (2d10) & 4d6 & 4,5 m\\
    18°-20° & 2d10 (2d12) & 5d6 & 6 m\\
\end{DndTable}

\subsection{Parata impetuosa}
A partire dal 7° livello, se il Guerriero impugna \textit{Acciaio e Vento} può lanciare l'incantesimo \textit{Scudo} con essa un numero di volte pari al suo bonus di competenza per riposo breve. Se para l'attacco di un nemico usando questo privilegio, la velocità di movimento del Guerriero aumenta fino alla fine del suo prossimo turno come indicato sulla tabella.

\subsection{Aura regale}
A partire dal 10° livello, mentre \textit{Acciaio e Vento} è rivelata, tutte le creature alleate del Guerriero nel raggio di 9 m da essa hanno vantaggio nei tiri salvezza, un bonus pari al modificatore di Forza del Guerriero alle prove di caratteristica e un bonus pari al bonus di competenza del Guerriero ai danni effettuati con attacchi corpo a corpo. \\ Inoltre, i danni inflitti da \textit{Acciaio e Vento} e tutte le creature alleate nell'area sono considerati magici ai fini di ignorare le resistenze.

\subsection{Rocca della Tavola Rotonda}
A partire dal 15° livello, una volta per riposo lungo, il Guerriero può piantare a terra \textit{Acciaio e Vento} e lanciare l'incantesimo \textit{Reggia Meravigliosa di Mordenkainen} senza spendere componenti materiali. \\ L'incantesimo permane finchè la spada è piantata a terra, e il Guerriero è l'unico che la può estrarre.

\subsection{Miracolo del Re}
A partire dal 18° livello, alla fine del suo turno il Guerriero può usare la sua azione bonus per cominciare a caricare \textit{Acciaio e Vento} di energia (purchè sia già rivelata). Dall'inizio del suo turno successivo, può usare la sua azione per liberarla di fronte a sè. \\ Tutte le creature che si trovano in un'area larga 3 metri e lunga 27 metri di fronte al Guerriero devono effettuare un tiro salvezza su Costituzione. Se lo falliscono subiscono 12d8 danni radiosi e sono accecati fino all'inizio del turno successivo del Guerriero, altrimenti subiscono solo la metà dei danni. Le creature di allineamento malvagio hanno svantaggio al tiro, mentre le creature di allineamento buono hanno vantaggio. \\ Dopo aver usato questo privilegio, \textit{Acciaio e Vento} rimane rivelata fino al successivo riposo breve o lungo. Dopo aver usato questo privilegio, non può più essere usato fino al successivo riposo lungo.

\chapter{Ladro}

\begin{DndReadAloud}
  \it
  "E lei che fa nella vita?" \\ L'halfling, con una naturalezza e un'innocenza disarmanti, sorrise ed esclamò: "Io rubo!"
\end{DndReadAloud}

\section{Privilegi di Classe Opzionali}

\subsection{Introvabile persino per il DM}

A partire dal 1° livello, se nella narrazione non è specificato che stia facendo qualcos'altro, il Ladro può effettuare una prova di Furtività con una CD di 35. \\ Se la supera (o ottiene un 20 naturale), si materializza nella scena attualmente narrata ed è considerato nascosto da tutte le creature coscienti. \\ Se lo ritiene opportuno e il giocatore non è presente, il DM può effettuare questa prova al posto del giocatore.

\section{Il Camminatore delle Ombre}

\begin{DndReadAloud}
  \it
  "Beh, c'è chi dice che è morto. Baggianate, così penso io. No! Per me è ancora in circolazione..." \\ (Harry Potter e la Pietra Filosofale, 2001)
\end{DndReadAloud}

\subsection{CD del tiro salvezza}

Alcune abilità del Camminatore delle Ombre impongono ai bersagli dei tiri salvezza oppure gli garantiscono la possibilità di lanciare incantesimi che impongono tiri salvezza. La CD di questi tiri è 8 + il bonus di competenza del Ladro + il suo modificatore di Destrezza.

\subsection{Adottato dall'Ombra}

A partire dal 3° livello, il Ladro acquisisce \textit{Scurovisione} e la capacità di vedere nell'oscurità magica, se non le possiede già.

\subsection{Lampo nel Buio}

A partire dal 3° livello, quando il Ladro colpisce una creatura col suo attacco furtivo e lui stesso si trova in uno spazio in condizioni di luce fioca o oscurità, questa deve superare un TS su Saggezza. Se lo fallisce, subisce altri danni radiosi ed è accecata fino all'inizio del turno successivo del ladro. \\ I danni radiosi subiti sono 1d6 per ogni 2d6 lanciati dal Ladro per l'attacco furtivo.

\subsection{Passo nell'Ombra}

A partire dal 9° livello, il Ladro può usare la sua Azione Scaltra per teletrasportarsi nell'ombra di una creatura che sia in grado di vedere. \\ Se il Ladro era nascosto alla creatura lo rimane, altrimenti può tirare per una prova di Furtività per nascondersi nell'ombra della creatura.

\subsection{Cortina di Fumo!}

A partire dal 13° livello, il Ladro può usare la sua Azione Scaltra per lanciare il trucchetto \textit{Illusione Minore} e gli incantesimi \textit{Camuffare Sè Stesso}, \textit{Movimenti del Ragno} e \textit{Oscurità} a volontà. \\ Inoltre, quando si trova in condizioni di luce fioca o oscurità, può usare la sua Azione Scaltra per lanciare l'incantesimo \textit{Invisibilità Superiore} su sè stesso a volontà.

\subsection{Intoccabile come un'Ombra}

A partire dal 17° livello, il Ladro quando si trova in condizioni di luce fioca o oscurità supera automaticamente tutti i tiri salvezza su Destrezza. \\ Inoltre il suo punteggio di Destrezza aumenta di 4, e il valore massimo del suo punteggio di Destrezza è 24.

\subsection{Nemmeno Omatara potrebbe trovarlo}

A partire dal 17° livello, se i punti ferita del Ladro scendono a 0 mentre si trova in condizioni di luce fioca o oscurità, questi rimane a 1 punto ferita e viene considerato automaticamente nascosto da tutte le creature coscienti, alleati compresi. Questo privilegio può essere utilizzato una sola volta per riposo lungo.

\chapter{Mago}

\section{Privilegi di Classe Opzionali}

\subsection{Controcanto della Lama}

Dal 6° livello un Cantore della Lama, quando un nemico fallisce un attacco da mischia contro di lui, può usare la sua reazione per contrattaccare con la propria arma senza aggiungere il modificatore di caratteristica al danno. È comunque necessario il tiro per colpire.

\section{Il Criminale di Guerra}

\begin{DndReadAloud}
    \it
    Mi piace l'odore del napalm al mattino. Una volta abbiamo bombardato una collina, per dodici ore, e finita l'azione siamo andati a vedere. Non c'era più neanche l'ombra di quegli sporchi bastardi. Ma quell'odore... sai quell'odore di benzina? Tutto intorno. Profumava come... come di vittoria.\\ (Apocalypse Now, 1979)
\end{DndReadAloud}

\subsection{livello 2}
A partire dal 2° livello, 

\subsection{livello 6}
A partire dal 6° livello, il Mago 

\subsection{Bombardamento a Tappeto}
A partire dal 10° livello, prima di lanciare un incantesimo a distanza con un'area d'effetto circolare, il Mago può usare la sua azione bonus per chiamare un bombardamento a tappeto. Una volta che lo fa, lancia l'incantesimo e questo viene lanciato 2d6 volte in punti casuali determinati dal DM.\\ Una volta che questo privilegio è stato utilizzato, non può più essere utilizzato fino al prossimo riposo lungo.

\subsection{Cupola di Ferro}
A partire dal 14° livello, una volta per riposo lungo, il Mago può usare la sua azione per spendere uno slot di 5° livello o superiore ed entrare in concentrazione e azzerare la sua velocità di movimento. \\ Fino alla fine della sua concentrazione, ogni proiettile generato fuori da una sfera di raggio 18 m e centro nel Mago e diretto verso l'interno di questa sfera, viene intercettato da dei \textit{Dardi Incantati} partiti dal Mago non appena entra nella sfera. \\ La CA di ogni creatura all'interno della sfera contro attacchi a distanza provenienti dall'esterno della sfera riceve un bonus pari al livello dello slot utilizzato, come il tiro salvezza contro proiettili ad area d'effetto circolare provenienti dall'esterno della sfera. \\ Gli incantesimi \textit{Dardo Incantato} provenienti dall'esterno della sfera e diretti verso l'interno non hanno effetto.

\nchapter{Monaco}

\begin{DndReadAloud}
  \it
  "Li vedi questi?" il Monaco cieco mostrò i suoi pugni, splendenti di energia verde, al povero borseggiatore colto in flagrante. \\ "Questo è Porta e questo è Fogli."
\end{DndReadAloud}

\section{Privilegi di Classe Opzionali}

\subsection{Riserva di ki aumentata}

Il numero massimo di punti ki di un Monaco è pari al suo livello + il suo bonus di competenza.

\subsection{Arti Marziali ampliate}

Un Monaco competente può usare un punto ki per aggiungere il suo modificatore di Destrezza alle prove di caratteristica di Atletica per afferrare una creatura.

\subsection{Volontà infrangibile}

La volontà dei Monaci più esperti è talmente ferrea da poter superare ogni barriera mortale. A partire dal 14° livello, quando muore in combattimento, il ki di un Monaco può essere completamente liberato, consentendogli di rialzarsi per continuare a combattere.\\ Quando un Monaco fallisce il suo ultimo tiro salvezza contro morte, può rialzarsi recuperando tutti i suoi punti ki e agire nello stesso turno, con le seguenti condizioni: \\
\paragraph{Aura bruciante} All'inizio di ogni suo turno, il Monaco tira il suo dado da Monaco e perde quel numero di punti ki. Quando li ha esauriti, muore e non può essere resuscitato se non dall'incantesimo \textit{Desiderio} o dall'intervento di una divinità. Per via del calore immenso generato dalla sua aura che si consuma, ogni contatto con il Monaco procura 1d4 danni da fuoco.
\paragraph{Furia rediviva} Mentre si trova in questo stato, il Monaco è immune a ogni danno e supera automaticamente ogni tiro salvezza contro le influenze mentali. Durante il suo turno è costretto a seguire il corso d'azioni che più si avvicini alle sue ultime volontà, senza possibilità di cambiare idea.
\paragraph{Perdita di controllo} Mentre si trova in questo stato, il Monaco non può spendere nessun punto ki, il suo bonus di competenza diventa pari a 0 e non può comunicare in nessun linguaggio.
\paragraph{Potenza sovraumana} Mentre il Monaco si trova in questo stato, il bonus alla sua velocità di movimento di \textit{Movimento senza armatura} raddoppia, possiede vantaggio ai tiri salvezza su Forza, Destrezza e Costituzione, ogni sua azione di attacco gli garantisce tre attacchi e può usare un'azione bonus per effettuarne altri due.

\section{Tradizione Monastica: Via dell'Ombra Redenta}

\subsection{Pentimento e Redenzione}

\DndDropCapLine{S}{}ebbene tutti i monaci condividano la saggezza degli antichi maestri, ci sono tradizioni che si allontanano dalla loro illuminazione: è questo il caso della Via dell'Ombra. \\ Talvolta però accade che un Monaco, dopo aver sfiorato la morte ad esempio, possa decidere di allontanarsi da queste pratiche e riscoprire la dimensione più meditativa, pietosa e naturale delle arti del ki.

\subsection{Magia del ki}

Tra i molti privilegi concessi dalla via dell'Ombra, vi è quello di lanciare alcuni incantesimi, ma si tratta di un potere oppresso e incompleto. Una volta che un Monaco si libera dall'influenza dell'oscurità, può finalmente imparare a incanalare il suo ki per lanciare incantesimi più raffinati.

\paragraph{Trucchetti} Un Monaco Redento conosce due trucchetti a sua scelta tratti dalla lista degli incantesimi del druido. Apprende un trucchetto da druido aggiuntivo a sua scelta al 10° livello.

\paragraph{Slot incantesimo}La tabella indica quanti slot incantesimo possiede un Monaco Redento per lanciare i suoi incantesimi di 1° livello e di livello superiore. \\ Per lanciare uno di questi incantesimi, il Monaco deve spendere uno slot incantesimo di livello pari o superiore al livello dell'incantesimo. Per esempio, se un Monaco ha preparato l'incantesimo di 1° livello \textit{Scudo} e possiede uno slot incantesimo di 1° livello e uno slot incantesimo di 2° livello, può lanciare \textit{Scudo} usando uno qualsiasi dei due slot. \\ Il Monaco recupera tutti gli slot incantesimo spesi quando completa un riposo lungo.

\paragraph{Incantesimi preparati di 1° livello e superiore}Un Monaco Redento prepara tre incantesimi di 1° livello a sua scelta dalla lista di incantesimi del druido. La colonna "Incantesimi Preparati" nella tabella indica quanti incantesimi di 1° livello o di livello superiore un Monaco può preparare. Ognuno di questi incantesimi deve appartenere a un livello di cui il Monaco possiede degli slot incantesimo.\\ Ogni volta che il Monaco completa un riposo lungo, può preparare incantesimi diversi della lista degli incantesimi del druido. Tutti gli incantesimi preparati devono essere di un livello di cui il Monaco possiede almeno uno slot incantesimo.

\paragraph{Caratteristica da incantatore}Saggezza è la caratteristica da incantatore usata per gli incantesimi da Monaco Redento. Il potere dei suoi incantesimi deriva dalla sua sintonia con il Ki intorno a lui. Un Monaco usa Saggezza ogni volta che un incantesimo fa riferimento alla sua caratteristica da incantatore. Usa inoltre il suo modificatore di Saggezza per definire la CD del tiro salvezza di un incantesimo da Monaco da lui lanciato e quando effettua un tiro per colpire con un incantesimo.

\paragraph{CD del tiro salvezza degli incantesimi}= 8 + il bonus di competenza del Monaco + il modificatore di Saggezza del Monaco. 

\paragraph{Modificatore di attacco dell'incantesimo}= il bonus di competenza del Monaco + il modificatore di Saggezza del Monaco

\begin{DndTable}[header= Slot Incantesimo per livello]{XXXXXXXXXX}
  Livello da Monaco & & Trucchetti conosciuti & & Incantesimi preparati & & 1° Lv. & 2° Lv. & 3° Lv. & 4° Lv.\\
  3° & & 2 & & 3 & & 2 & - & - & - \\
  4° & & 2 & & 4 & & 3 & - & - & - \\
  5° & & 2 & & 4 & & 3 & - & - & - \\
  6° & & 2 & & 4 & & 3 & - & - & - \\
  7° & & 2 & & 5 & & 4 & 2 & - & - \\
  8° & & 2 & & 6 & & 4 & 2 & - & - \\
  9° & & 2 & & 6 & & 4 & 2 & - & - \\
  10° & & 3 & & 7 & & 4 & 3 & - & - \\
  11° & & 3 & & 8 & & 4 & 3 & - & - \\
  12° & & 3 & & 8 & & 4 & 3 & - & - \\
  13° & & 3 & & 9 & & 4 & 3 & 2 & - \\
  14° & & 3 & & 10 & & 4 & 3 & 2 & - \\
  15° & & 3 & & 10 & & 4 & 3 & 2 & - \\
  16° & & 3 & & 11 & & 4 & 3 & 3 & - \\
  17° & & 3 & & 11 & & 4 & 3 & 3 & - \\
  18° & & 3 & & 11 & & 4 & 3 & 3 & - \\
  19° & & 3 & & 12 & & 4 & 3 & 3 & 1 \\
  20° & & 3 & & 13 & & 4 & 3 & 3 & 1 \\
\end{DndTable}

\subsection{Aura viva}

A partire dal 3° livello, il ki di un Monaco Redento si manifesta visualmente come un'aura intorno a lui, che sia cosciente o incosciente. Questa funge da fonte di luce fioca per tutte le creature non ostili. \\ Tutti i danni inferti dal Monaco a creature viventi sono danni non letali.

\subsection{Auree nell'oscurità}

A partire dal 6° livello, il Monaco perde la sua visione. Guadagna la capacità di vedere le auree delle creature viventi con la sua mente e può ricavarne diverse informazioni, come l'allineamento, i punti ferita, la capacità di lanciare incantesimi e la forza magica. Tuttavia grazie alla sua sintonia con il mondo circostante, guadagna vista cieca nel raggio di 18m. \\ A causa della sua cecità, quando usa il privilegio "Deviare i proiettili" il Monaco non può più rilanciare indietro i proiettili neutralizzati.
\paragraph{Consigli per il DM}È consigliabile comunicare al giocatore queste informazioni in modo implicito, tramite ad esempio una descrizione dell'aura della creatura, piuttosto che dargliele direttamente.

\subsection{Nulla si crea, nulla si distrugge, tutto si trasforma}

A partire dall'11° livello, una volta per riposo lungo, quando il Monaco vede l'aura di una creatura vivente scendere a 0 punti ferita, può spendere un turno in concentrazione per entrare in sintonia con il suo corpo e assorbire da essa un numero di punti ki pari al suo bonus di competenza, ma per ogni punto ki recuperato in questo modo perde lui stesso due punti ferita. \\ In qualunque momento, il Monaco può convertire i suoi punti ki in slot incantesimo temporanei e viceversa, con un tasso di un punto ki per livello (purchè possieda già uno slot di quel livello).

\subsection{Sacrificio assoluto}

A partire dal 16° livello, con un atto altruistico definitivo, il Monaco può attingere a tutto il suo ki per poi farlo esplodere con un'emanazione del raggio di 15 metri. Tutti gli alleati morti che siano all'interno dell'emanazione sono riportati in vita, come se fossero soggetti ad un incantesimo \textit{Resurrezione Pura}. \\ Il Monaco viene completamente distrutto. Un Monaco distrutto in tal modo non può più tornare in vita, nemmeno tramite un incantesimo \textit{Desiderio} o \textit{Miracolo} o grazie al potere di una divinità. \\ Inoltre, il nome del Monaco può essere pronunciato ma non potrà più essere scritto. Tutti i riferimenti scritti del suo nome divengono nient'altro che spazi bianchi e tutti gli oggetti magici con cui era in sintonia perdono ogni proprietà magica. \\ Tutto ciò che rimane del Monaco dopo aver usato questo privilegio è un sacchetto di semi di \textit{Principessa Serena} con su cucita l'unica testimonianza scritta del suo nome.



\chapter{Paladino}

\section{Privilegi di Classe Opzionali}

\subsection{Imposizione delle mani}
Invece di avere una riserva di punti ferita pari a cinque volte il suo livello da Paladino, il Paladino possiede una riserva di punti ferita pari al suo livello da Paladino in d6, che tira al termine di ogni riposo lungo.

\section{Giuramento di Veramente Vera Verità}

\textit{In accordo alla sua natura, questa sottoclasse è stata generata da un'intelligenza artificiale e ovviamente corretta dal sottoscritto}

\begin{DndReadAloud}
  \it
  \begin{math}[a = A] \Rightarrow [a = (-1)^{2a^2}(A-1)+1]\end{math} \ "Banale aritmetica" \\ (R. Zunino)
\end{DndReadAloud}

\begin{DndTable}{XX}
  Livello da Paladino & Incantesimi \\
  3°  & \\
  5°  & \\
  9°  & \\
  13° & \\
  17° & \\
\end{DndTable}

\subsection{Incanalare Divinità}
Quando il Paladino acquisisce questo giuramento al 3° livello, ottiene le seguenti opzioni di Incanalare Divinità:
\paragraph{Circuiti Infallibili}A partire dal 3° livello, il Paladino può usare un'azione per intessere circuiti logici invisibili attorno a sé. Questi circuiti durano 1 ora, garantiscono un bonus pari al suo modificatore di Carisma (almeno +1) alle sue prove di Percezione e lo rendono immune al fascino e alle illusioni magiche. Il Paladino ottiene vantaggio ai tiri salvezza contro incantesimi e abilità che causano confusione o manipolazione mentale.
\paragraph{Inferenza Logica} Il Paladino può passare un minuto a concentrarsi per comporre una dimostrazione ed ottenere la conferma di verità o falsità assoluta di una proposizione a sua scelta.

\subsection{Esecuzione Deterministica}
A partire dal 7° livello, il Paladino è in grado di agire con precisione e rapidità estrema, seguendo una sequenza di azioni predeterminate. \\ Una volta al giorno, può dichiarare di eseguire una serie di azioni con una precisione quasi perfetta, ottenendo vantaggio su tutti i tiri per colpire e un bonus pari al suo modificatore di Carisma (almeno +1) a ogni istanza dei danni da lui inflitti per il suo prossimo turno.

\subsection{Assioma Protettivo}
A partire dal 15° livello, il Paladino può creare un campo di forza logico intorno a sé e ai suoi alleati che dura 10 minuti. Quando un alleato all'interno del raggio di 30 metri viene colpito da un attacco, il paladino può reagire istantaneamente e spendere uno slot incantesimo per ridurre i danni subiti da quell'alleato. Il danno viene ridotto di una quantità pari al suo bonus di Carisma + il suo bonus di competenza + il livello dello slot speso.

\subsection{Teorema di Correttezza}
Al raggiungimento del 20° livello, il paladino raggiunge la massima padronanza della logica costruttivista. \\ Una volta per riposo lungo, può lanciare l'incantesimo \textit{Verità Assoluta} senza consumare uno slot incantesimo. \\ Inoltre, l'opzione di Incanalare Divinità \textit{Inferenza Logica} viene potenziata, permettendo al Paladino di derivare un'informazione corretta che non conosce.

\chapter{Ranger}

\section{Privilegi di Classe Opzionali}



\chapter{Stregone}

\section{Privilegi di Classe Opzionali}

\section{Dinastia di Analisti}



\chapter{Warlock}

\begin{DndReadAloud}
  \it
  "Uh" il giovane Aiden osservò i tre Galeb Duhr che lo circondavano e sparì con un sonoro "Pop". \\ Prima che i tre elementali potessero capire cosa fosse successo, il braccio di uno dei Guardiani di Pietra esplose in un lampo di energia arancione brillante.
\end{DndReadAloud}

\section{Privilegi di Classe Opzionali}

\subsection{Suppliche Occulte}

\subsubsection{Occhi Aperti}
\textit{Prerequisito: 18° livello} \\ Il potere del patrono, apre gli occhi al Warlock, che potrebbe non essere più in grado di richiuderli: ottiene \textit{vista pura}, non può essere limitata o disattivata se non rimuovendo questa supplica o accecandosi volontariamente. \\
La costante esposizione al Piano Etereo è estremamente impegnativa anche per la mente di incantatori molto esperti, quindi il Warlock ha bisogno di tempo ed esperienza per abituarsi.
\paragraph{Dopo nessun aumento di livello}Il Warlock ha costanti emicranee di intensità variabile in base alla presenza di auree molto potenti; in presenza di creature la cui forma nel Piano Etereo risulterebbe sconvolgente o disturbante (a discrezione del DM), il Warlock ha svantaggio ai TS per mantenere la concentrazione e ogni attacco inflitto da tali creature provoca 1d4-1 danni psichici aggiuntivi.
\paragraph{Dopo un aumento di livello}Il Warlock si sta abituando alla forma del mondo sul Piano Etereo. Non subisce più danni aggiuntivi e non ha più svantaggio ai TS per mantenere la concentrazione, a meno che questa non sia disturbata direttamente dalle creature dall'aspetto più disturbante.
\paragraph{Dopo due aumenti di livello o al 20° livello} Ormai la \textit{vista pura} è l'unica vista che il Warlock conosca, il Piano Etereo e il Piano Materiale per lui sono assolutamente inseparabili. La \textit{vista pura} non gli causa più nessun problema. Da questo momento in poi, rimuovere questa supplica gli causerebbe cecità permanente, incurabile se non da altre sorgenti di \textit{vista pura} (come \textit{Visione del Vero}). \\
Riottenere questa supplica dopo averla rimossa richiede di riabituarsi a essa.

\subsubsection{Deflagrazione Aleatoria}
\textit{Prerequisito: trucchetto Deflagrazione Occulta} \\
Un numero di volte pari al suo bonus di competenza per riposo breve, quando il Warlock colpisce un bersaglio con il trucchetto Deflagrazione Occulta, come azione bonus può imporre al bersaglio di tirare dalla tabella della Magia Selvaggia un numero di volte pari al numero di raggi da cui è stato colpito.

\section{Lo Spazio Proiettivo}

\begin{DndReadAloud}
  \it
  "E quando l'ho visto, ragazzi, ho esclamato 'Ma questa è Geometria Pura!'" \\ (M. Andreatta)
\end{DndReadAloud}

\DndDropCapLine{G}{}irano voci su una certa entità... Un oggetto legato alla stessa natura dei piani dove ci muoviamo... coloro che ne hanno ricevuto l'intuizione la descrivono come "una ganzata pazzesca" o "roba da Harry Potter", perdono l'uso della ragione e vanno in giro a importunare i passanti con frasi terrificanti come "Ragazzi, avete della Geometria?"... roba da brividi...

\subsection{Lista degli incantesimi ampliata}

\begin{DndTable}{XX}
  Livello dell'incantesimo  & Incantesimi \\
  1°  &  Caduta Morbida, Colpo Intrappolante\\
  2°  &  Levitazione, Zona di Verità\\
  3°  &  Glifo di Interdizione, Lentezza\\
  4°  &  Divinazione, Sfera Elastica di Otiluke\\
  5°  &  Cerchio di Teletrasporto, Legame Planare\\
\end{DndTable}

\subsection{Forma lineare associata}

A partire dal 1° livello, una volta per riposo lungo, il Warlock può indicare una creatura che è in grado di vedere. Con un'azione può azzerare il bonus del tiro per colpire di quella creatura fino all'inizio del proprio prossimo turno.

\subsection{Omogeneizzazione}

A partire dal 6° livello, un numero di volte pari al suo bonus di competenza per riposo lungo, quando una creatura è a terra con 0 punti ferita, il Warlock può usare la sua azione e toccarla per trasferirla in un semipiano temporaneo. \\ Finchè la creatura si trova nel semipiano, il Warlock ha uno slot incantesimo aggiuntivo. Quando il Warlock inizia un riposo breve o usa questo privilegio su un'altra creatura, la creatura intrappolata viene liberata e diventa stabile.

\subsection{Proiezione}

A partire dal 10° livello, un numero di volte pari al suo bonus di competenza per riposo lungo, il Warlock può usare la sua azione per proiettare la sua immagine su una superficie solida che è in grado di vedere. \\ Per non più di 1 minuto, la posizione del Warlock diventa quella della sua immagine. Fintanto che si trova su una superficie, il Warlock può muoversi liberamentein ogni direzione lungo la stessa, al doppio della sua velocità di movimento, ma non può effettuare azioni che coinvolgano il mondo esterno ad eccezione del parlare. \\ Dopo 1 minuto, il Warlock esce dalla superficie e si trova nello spazio disponibile più vicino a dove si trovava sulla superficie. Questo effetto termina in anticipo se la superficie dove si trova il Warlock viene distrutta o se usa la sua azione per uscirne.

\subsection{Chiusura Proiettiva di Bézout}

A partire dal 14° livello, una volta per riposo breve, il Warlock può usare la sua azione bonus per generare un'aura di Geometria Pura nel raggio di 36m intorno a sè. \\ Fino alla fine del prossimo turno del Warlock, tutti gli attacchi a distanza che richiedono un tiro per colpire effettuati da creature dentro l'aura contro altre creature dentro l'aura vanno automaticamente a segno senza effettuare il tiro per colpire.

\section{Il Signore dell'Assurdo}

\DndDropCapLine{N}{}el multiverso sono poche le creature in grado di esistere contemporaneamente in ogni realtà. È questo il caso dell'entità conosciuta nel nostro mondo come il Signore dell'Assurdo. A voi potrebbe essere familiare sotto altri nomi, come Il Triangolo, Fancy Dorito, Alex, o il suo preferito... Bill. \\ Il suo vero obiettivo non è chiaro, ma una cosa è certa: ciò che pretende dai suoi Warlock è una sana dose di divertimento.

\subsection{Lista degli incantesimi ampliata}

\begin{DndTable}{XX}
  Livello dell'incantesimo  & Incantesimi \\
  1°  & Risata Incontenibile di Tasha, Dardo Tracciante\\
  2°  & Alterare Sè Stesso, Trucco della Corda \\
  3°  & Palla di Fuoco, Parola Guaritrice di Massa \\
  4° & Santuario Privato di Mordenkainen, Tentacoli Neri di Evard \\
  5° & Dominare Persone, Ristorare Superiore \\
\end{DndTable}

\subsection{Squarcio della fortuna}

A partire dal 1° livello, dopo aver lanciato un incantesimo il Warlock deve tirare il d20. Se il risultato del tiro è uguale o inferiore a 2 + il livello dello slot utilizzato, 2+0 nel caso di un trucchetto, il Warlock deve tirare il d100 e subire il relativo effetto dalla tabella della Magia Selvaggia. \\ Se il risultato dalla tabella è il lancio di un incantesimo, subisce anch'esso gli effetti di Squarcio della fortuna.\\ In compenso, il Warlock ha uno slot incantesimo per riposo breve in più, una supplica occulta in più e un utilizzo dell'arcanum mistico di ogni livello per riposo lungo in più.

\begin{DndTable}{XX}
  Livello da Warlock & Dado esplosivo \\
  1°-4°  & d4\\
  5°-10°  & d6 \\
  11°-16°  & d8 \\
  17°-20° & d10 \\
\end{DndTable}

\subsection{Deflagrazione deflagrante}

A partire dal 1° livello quando il Warlock colpisce un avversario con il trucchetto "Deflagrazione Occulta", come azione bonus può aggiungere il suo dado esplosivo ai danni inflitti da ogni raggio. \\ Se il Warlock è sotto un qualunque effetto di massimizzazione del risultato dei tiri per i danni, non può aggiungere il dado esplosivo. \\ Questo privilegio può essere usato un numero di volte pari al bonus di competenza del Warlock per riposo breve.

\subsection{(S)fortunato}

A partire dal 6° livello, il Warlock può appellarsi al suo patrono per alterare il fato in suo favore. \\ Ottiene il talento "Fortunato", ma ogni volta che lo usa deve tirare dalla tabella della Magia Selvaggia.

\subsection{Divertimento Extraplanare}

L'influenza del patrono si espande alle creature intorno al Warlock. \\ Al 10° livello il Warlock ottiene il privilegio "Ispirazione bardica", il dado di ispirazione è il suo dado esplosivo (che se usato in questo modo non esplode). \\ Se il dado di ispirazione risulta in un 1, l'utilizzatore deve tirare dalla tabella della Magia Selvaggia.

\subsection{Pandemonio}

Una volta per riposo lungo, il Warlock può imporre con un'azione a tutte le creature coscienti (compreso sè stesso) nel raggio di 18m di tirare dalla tabella della Magia Selvaggia e tirare il dado esplosivo del Warlock. Ogni creatura riceve il totale dei danni da lei tirati come danni psichici.



\chapter{Talenti}


\subsection{Campo Soprannaturale}
\paragraph{Campo Soprannaturale} Fintanto che il personaggio è cosciente, intorno a lui nel raggio di 27 metri sono attivi gli effetti di una \textit{Regione Soprannaturale} a sua scelta determinata tra quelle a pagina 150 del \textit{Calderone Omnicomprensivo di Tasha}, escludendo il \textit{Fulgore Benedetto}, quando acquisisce questo talento.
\paragraph{Inneschi} Gli effetti casuali del campo si attivano come specificato per quella regione.
\paragraph{Privilegio} In base alla regione scelta, il personaggio ottiene dei privilegi grazie alla sua sintonia con essa:
\subparagraph*{Infestazione} Il personaggio è permanentemente sotto l'effetto dell'incantesimo \textit{Movimenti del Ragno}. Inoltre è immune alla condizione di spaventato quando imposta da dei ragni o insetti.\\
\subparagraph*{Luoghi Infestati} Una volta per riposo lungo, il personaggio può lanciare su di sè l'incantesimo \textit{Invisibilità Superiore} senza usare componenti o slot incantesimo. Inoltre è immune alla condizione di spaventato quando imposta da dei fantasmi.\\
\subparagraph*{Magia Sfibrata} Il personaggio è immune agli incantesimi di Divinazione, sia avversari che alleati che i suoi stessi, e a tutte le forme di localizzazione o individuazione magica.\\
\subparagraph{Reame Remoto} I pensieri del personaggio risultano espressi in una lingua incomprensibile a chiunque tenti di leggerli, a meno che il personaggio non sia consenziente. Inoltre è immune alla condizione di spaventato quando imposta da delle aberrazioni.\\
\subparagraph*{Risonanza Psichica} Il personaggio può comunicare telepaticamente con le creature che è in grado di vedere, anche senza rivelare di essere la sorgente della comunicazione.\\
\subparagraph*{Zona Specchio} Un numero di volte pari al suo bonus di competenza per riposo lungo, il personaggio può usare un'azione bonus per scambiarsi di posto con una creatura umanoide di taglia comparabile che egli sia in grado di vedere.\\
\paragraph{Precisazione} A mio parere, quando una creatura attraverso un effetto di una di queste regioni ottiene la capacità di usare un privilegio o lanciare un incantesimo, è molto più divertente tenere a mente che non lo sa. Ad esempio, se uscisse un risultato di 96-100 come effetto della regione di \textit{Magia Sfibrata}, il personaggio non sarebbe a conoscenza di essere in grado di lanciare una volta l'incantesimo \textit{Desiderio}, e potrebbe ritrovarsi a sprecare questa chance o usarla in modo stupido, ad esempio con un "Come vorrei..." retorico.

\subsection{Negazione divina}

\paragraph{Prerequisiti}Essere un negoziante nell'esercizio della sua professione e nella protezione della sua attività.
\paragraph{Aumento dei punteggi di caratteristica} Fintanto che i requisiti di questo talento sono soddisfatti, i punteggi di caratteristica del personaggio diventano tutti 20.
\paragraph{Competenze bonus}Fintanto che i requisiti di questo talento sono soddisfatti, il personaggio diventa competente in tutte le abilità.
\paragraph{Parola del Potere: No}Fintanto che i requisiti di questo talento sono soddisfatti, il personaggio può lanciare l'incantesimo di 10° livello \textit{Parola del Potere: No} a volontà senza spendere componenti materiali o slot incantesimo.

\chapter{Incantesimi}

\section{Lista degli incantesimi espansa}

\DndDropCapLine{L}{}a più grande fonte di flessibilità in D\&D è la gargantuesca lista di incantesimi fornita dal \textit{Manuale del Giocatore}, per non parlare delle innumerevoli espansioni presenti negli altri manuali ufficiali. \\ Detto questo, sembra che giocatori e dungeon master non facciano altro che aggiungere incantesimi su incantesimi, non ce la si fa più. Figuriamoci se io e i miei giocatori non ne abbiamo aggiunti a nostra volta.

\subsection{Trucchetti}

\subsection{Incantesimi}
\paragraph{1° livello}
\paragraph{2° livello}
\paragraph{3° livello}
\paragraph{4° livello}
\paragraph{5° livello}
\paragraph{6° livello}
\paragraph{7° livello}
\paragraph{8° livello}
\paragraph{9° livello}
\paragraph{10° livello} \textit{Questi incantesimi sono paragonabili all'intervento di una divinità maggiore}
\paragraph{11° livello} \textit{Questi incantesimi sono paragonabili all'intervento di un demiurgo}
\paragraph{12° livello} \textit{Questi incantesimi sono paragonabili all'intervento di Eru Iluvatar}

\section{Descrizione degli incantesimi}

\DndSpellHeader%
  {Limita Magie}
  {Abiurazione di 7° livello}
  {1 reazione, che l'incantatore effettua quando viene lanciato un incantesimo con bersaglio entro gittata}
  {9 metri}
  {S}
  {Concentrazione, finché non viene dissolto}
L'incantesimo crea una bolla di raggio a scelta dell'incantatore compreso tra 0,5 metri e 3 metri entro gittata. Ogni incantesimo che si trova all'interno della bolla viene bloccato nell'istante in cui compare la bolla. Quando la bolla scompare, gli incantesimi al suo interno riprendono il proprio corso.

\DndSpellHeader%
  {Parola del Potere: No}
  {Ammaliamento di 10° livello}
  {1 reazione}
  {Vista}
  {V}
  {Istantanea}
L'Incantatore pronuncia una parola del potere per negare le azioni e le intenzioni di un gruppo di creture in grado di sentirlo.
Le azioni compiute dalle creature influenzate dalla fine dell'ultimo turno dell'incantatore vengono annullate.
Tutte le creature influenzate sono spaventate fino alla fine del prossimo turno dell'Incantatore.

\DndSpellHeader{Più Profondo dell'Abisso, più Alto del Firmamento}
{Invocazione di 11° livello}
{1 minuto}
{Incantatore (raggio d'azione infinito)}
{V, S}
{Istantanea/Infinita}
L'incantatore proietta la sua coscienza lungo tutti i piani dell'universo, in ogni direzione, nel passato e nel futuro, portando in tensione tutta l'energia dell'universo. \\ Al termine del tempo di lancio, questa energia viene rilasciata ed il tessuto dello spazio-tempo stesso inizia a intonare un accordo, risuonando perfettamente con l'anima dell'incantatore. \\ La realtà stessa viene completamente riplasmata secondo questo accordo, ricreando tutto l'universo in una nuova forma.

\DndSpellHeader%
  {Punizione del Caos}
  {Invocazione di 3° livello}
  {1 azione bonus}
  {Incantatore}
  {V}
  {Concentrazione, fino a 1 minuto}
La prossima volta che l'incantatore colpisce una creatura con un attacco con un'arma da mischia entro la durata di questo incantesimo, l'arma in questione viene pervasa da un'aura di \textit{energia caotica} e l'attacco infligge 3d8 danni necrotici extra al bersaglio. 
\subparagraph{Ai livelli superiori}Quando questo incantesimo viene lanciato con uno slot di livello superiore al 3°, i danni inflitti aumentano di 1d8 ogni due livelli.

\DndSpellHeader{Scarica Elementale}
{Invocazione di 3° livello}
{1 azione}
{Incantatore (linea di 30 metri)}
{V, S}
{Istantanea}
Un raggio di energia elementale parte dall'incantatore in una direzione a sua scelta, formando una linea lunga 30 metri a larga 1,5 metri. Ogni creatura situata entro una sfera del raggio di 6 metri centrata su quel punto deve effettuare un tiro salvezza su Destrezza. Se lo fallisce, subisce 8d6 danni da  energia elementale del tipo scelto dall'incantatore (acido, freddo, fulmine, fuoco o tuono), mentre se lo supera, subisce soltanto la metà di quei danni. 
L'esplosione si diffonde oltre gli angoli e danneggia ogni oggetto vulnerabile al tipo di danno scelto nell'area che non sia indossato o trasportato.

\DndSpellHeader{Verità Assoluta}
{Divinazione di 9° livello}
{1 azione}
{Incantatore (Raggio 30 metri)}
{V,S}
{Concentrazione, 1 minuto}
L'incantatore emana un'aura di pura verità che avvolge la sua persona e si espande in un raggio di 30 metri. \\ Durante la durata dell'incantesimo, ogni creatura all'interno dell'area di effetto deve superare un tiro salvezza su Saggezza o essere obbligata a dire la verità in risposta a qualsiasi domanda posta da chiunque nell'area, incluso l'incantatore. \\ Le creature che falliscono il tiro salvezza non possono mentire né ingannare in alcun modo durante la durata dell'incantesimo. Se una creatura che ha fallito il tiro salvezza cerca di mentire o ingannare, subisce automaticamente 6d6 danni psichici. \\ Le creature che superano il tiro salvezza non possono comunque mentire, ma al massimo possono rifiutarsi di rispondere. \\ Inoltre, l'aura di verità assoluta permea il campo e rivela qualsiasi tentativo di distorsione o inganno nella comunicazione, rendendo l'Incantatore consapevole di tutti i tiri salvezza superati. \\ Una volta terminata la durata dell'incantesimo o quando l'incantatore interrompe la concentrazione, l'aura di verità si dissipa e le creature precedentemente soggette all'incantesimo possono tornare a mentire o ingannare normalmente.



\chapter{Oggetti magici}

\section{Oggetti legati ai personaggi dei giocatori}

Questi sono oggetti particolarmente studiati che sono fatti per essere legati ai personaggi creati dai miei giocatori (e a Fillianore e Cianar), diventano più forti man mano che il loro livello aumenta e sono parte della loro storia.

\subsection{Tavoletta di Minera Ioun}
\textit{Tavoletta interattiva (artefatto) meravigliosa leggendaria, richiede sintonia con un Sapiente specializzato in Strategia}

\subsubsection{Funzioni elementari}
A partire dal 1° livello, la tavoletta può essere usata per accedere ad una mappa accurata (ma non omnicomprensiva) della regione di mondo in cui si trova, un cannocchiale, una bussola, un astrolabio e un telescopio. Inoltre permette di scattare delle fotografie e di prendere appunti. Quando viene utilizzata, galleggia a mezz'aria e si muove seguendo l'utilizzatore.\\ Con un'azione, l'utilizzatore può lanciare l'incantesimo \textit{Scurovisione} sulla tavoletta, e guardare attraverso il suo schermo per condividerne gli effetti.\\ La tavoletta ha infine un punteggio di Intelligenza di 14, e ha competenza in tutte le abilità su intelligenza, condividendo il bonus di competenza dell'utilizzatore. Con un'azione bonus, essa può effettuare prove di caratteristica su Intelligenza.\\ Utilizzare la tavoletta richiede almeno una mano libera.

\subsubsection{Amo il fashion, infatti sono un fascista}
A partire dal 1° livello, con un'azione bonus, la tavoletta permette di cambiare l'aspetto estetico dei vestiti dell'utilizzatore. Le modifiche all'abbigliamento sono meramente estetiche non hanno alcun effetto sulle statistiche, tuttavia alla vista e al tatto le modifiche all'abbigliamento sono indistinguibili da un abito reale.

\subsubsection{Stasys}
A partire dal 4° livello, quando una creatura nel raggio di 6 metri dall'utilizzatore è bersaglio di un attacco con arma a distanza (con un proiettile di velocità comparabile ai riflessi umani), l'utilizzatore può usare la sua reazione per aggiungere 1d6 alla sua CA.\\
Se il colpo manca il bersaglio, esso rimane fermo a mezz'aria a 1,5 metri da esso fino alla fine del prossimo turno del bersaglio, dopodichè riparte con la stessa velocità e traiettoria.

\subsubsection{Golem Gufo}
A partire dal 4° livello, con un'azione l'utilizzatore può evocare un Golem Gufo (vedi appendice) che è in grado di controllare attraverso la tavoletta con un'azione bonus fintanto che esso si trovi entro 30m da lui.\\ Il Golem Gufo trasmette alla tavoletta suoni e immagini, agisce nello stesso turno dell'utilizzatore, ma a meno che l'utilizzatore non gli dia degli ordini usando la sua azione bonus, può compiere solo l'azione di Schivata.\\ Se l'utilizzatore è incapacitato, il Golem Gufo agisce autonomamente secondo quello che può ritenere essere il migliore corso d'azione.\\ Se il Golem Gufo viene abbattuto, l'utilizzatore deve passare un'ora di tempo a ripararlo per poterlo evocare di nuovo. Può essere evocato un solo Golem Gufo alla volta.

\subsubsection{Golem aiutanti}
A partire dall'8° livello l'utilizzatore può usare un'azione per lanciare l'incantesimo \textit{Servitore Inosservito} per evocare fino a tre Golem Aiutanti con le seguenti caratteristiche:
\paragraph{Aspetto} I Golem Aiutanti non sono invisibili
\paragraph{Classe armatura} la CA dei Golem Aiutanti è pari a 14
\paragraph{Punteggi di caratteristica} I Golem Aiutanti hanno un punteggio di 8 in ogni statistica
\paragraph{Punti ferita} I Golem Aiutanti hanno 30 punti ferita ciascuno.
Se un Golem viene distrutto, prima di poterlo rievocare l'utilizzatore deve passare un'ora a ripararlo.

\subsubsection{Analisi avanzata}
A partire dall'8° livello, con un'azione, l'utilizzatore può lanciare l'incantesimo \textit{Visione del Vero} sulla tavoletta, e guardare attraverso il suo schermo per condividerne gli effetti. Quando l'incantesimo termina, la tavoletta è disabilitata fino all'alba successiva per l'eccesso di informazioni.\\ A partire dal 16° livello, può farlo senza ripercussioni.

\subsubsection{Stasys+}
A partire dal 12° livello, con un'azione l'utilizzatore può lanciare una versione limitata dell'incantesimo \textit{Fermare il Tempo}.\\
Sceglie un bersaglio entro 9 metri e gli impone un tiro salvezza su Costituzione con CD pari a 8 + il suo bonus di competenza + il suo modificatore di Intelligenza. Se questo fallisce o sceglie di fallire, è congelato nel tempo fino alla fine del turno successivo dell'utilizzatore o finchè esso non vi pone fine con una reazione.\\
Quando l'effetto termina, il bersaglio subisce tutti i danni accumulati durante il blocco temporale.

\subsubsection{Controllare costrutti}
A partire dal 12° livello, con un'azione l'utilizzatore può tentare di far prendere il controllo di un costrutto entro 18 metri all'intelligenza artificiale della tavoletta.\\ Per farlo deve superare una prova di Intelligenza con CD determinata dal DM in base alle dimensioni e alla complessità del costrutto.\\
Se ha successo, l'intelligenza artificiale della tavoletta si trasferisce nel costrutto e può controllarlo come se fosse un suo corpo, tornando automaticamente nella tavoletta se questo viene distrutto.\\
Questo privilegio può essere usato una volta per riposo breve.

\subsubsection{Costrutti migliorati}
A partire dal 16° livello, i punti ferita del Golem Gufo aumentano a 52 (8d8+10), i suoi punteggi di Forza e Destrezza aumentano a 20 e questo può portare in volo fino a due creature di taglia media.\\
Uno dei Golem Aiutanti può essere sostituito da un Golem Guerriero (vedi appendice).

\subsubsection{Macrocostrutto}
A partire dal 19° livello, la tavoletta può prendere il controllo di cinque costrutti contemporaneamente.\\ Se lo fa, può assemblarli in un cazzo di Megazord (vedi appendice).

\subsubsection{Stasys++}
A partire dal 19° livello, con un'azione l'utilizzatore può lanciare l'incantesimo \textit{Fermare il Tempo}.\\
Questo privilegio può essere usato una volta per riposo lungo.

\subsection{Quaderno del Demiurgo}
\textit{Quaderno (artefatto) meraviglioso leggendario, richiede sintonia con un Bardo del Collegio della Creazione o un Warlock del Benefattore Temporale}\\
Un quaderno che pare avere un numero illimitato di pagine, di qualsiasi colore.
Staccandone una pagina e realizzando un origami esso acquisisce certe caratteristiche magiche.

\subsubsection{Origami}
A partire dal 1° livello, l'utilizzatore può staccare una pagina per realizzare un origami magico. La realizzazione di un origami, in base alla sua complessità, può richiedere tra un minuto e una mezzora.\\
Quando l'utilizzatore realizza l'origami di una creatura con Grado Sfida minore o uguale al suo bonus di competenza, può lanciare l'incantesimo \textit{Trova Famiglio} e animarlo in modo da usarlo come famiglio.\\
Il famiglio ha punti ferita massimi pari al livello dell'utilizzatore. Quando scende a 0 punti ferita, l'origami viene distrutto.\\
Quando l'origami viene distrutto o l'utilizzatore decide di farlo tornare un normale origami, può riutilizzare questo privilegio.

\subsubsection{Segreti e segrete}
Moltissime strutture a Eovras hanno un leggio di pietra davanti alla loro entrata.\\
A partire dal 1° livello, se l'utilizzatore appoggia il quaderno aperto su questo leggio, sulla pagina aperta verrà disegnata magicamente una mappa della struttura, che segnerà in tempo reale la posizione del quaderno fintanto che questo si troverà al suo interno.

\subsubsection{Fogli a millemila}
A partire dal 4° livello, come parte della creazione di un origami, l'utilizzatore può alterarne la taglia da minuscolo a enorme. Successivamente, può usare un'azione per toccarlo e alterarla ulteriormente.

\subsubsection{L'inchiostro è più denso del sangue}
A partire dall'8° livello, quando l'utilizzatore crea un origami come famiglio può trasformarlo in una creatura in carne ed ossa.\\
Quando questa scende a 0 punti ferita o viene congedata, torna a essere un origami.

\subsubsection{Cartapentiere}
A partire dal 12° livello, l'utilizzatore può usare un origami per usare una seconda volta il privilegio del Collegio della Creazione \textit{Compimento della Creazione}.

\subsubsection{Animato a mano}
A partire dal 16° livello, l'utilizzatore può usare il quaderno per usare su due oggetti il privilegio del Collegio della Creazione \textit{Creazione Animata} e usare la stessa azione bonus per impartire ordini diversi a ciascuno.

\subsubsection{Iperrealismo}
A partire dal 19° livello, quando l'utilizzatore realizza un origami che rappresenta un oggetto inanimato, può usarlo per lanciare l'incantesimo \textit{Immagine Maggiore}. In questo caso, l'oggetto può essere di dimensione fino a Gargantuesca.\\
Se nessun essere senziente svela l'illusione e il Bardo mantiene concentrazione per una durata pari a un'ora per ogni grado di dimensione, (minuscolo o meno: un'ora, piccolo: due ore, e così via) l'oggetto smette di essere un'illusione e diventa reale.

\subsection{Prigione fluida di Absynthe}
\textit{Arma simbiotica (artefatto) leggendaria, richiede sintonia con un Warlock della Lama del Sortilegio o un Paladino del Giuramento di Vendetta}

\subsection{Borsa degli arnesi stupefacenti di Skardy}
\textit{Borsa di oggetti (artefatto) meravigliosi leggendari, richiede sintonia con un Druido del Circolo della Luna}

\subsection{Sintetizzatore d'Ebano di Tauriel Halamis}
\textit{Anello (artefatto) leggendario, richiede sintonia con un Warlock del Signore dell'Assurdo}

\subsubsection{Sintesi assoluta}
A partire dal 1° livello, l'utilizzatore può far prendere all'anello la forma di un qualsiasi strumento musicale (di dimensioni contenute) in Ebano, paragonabile a uno strumento "vero" di ottima fattura.\\ Quando è in questa forma, l'utilizzatore può usarlo come focus per i suoi incantesimi da Warlock.\\ Finchè è in sintonia con questo oggetto, l'utilizzatore è competente nell'uso di qualsiasi strumento musicale, anche quelli in cui l'anello non è in grado di trasformarsi (ad esempio, un organo a canne).

\subsection{Scudo della Lealtà di Artorias}
\textit{Amuleto (artefatto) leggendario, richiede sintonia con un Guerriero in grado di trasformarsi in un Lupo}

\subsection{Lama della Distruzione (nome placeholder)}
\textit{Spada magica a due mani (artefatto) meravigliosa leggendaria, richiede sintonia con un Barbaro del Cammino del Caos}

\subsubsection{Una prima scintilla di magi}
A partire dal 1° livello, la magia intrisa nella spada le mermette di produrre luce fioca entro 9 metri e di essere utilizzata come focus da incantatore. 

\subsubsection{Rilevatore}
A partire dal 4° livello, la lama della spada può riusuonare con la magia circostante e permettere al Barbaro di lanciare gli incantesimi \textit{Chiaroveggenza} e \textit{Individuazione del magico} senza consumare slot complessivamente per un numero di volte pari al suo bonus di competenza per riposo breve.

\subsubsection{Punizione del caos}
A partire dall'8° livello, un numero di volte pari al proprio bonus di competenza per riposo lungo, il Barbaro può lanciare l'incantesimo \textit{Punizione del caos} senza consumare slot, anche se è in ira. 

\subsubsection{Incanlazione magica}
A partire dal 12° livello, il Barbaro (anche se in ira) può usare la propria azione di attacco per lanciare un trucchetto o un incantesimo che richiedano tiro per colpire attraverso la spada. La caratteristica da incantatore per questi incantesimi è forza. 

\subsubsection{Colpo rotante}
A partire dal 12° livello, il Barbaro può usare un'azione per imporre ai nemici in portata un tiro salvezza su destrezza, altrimenti subiscono 2d8+palle e vengono allontanati dal Barbaro di 1,5 metri. In caso di successo subiscono solo la metà dei danni.

\subsubsection{Un rifugio sempre a portata di mano}
A partire dal 16° livello, la Spada può essere piantata a terra, diventando inestraibile. Chiunque provi ad estrarla, fatta eccezione per il possessore, subisce il danno di un attacco ricevuto dalla spada, che tuttavia non può ridurlo a 0 Punti Ferita. Quando la Spada è piantata in questo modo, il possessore può entrarvi e portare con se delle creature consenzienti. I personaggi all'interno della spada sono in un semipiano altrimenti inaccessibile.

\subsubsection{Un rifugio sempre a portata di mano}
A partire dal 16° livello, un numero di volte pari al proprio modificatore di competenza per riposo lungo, il Barbaro può piantare la Spada a terra e rilasciare un'esplosione di magia del Caos. In questo modo lancia l'incantesimo \textit{Scossa Tellurica} di 5° livello senza consimare slot (anche se in Ira). Il raggio dell'incantesimo è incrementato di 1,5 metri e il terreno diventa difficile a prescindere della composizione, e il Barbaro può decidere di escludere un numero di persone pari al proprio modificatore di forza. 

\subsubsection{Un potente alleato}
A partire dal 19° livello, l'anima magica della spada può manifestarsi e combattere a fianco del suo Padrone. Essa agisce con una propria iniziativa, ma è controllata dal proprio Padrone. 


\section{Oggetti non legati ai personaggi dei giocatori}

\subsection{Errori di battitura}
Esiste tutta una classe di oggetti magici che nascono quando il DM fa qualche errore di battitura. Ne incontrerete molti, e verranno tutti messi qui.

\subsubsection{Alabarba}
\textit{Arma magica comune}\\
Quando un personaggio impugna l'Alabarba, gli cresce istantaneamente una barba. Se questi possiede già una barba, a questa crescerà un'altra barba. Se il personaggio smette di impugnare l'Alabarba, la barba sparisce istantaneamente. Finchè impugna l'Alabarba, il personaggio ha un bonus di +1 alle prove di Carisma.\\
L'Alabarba è assolutamente indistinguibile da un'alabarda normale. Qualsiasi alabarda quando viene fabbricata ha una piccola possibilità (1\%) di diventare un'Alabarba.

\subsection{Frattura}
\textit{Pugnale magico (+0, 1d6 danni magici taglienti) che richiede sintonia con un Ladro, la sua potenza dipende dal livello del Ladro, oggetto meraviglioso, leggendario}

\subsubsection{Squarcio dimensionale}
A partire dal 1° livello, l'utilizzatore può usare un'azione bonus per tagliare lo spazio e aprire un portale di circa 30 cm di raggio per un punto entro 9m che è sia grado di vedere. Può attaccare attraverso questo portale, che si richiude alla fine del suo turno.

\subsubsection{Fenditura silenziosa}
A partire dal 4° livello, l'utilizzatore quando usa \textit{Squarcio dimensionale} può scegliere una creatura che sia in grado di vedere ed effettuare una prova di caratteristica di Destrezza (Furtività) contro la sua percezione passiva. Se la supera, il portale si apre dietro la creatura (che ne è inconsapevole) e il prossimo attacco effettuato attraverso di esso ha vantaggio.

\subsubsection{Strappo duraturo}
A partire dall'8° livello, i portali aperti con \textit{Squarcio dimensionale} permangono per un'ora, o fino a che l'utilizzatore non li chiude con un'azione bonus o usa di nuovo \textit{Squarcio dimensionale} per aprirne un altro.

\subsubsection{Gringott}
A partire dal 10° livello, l'utilizzatore può usare \textit{Squarcio dimensionale} per lanciare l'incantesimo \textit{Semipiano}. Il portale si apre ogni volta su un semipiano di nome "Gringott", legato al pugnale. Solo creature in sintonia con Frattura possono accedere a Gringott lanciando l'incantesimo \textit{Semipiano}.

\subsubsection{Espansione immobiliare}
A partire dal 12° livello, quando l'utilizzatore scopre un oggetto a cui è legato un semipiano, può usare la sua azione per effettuare una prova di caratteristica di Destrezza (Rapidità di mano) contro la CD per il TS contro gli incantesimi dell'incantatore che ha creato il semipiano.\\
Se ha successo, il semipiano diventa parte di Gringott. Una porta per esso appare su uno dei muri di Gringott scelto dall'incantatore. Se Gringott è già composta da diversi semipiani, la porta può apparire in uno qualsiasi di essi. L'utilizzatore può riarrangiare la struttura dei semipiani di Gringott a suo piacimento.\\
L'oggetto da cui è stato sottratto il semipiano mantiene le sue proprietà di porta per esso per un ultimo viaggio, dopodichè perde queste proprietà magiche, ma non viene distrutto e mantiene le altre, se ne aveva.

\subsubsection{Fendispazio}
A partire dal 16° livello, i danni di Frattura aumentano a 2d6 e \textit{Squarcio dimensionale} può essere usato per lanciare l'incantesimo \textit{Portale} verso una destinazione sullo stesso piano di esistenza.

\subsubsection{Fendispazio interplanare}
A partire dal 19° livello, \textit{Squarcio dimensionale} può essere usato per lanciare l'incantesimo \textit{Portale} senza limitazioni.

\subsection{Guanti del CleptoMana}
\textit{Richiede sintonia con un Warlock} 

\subsubsection{Un regalo da molto lontano... ma in che direzione?}
Osservando questi guanti, è normale chiedersi da dove vengano: la loro fattura è inusuale, precisa, come se fossero stati costruiti con attrezzi perduti... o non ancora inventati? \\
Un oggetto bizzarro, pensando al suo potere sembra strano che non ci siano leggende al riguardo, ma di certo l'incantatore in suo possesso ha tutte le carte giuste per scriversene una sua…

\subsubsection{Ladro di Magia}
Un numero di volte pari al suo bonus di competenza per riposo lungo, l'incantatore può scegliere una creatura che egli sia in grado di vedere e usare la sua azione per tentare utilizzare uno slot incantesimo del bersaglio di 5° livello o inferiore. \\
Quando lo fa, come parte della stessa azione lancia un trucchetto a partire dalla posizione del bersaglio. L'incantatore deve essere in grado di vedere il bersaglio del suo trucchetto. Il trucchetto è considerato lanciato dall'incantatore con la sua caratteristica da incantatore. \\ Se il bersaglio non possiede slot di 5° livello o inferiore, i guanti non hanno alcun effetto e sia l'azione che l'utilizzo sono considerati utilizzati.

\subsubsection{Controincantesimo}
Un numero di volte pari a metà del suo bonus di competenza per riposo lungo, l'incantatore può lanciare l'incantesimo \textit{Controincantesimo} come incantesimo di 5° livello senza spendere slot incantesimo. \\ Come parte della stessa reazione, può lanciare un trucchetto a partire dalla posizione del bersaglio (vedi \textit{Ladro di Magia}).

\subsubsection{Pura Cleptoman(i)a}
Una volta per riposo lungo, l'incantatore può scegliere un altro incantatore che sia in grado di vedere e impadronirsi di un suo slot di 6° livello o superiore, convertendolo in uno slot di 5° livello. \\ Se il bersaglio non possiede slot di 6° livello o superiore, i guanti non hanno alcun effetto e sia l'azione che l'utilizzo sono considerati utilizzati.

\subsection{Occhi di Omatara}
\textit{Coppia di anelli leggendari, richiedono sintonia con due incantatori diversi contemporaneamente}

\subsubsection{Sotto lo sguardo della morte}
Anelli magici la cui favola narra siano stati concessi da Omatara, dea della morte, a due incantatori fratelli che riuscirono a ingannarla. \\ Sempre secondo la leggenda, se un incantatore solo dovesse indossarli entrambi, Omatara in persona si preoccuperebbe di provvedere a dargli una morte rapida e fatidica, ma si tratta solo di una leggenda… vero? 

\subsubsection{L'occhio sinistro}
Come azione bonus, l'indossatore può scegliere una creatura senziente che sia in grado di vedere. \\ Per i prossimi 6 secondi, l'indossatore può percepire il mondo dagli occhi di quella creatura e controllarne lo sguardo. La creatura deve effettuare un TS su saggezza con CD 16; se lo fallisce, è accecata durante l'effetto di questo oggetto; L'incantatore può decidere di usare questo effetto senza accecare. \\ Questo privilegio può essere usato un numero di volte pari al bonus di competenza dell'indossatore per riposo breve. La creatura influenzata non può sapere da chi sta venendo influenzata.

\subsubsection{L'occhio destro}
Come azione, l'indossatore può toccare due volte l'anello e teletrasportarsi nell'ombra di una creatura senziente che possa vedere, potendo effettuare un attacco a mani nude. \\ Questo privilegio può essere usato un numero di volte pari al bonus di competenza dell'indossatore per riposo breve.

\subsubsection{Stereoscopia}
Se i due anelli sono indossati da persone diverse, come azione l'indossatore dell'occhio destro può girare l'anello sul dito e percepire il mondo dalla prospettiva dell'indossatore dell'occhio sinistro, non importa dove esso si trovi. A sua volta, l'indossatore dell'occhio sinistro può usare la sua azione per teletrasportarsi nell'ombra dell'indossatore dell'occhio destro. \\ Usare questo privilegio conta come un utilizzo. Se uno dei due anelli sta venendo utilizzato, l'altro indossatore ne è a conoscenza.

\part{Il Mondo di Eovras}

\chapter{La Magia e gli Dei}

\begin{DndReadAloud}
    \it
    Nel mezzo di questo scontro, che scosse le aule d'Ilúvatar e che diffuse un tremito nei silenzi ancora immoti, Ilúvatar si levò una terza volta e il suo volto era terribile a vedersi. Poi egli alzò entrambe le mani e, con un unico accordo, più profondo dell'Abisso, più alto del Firmamento, penetrante come la luce dell'occhio d'Ilúvatar, la Musica cessò. \\ (J. R. R. Tolkien, Il Silmarillion, 1973)
\end{DndReadAloud}

\section{La Magia a Eovras}

\DndDropCapLine{N}{}el mio studio del continente di Eovras mi sono imbattuto in una forma molto peculiare di magia tra i mondi che ho visitato. Sembra che a Eovras la magia funzioni in modo molto simile alla musica: ci sono diversi modi di approcciarvisi, ma sono tutte strade verso la stessa energia. \\ Questo non è un caso: dai miti e dalle leggende sembra evidente che magia e musica siano non solo molto simili, ma praticamente la stessa cosa: l'essenza stessa della realtà.

\paragraph{Gli studi dei Maghi}L'approccio dei maghi è quello più accademico ovviamente, un mago esperto è come un pianista tecnicamente impeccabile, ma potrebbe mancare di spontaneità. Il migliore dei maghi deve eccellere tanto per la sua capacità tecnica quanto per la sua creatività.
\paragraph{Il talento degli Stregoni}Credo che chiunque abbia mai provato ad approcciarsi allo studio di uno strumento musicale prima o poi si sia sentito umiliato dalle capacità di un bambino prodigio, che grazie al suo talento esibisce una maestria del proprio strumento che non si riuscirà mai ad eguagliare. Ecco, questi sono gli stregoni. Sì maghi, siete autorizzati a detestarli.
\paragraph{L'ispirazione divina dei Chierici}I chierici sono un po' come quei compositori, dal medioevo al neoclassicismo, che scrivevano per ispirazione divina, dando voce alla loro fede... D'altronde Vivaldi stesso (per quanto io lo detesti) era conosciuto come "il frate rosso".
\paragraph{L'intuizione lirica dei Druidi}Sono innumerevoli le storie di musicisti, compositori e cantanti "aiutati" da qualche... "supplemento" naturale nel loro lavoro. Dai druidi, non fate finta di non saperlo, sappiamo da dove viene la vostra "magia".
\paragraph{Il playback degli Warlock}Come ai Warlock non piace ricordare, il loro potere in realtà deriva da quello di altre entità, in un certo senso gli warlock nel migliore dei casi fanno una cover, nel peggiore cantano in playback...
\paragraph{I Bardi ve li spiegate da soli}La spiegazione del bardo è banale e lasciata come esercizio per il lettore.

\subsection{Desiderio}

Desiderio è un incantesimo potentissimo, il cui potenziale è compreso da pochissime creature, probabilmente solo un paio di esse di origine mortale. \\ Purtroppo a Eovras la conoscenza di come lanciare Desiderio è stata perduta, quindi gli incantatori che avrebbero accesso a questo leggendario incantesimo non possono sceglierlo o ottenerlo passivamente. \\ Ci sono tuttavia entità in grado di soddisfare desideri, e di sicuro un avventuriero sufficientemente determinato potrà ottenere in qualche modo almeno un utilizzo di questo incantesimo... forse...

\subsection{Resurrezione}

Resuscitare qualcuno è una faccenda relativamente semplice a livello magico, posto di avere il giusto catalizzatore. Purtroppo tale catalizzatore, ovvero il diamante, non esiste in natura. Girano voci tuttavia di un nano, un certo Hausdwarf, che nella sua dimora abbia trovato il modo di fabbricarli, ma sono solo dicerie... forse...

\section{Il pantheon di Eovras}

\DndDropCapLine{I}{}ndipendentemente dalla volontà di un certo Enefeles, a Eovras esistono in effetti diverse divinità, le quali regolano vari aspetti dell'esistenza insieme a moltissimi spiriti minori. Per mia mancanza di creatività, il Pantheon di Eovras ricalca quello ideato da J. R. R. Tolkien ne "Il Silmarillion". I Valar fungono da divinità maggiori, i Maiar da divinità minori e spiriti vari. Quindi sì, se ve lo steste chiedendo, Olorin (anche conosciuto come Mithrandir, Gandalf il Grigio, Gandalf il Bianco, Gandalf il Gandalf, Ganjalf e Gandalf il Rimbambito) esiste a Eovras.

\chapter{Foresta di Mythrenwald}

\begin{DndReadAloud}
  \it
  "Minchia, quante foglie" disse Fillianore addentrandosi nella foresta con i suoi compagni. "Forse venire qui in autunno non è stata una buona ide-" La giovane tiefling non riuscì a finire la frase: appena vide l'albero al centro della radura dove erano appena entrati rimase a bocca aperta e si ammutolì.
\end{DndReadAloud}

\section{Luoghi}

\DndDropCapLine{A}{} tutti coloro che in un modo o nell'altro giungono alle porte della foresta di Mythrenwald basta uno sguardo per rendersi conto di non trovarsi di fronte al boschetto dietro il loro villaggio. Essa non è solo la più grande del continente, ma si dice che sia la culla della vita stessa fin dalla creazione di Eovras ed è pregna di un potere primordiale e sconfinato, ma ancora più grande è il numero di misteri che essa racchiude. Nel cuore della foresta torreggia il Grande Albero di Mythrenbaum, più alto e più antico di ogni struttura mai concepita da menti mortali.

\subsection{Il Grande Albero di Mythrenbaum}

Ben poco si sa della colossale pianta, le cui radici si estendono fino alle profondità della terra e la cui chioma sfonda il bianco soffitto delle nuvole. Oltre alle sue gargantuesche dimensioni, la caratteristica più sorprendente dell'Albero è la sua corteccia: un fittissimo reticolo di rune lo abbraccia completamente, che si muovono, cambiano, splendono di luce stellare. Nessuna mente, mortale o divina, può anche solo sperare di possedere tutta la conoscenza incisa sulla corteccia del Grande Albero, vi sono scritte storie di epoche dimenticate, dati astronomici dalla precisione sorprendente, profezie di eventi che avverranno tra millenni e millenni, incantesimi di potenza incalcolabile e la ricetta per un mix di spezie da usare quando si cucina il pollo arrosto (provare per credere).

\subsection{Il borgo di Mythrenberg}

Nella radura intorno al Grande Albero sorge il borgo di Mythrenberg, sede del Circolo delle Stelle e nucleo abitato più grande della foresta. Mythrenberg sicuramente non è grande come una città volante dei forgiati, o nemmeno come il grande porto di Romboporto, ma c'è tutto quello che serve per condurre una vita pacifica.

\subsubsection{La sede del Circolo} Nulla di particolare, un edificio dalle dimensioni relativamente contenute dotato di un dormitorio, una mensa, uno scriptorium e una sala riunioni per il consiglio degli anziani.

\subsubsection{La biblioteca della corteccia} Visto che le rune del Grande Albero sono scritte in una miriade di lingue, molte di queste dimenticate, parte del compito dei druidi novizi è tradurle e ricopiarle su dei banali libri per la consultazione di chiunque lo richieda, per la gioia di Kur il bibliotecario.

\subsubsection{La falegnameria di Y'Keah} Una normale falegnameria, molto semplice, troppo semplice considerando la domanda di mobili in tutta la foresta.

\subsection{I villaggi di Mythrenwald}

Disseminati per la foresta esistono tutta una serie di piccoli villaggi, sarebbe impossibile elencarli tutti, abitati da varie razze in coesistenza pacifica: elfi di ogni discendenza, gnomi, halfling, umani, aaracockra, ma anche dragonidi, mezzorchi, tiefling, forgiati addirittura... c'è chi dice che da qualche parte si nasconano persino dei divorati elementali...

\section{Abitanti}

\subsection{Halimath Selevarum}

\subsubsection{Il cieco con gli occhi aperti}

Nessuno conosce davvero la storia di come il grande Guardiano di Mythrenwald abbia perso la vista, ma tutti coloro che ne abbiano mai sentito parlare sanno bene che non è saggio assumere che Halimath Selevarum non sia altro che un monaco cieco e indifeso. \\ La sua sconfinata saggezza è frutto dell'esperienza di quasi nove secoli, la sua è una storia di redenzione e di introspezione... ma non è detto che sia così propenso a raccontarvela.

\subsubsection{Un passato oscuro}

Sono in pochi coloro che sanno che in realtà Halimath è giunto a Eovras quando ormai era già un guerriero esperto. \\ Una notte d'estate fu trovato in una radura, nudo e privo di sensi, il suo corpo pieno di bruciature e cicatrici, ma con un sorriso sereno in volto. Quando i druidi di Mythrenwald riuscirono a fargli riprendere i sensi, si trovarono davanti un elfo completamente in pace con sè stesso. Non parlò mai a nessuno del suo passato.

\begin{DndMonster}[float*=b,width=\textwidth + 8pt]{Halimath Selevarum}
    \begin{multicols}{2}
      \DndMonsterType{Elfo dei boschi, buono neutrale}
  
      % If you want to use commas in the key values, enclose the values in braces.
      \DndMonsterBasics[
          armor-class = {20},
          hit-points  = {\DndDice{40d8 + 200}},
          speed       = {19.5 m},
        ]
  
      \DndMonsterAbilityScores[
          str = 12,
          dex = 20,
          con = 20,
          int = 20,
          wis = 20,
          cha = 16,
        ]
  
      \DndMonsterDetails[
          saving-throws = {Str +13, Dex +17, Con +5, Int +17, Wis +17, Cha +5},
          skills = {Animal Handling +17, Arcana +17, Athletics +13, Insight +17, Perception +17, Sleight of Hand +17, Stealth +17},
          %damage-vulnerabilities = {cold},
          %damage-resistances = {bludgeoning, piercing, and slashing from nonmagical attacks},
          %damage-immunities = {poison},
          condition-immunities = {Avvelenato,},
          senses = {Vista cieca 36 m, vista delle auree 72 m. Percezione Passiva 27},
          languages = {Comune, Elfico, Silvano, Gnomesco, Druidico, Primordiale, Draconico},
          challenge = 1,
        ]
      % Traits

      \DndMonsterAction{Tratti di classe}
      Halimath è un Monaco dell'Ombra Redenta di 20° livello e un Druido del Circolo delle Stelle di 20° livello. Possiede tutti i tratti garantitigli da queste due classi.

      \DndMonsterAction{Incantesimi}
      Halimath è un druido di 20° livello. Conosce tutti gli incantesimi da druido. Recupera i suoi slot dopo ogni riposo lungo.
      \begin{DndTable}[header=Slot per livello]{XXXXXXXXX}
        1° & 2° & 3° & 4° & 5° & 6° & 7° & 8° & 9°\\
        8  & 6  & 6  & 4  & 3  & 2  & 2  & 1  & 1 \\
      \end{DndTable}
  
      \DndMonsterSection{Azioni}
      \DndMonsterAction{Multiattacco}
      Halimath compie tre attacchi da mischia.
  
      %Default values are shown commented out
      \DndMonsterAttack[
        name=Pugni,
        distance=melee, % valid options are in the set {both,melee,ranged},
        %type=weapon, %valid options are in the set {weapon,spell}
        mod=+17,
        %reach=1.5,
        %range=20/60,
        %targets=bersaglio singolo,
        dmg=\DndDice{1d10+5},
        dmg-type=forza,
        %plus-dmg=,
        %plus-dmg-type=,
        %or-dmg=,
        %or-dmg-when=,
        %extra=,
      ]
  
      % Legendary Actions
      \DndMonsterSection{Punti ki}
      Halimath possiede 32 punti ki che può usare per i poteri di un Monaco dell'Ombra Redenta di 20° livello.
    \end{multicols}
\end{DndMonster}

\subsection{I Druidi di Mythrenwald}

La foresta di Mythrenwald, per quanto antica e potente, non è invincibile di fronte a ogni minaccia, per questo secoli fa, dopo la prima terrificante Guerra di Eovras, un gruppo di rifugiati trovatisi al cospetto del grande Albero decise di votare la propria vita alla protezione della foresta... nacque il Circolo Druidico delle Stelle.

\subsubsection{I principi del Circolo}

\paragraph{Protezione della Foresta} Il più fondamentale dei compiti di un druido del Circolo delle Stelle è proteggere la foresta di Mythrenwald, sia contro le minacce immediate (ad esempio un dragonide con il raffreddore) sia contro quelle più remote. Non è raro trovare druidi di Mythrenwald in giro per il mondo per cercare di sfatare eventi o conflitti che possano mettere in pericolo la foresta.

\paragraph{Protezione della vita} Oltre a proteggere la foresta in particolare, i druidi di Mythrenwald nel corso dei loro viaggi devono sempre mantenere una condotta che minimizzi i danni nei confronti della vita che li circonda, qualunque sia la sua natura.

\paragraph{Rispetto della morte} Ogni druido sa che vita e morte sono due facce della stessa medaglia, da onorare con la stessa dignità. Un druido di Mythrenwald è tenuto anche ad amministrare i riti funebri di ogni creatura che incontri la morte lungo il suo stesso cammino.

\paragraph{Studio della corteccia e del cielo} La conoscenza incisa sul Grande Albero di Mythrenbaum e il suo riscontro astronomico sono le risorse più importanti per un drudio del Circolo delle Stelle, dalle quali egli trae consiglio e forza. Tra il ritorno da un viaggio e la partenza per un altro, un druido di Mythrenwald deve passare almeno una settimana a studiare le rune della corteccia e il cielo notturno.

\paragraph{Accoglienza e ospitalità} Il Circolo è nato da un gruppo di rifugiati in cerca di una casa, questo non deve essere mai dimenticato. Chiunque voglia unirsi è bene accetto, come ogni viaggiatore in cerca di ospitalità, indipendentemente dalla sua razza o dalle sue origni.

\subsection{La struttura del Circolo}

Sebbene in origine il Circolo delle Stelle fosse un'associazione piuttosto disorganizzata, col tempo è andata costituendosi una struttura gerarchica basata sulla conoscenza delle rune della corteccia.

\paragraph{Il Grande Guardiano} Il capo del Circolo, il più saggio tra i Druidi con la più profonda coscienza dei segreti della foresta. Nella storia di Mythrenwald ci sono stati solo due Grandi Guardiani: Samalas Mythrenwachter, ormai deceduto da almeno un millennio, e Halimath Selevarum, l'attuale Guardiano.

\paragraph{Kur il bibliotecario} Nessuno sa molto su Kur, il misterioso bibliotecario di Mythrenberg. Gira voce che sia misteriosamente apparso offrendosi come volontario il giorno della fondazione della biblioteca (circa tre millenni fa) e che da allora abbia fatto un lavoro impeccabile. Sembra essere l'unico a conoscenza del significato dei nomi di Mythrenwald, Mythrenwachter, Mythrenberg e Mythrenbaum, ma ogni volta che glielo si chiede scoppia a ridere e non riesce a rispondere. \\ \textit{Altre informazioni su Kur possono essere reperite nella Biblioteca Omnicomprensiva di Ker.}

\paragraph{Il consiglio degli anziani} Sebbene i druidi di Mythrenwald abbiano una componente di maggioranza molto giovane, le decisioni che riguardano tutto il Circolo sono prese dal consiglio degli anziani, formato dai sette membri più vecchi del Circolo e presieduto dal Grande Guardiano, che si limita ad agire come organo esecutivo invece che legislativo.

\paragraph{L'Assemblea} L'assemblea di tutti i membri del Circolo funge sia da organo legittimante delle decisioni del Consiglio sia da Tribunale. Il suffragio è universale ed il voto viene espresso segretamente tramite l'incisione di una runa su un pezzo di corteccia di un albero "normale" della foresta e la sepoltura tra le sue radici. Una delle rune del Grande Albero provvede al conteggio e a fornire il responso.

\paragraph{L'ordine dei viaggiatori} Non è mai successo che tutto il Circolo si trovasse nella foresta dopo il giorno della sua fondazione, è sempre esistito un nutrito gruppo di druidi in viaggio all'esterno di Mythrenwald.

\subsubsection{Y'Keah il carpentiere}

Un nano dal marcato accento nordico e dall'inesauribile voglia di lavorare. I suoi mobili non sono di qualità particolarmente pregiata ma senza dubbio risultano molto semplici ed economici, che alla fine sono i fattori più importanti per una regione senza un'economia particolarmente fiorente.

\part{Nemici e Pericoli}

\end{document}