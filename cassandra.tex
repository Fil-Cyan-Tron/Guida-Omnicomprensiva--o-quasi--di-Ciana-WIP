\documentclass[letterpaper,twocolumn,openany,nodeprecatedcode]{dndbook}

% Use babel or polyglossia to automatically redefine macros for terms
% Armor Class, Level, etc...
% Default output is in English; captions are located in lib/dndstrings.sty.
% If no captions exist for a language, English will be used.
%1. To load a language with babel:
%	\usepackage[<lang>]{babel}
%2. To load a language with polyglossia:
%	\usepackage{polyglossia}
%	\setdefaultlanguage{<lang>}
\usepackage[italian]{babel}
%\usepackage[italian]{babel}
% For further options (multilanguage documents, hypenations, language environments...)
% please refer to babel/polyglossia's documentation.

\usepackage[utf8]{inputenc}
\usepackage[singlelinecheck=false]{caption}
\usepackage{lipsum}
\usepackage{listings}
\usepackage{shortvrb}
\usepackage{stfloats}

\captionsetup[table]{labelformat=empty,font={sf,sc,bf,},skip=0pt}

\MakeShortVerb{|}

\lstset{%
  basicstyle=\ttfamily,
  language=[LaTeX]{TeX},
  breaklines=true,
}

\title{The Dark \LaTeX{} \\
\large An Example of the dndbook Class}
\author{The rpgTeX Team}
\date{2020/04/21}

\begin{document}

\chapter*{Cassandra Akatosh}

\begin{DndReadAloud}
  \it
  Alto o basso,\\
  Magro o grasso:\\
  Resta subito di sasso.\\
\end{DndReadAloud}

\DndDropCapLine{D}{}iventare un Ofita non è doloroso, ma ha degli effetti curiosi sull'organismo di umani ed elfi: uno di questi è il rapido e irrimediabile lisciarsi dei capelli e il tingersi di nero di questi e degli iridi, qualunque fosse il loro colore originale. Per questo, è sorprendente imbattersi in un'Ofita riccia, con gli occhi verdi e un singolo ribelle ricciolo rosso sulla fronte.\\
In effetti la matriarca Uriel del clan degli Akatosh è sempre stata turbata (se non preoccupata) dalla resilienza del cespuglio di capelli di fuoco e dello sguardo di smeraldo di quella che sembrava la classica orfanella trovata tra rimasugli e cadaveri di una carovana di mezzelfi assaltata dai predoni, ma col tempo anche gli altri membri della tribù si sono abituati al suo insolito look.\\

\section{Bracciali del Gorgone}
\textit{Oggetto meraviglioso (artefatto), richiede sintonia con un Ofita}\\
La reliquia più sacra degli Ofiti, persa durante la Grande Disinfestazione e presunta distrutta. Si tratta di due bracciali di [metallo simil Ebano] ornati da bassorilievi di un serpente che si avviluppa intorno ad essi; quando il loro potere viene utilizzato, i bassorilievi sembrano muoversi.

\subsection{Vita nell'ombra}
Anche prima delle persecuzioni che hanno portato alla Grande Disinfestazione, gli Ofiti sono sempre stati trattati con diffidenza dalle civiltà degli elfi e degli umani.\\
A partire dal 1° livello dell'utilizzatore, finchè questi bracciali sono indossati permettono di nascondere i tratti da Ofita del possessore, facendolo apparire come un normale elfo dei boschi come se fosse sotto l'effetto dell'incantesimo \textit{Cammuffare Sè Stesso}.

\subsection{Marchio del Serpente}
A partire dal 1° livello dell'utilizzatore, questi bracciali permettono di lanciare a volontà l'incantesimo \textit{Scritta Celeste} facendo comparire nel cielo un serpente di nuvole.

\subsection{Zanne del serpente}
A partire dal 4° livello dell'utilizzatore, questi bracciali forniscono un bonus di +1 ai tiri per colpire e ai danni ai colpi senz'armi del possessore.\\
Questo bonus cresce a +2 al 12° livello e a +3 al 16° livello.

\subsection{Muta}
Per gli Ofiti il Dalamadur simboleggia il tempo non solo nella sua capacità di divorare ogni cosa, ma anche per il processo di rinnovamento tipico dei serpenti, la muta.\\
A partire dal 4° livello dell'utilizzatore, durante un riposo lungo questi può fare una muta, curando gli effetti superficiali non magici di un combattimento (ad esempio graffi, cicatrici e lividi).\\
Effetti più profondi (ad esempio amputazioni o ferite più profonde) possono richiedere più mute per essere curati, a discrezione del DM.

\subsection{Muta migliorata}
A partire dal 8° livello dell'utilizzatore, questi può decidere di alterare alcune sue caratteristiche fisiche (ad esempio genere o altezza) durante una muta.

\subsection{Sguardo pietrificante}
A partire dal 12° livello dell'utilizzatore, questi può lanciare l'incantesimo \textit{Carne in Pietra} come azione bonus contro un bersaglio che sia in grado di guardare negli occhi pronunciando le opportune parole magiche.\\
Questo privilegio può essere usato un numero di volte pari a metà del bonus di competenza dell'utilizzatore per riposo lungo.

\subsection{Miasma illusorio}
A partire dal 16° livello dell'utilizzatore, questi può lanciare l'incantesimo \textit{Drago Illusorio} di tipo Veleno per evocare un'ombra del Dalamadur.\\
Questo privilegio può essere usato un numero di volte pari a metà del bonus di competenza dell'utilizzatore per riposo lungo.

\subsection{Il divoratore del mondo}
A partire dal 19° livello dell'utilizzatore, al più una volta al giorno, questi può usare un'azione per scatenare un'eclissi di sole o di luna (in base all'ora del giorno) ed evocare il vero Dalamadur, richiedendo il suo \textit{Intervento Divino} (senza tirare il dado percentuale).\\
La natura dell'intervento e il tempo necessario per poterlo chiedere nuovamente sono determinati dal DM.\\
Se il Dalamadur non dovesse manifestare la sua forma fisica, lo farebbe attraverso l'ambiente circostante all'utilizzatore (ad esempio le nuvole, l'acqua o la sabbia).

\section{Tradizione monastica: via della Vipera Veggente}
\DndDropCapLine{I}{} monaci Ofiti, tenendo fede al pacifismo della loro tradizione, hanno sempre mantenuto uno stile di combattimento basato sulla difesa, detto \textit{stile della Vipera}. Sulla base di questo, Cassandra ha trovato il modo di integrare le sue premonizioni per rendere il suo contrattacco più efficace.

\subsection{Profezia}
A partire dal 3° livello, il Monaco ottiene competenza nelle abilità Intuizione e Percezione; se la possiede già o la ottiene con altri mezzi, invece di ottenere competenza ottiene maestria.

\subsection{Arti marziali difensive}
A partire dal 3° livello, quando una creatura che può vedere compie un attacco da mischia contro il Monaco, questi può usare la sua reazione e un punto ki per lanciare il proprio Dado da Monaco e aggiungere il suo risultato alla propria Classe Armatura contro quell'attacco. Se l'attacco manca, il Monaco può contrattaccare con un tiro per colpire come parte della stessa reazione.

\subsection{Riflessi amplificati}
A partire dal 6° livello, il Monaco ottiene una reazione aggiuntiva per round (un singolo effetto può scatenare una singola reazione).\\
Ne ottiene un'altra all'11° e un'altra ancora al 17°, per un totale di 4 reazioni per round.

\subsection{Mano del Destino}
A partire dall'11° livello, il Monaco può usare la sua azione bonus per scegliere fino a tre creature nel raggio di 18m e spendere due punti ki per la prima e uno per ciascuna delle altre due per provare a modificare il loro fato.\\
Esse effettuano un Tiro Salvezza su Saggezza e in caso di fallimento, fino all'inizio del prossimo turno del Monaco esse saranno obbligate a tentare di attaccarlo senza utilizzare incantesimi.

\begin{DndReadAloud}
  \it
  Le pergamene del destino si dispiegano davanti ai miei occhi: dalla prima nascita all'ultima morte, dalla prima alba all'ultimo tramonto, dalla prima brezza all'ultimo uragano, io vedo tutto.\\
  Tu non vivrai per vedere il domani.
\end{DndReadAloud}

\subsection{Avatar del Fato}
A partire dal 17° livello, il Monaco può spendere tutti i suoi punti Ki quanto tira per iniziativa per concentrarsi ed entrare completamente in sintonia con lo scorrere del destino per ottenere questi privilegi fino alla fine del combattimento:
\paragraph{Ineluttabile} I privilegi \textit{Arti marziali difensive} e \textit{Mano del Destino} non richiedono punti ki per essere utilizzati;
\paragraph{Instancabile} Ogni volta che tira il suo dado da Monaco per aumentare la sua CA, ottiene un numero di punti ki o punti ferita temporanei pari il risultato del dado; questi punti ferita temporanei durano fino alla sua prossima meditazione;
\paragraph{Crudele} Quando contrattacca con il privilegio "arti marziali difensive", può spendere fino a 5 punti ki per estrarre dal passato o dal futuro del bersaglio degli eventi traumatici e proiettarli nella sua mente per infliggergli un numero di dadi da Monaco pari al numero di punti ki spesi più uno in danni psichici. In questo stato, il Monaco non è in grado di infliggere danni non letali e considera come "fine del combattimento" la morte delle creature ostili o il compimento di un loro destino specificato dal DM.
\paragraph{Sovraumano} Il monaco può spendere 9 punti ki per lanciare l'incantesimo "Fatale" usando gli eventi traumatici dal passato o dal futuro dei bersagli al posto delle loro paure.
\paragraph{Principio di Onniscenza Limitata} Fintanto che è in questo stato, il Monaco possiede un'istintiva e completa conoscenza del destino di ogni creatura che è in grado di vedere. Quando esce da questo stato, perde ogni ricordo di ciò che è successo durante il combattimento, ma nel suo subconscio possono rimanere alcuni stralci di visione o le emozioni che possono aver suscitato in lui.

\end{document}