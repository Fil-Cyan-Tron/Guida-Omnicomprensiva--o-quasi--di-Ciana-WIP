\chapter{La Magia e gli Dei}

\begin{DndReadAloud}
    \it
    Nel mezzo di questo scontro, che scosse le aule d'Ilúvatar e che diffuse un tremito nei silenzi ancora immoti, Ilúvatar si levò una terza volta e il suo volto era terribile a vedersi. Poi egli alzò entrambe le mani e, con un unico accordo, più profondo dell'Abisso, più alto del Firmamento, penetrante come la luce dell'occhio d'Ilúvatar, la Musica cessò. \\ (J. R. R. Tolkien, Il Silmarillion, 1973)
\end{DndReadAloud}

\section{La Magia a Eovras}

\DndDropCapLine{N}{}el mio studio del continente di Eovras mi sono imbattuto in una forma molto peculiare di magia tra i mondi che ho visitato. Sembra che a Eovras la magia funzioni in modo molto simile alla musica: ci sono diversi modi di approcciarvisi, ma sono tutte strade verso la stessa energia. \\ Questo non è un caso: dai miti e dalle leggende sembra evidente che magia e musica siano non solo molto simili, ma praticamente la stessa cosa: l'essenza stessa della realtà.

\paragraph{Gli studi dei Maghi}L'approccio dei maghi è quello più accademico ovviamente, un mago esperto è come un pianista tecnicamente impeccabile, ma potrebbe mancare di spontaneità. Il migliore dei maghi deve eccellere tanto per la sua capacità tecnica quanto per la sua creatività.
\paragraph{Il talento degli Stregoni}Credo che chiunque abbia mai provato ad approcciarsi allo studio di uno strumento musicale prima o poi si sia sentito umiliato dalle capacità di un bambino prodigio, che grazie al suo talento esibisce una maestria del proprio strumento che non si riuscirà mai ad eguagliare. Ecco, questi sono gli stregoni. Sì maghi, siete autorizzati a detestarli.
\paragraph{L'ispirazione divina dei Chierici}I chierici sono un po' come quei compositori, dal medioevo al neoclassicismo, che scrivevano per ispirazione divina, dando voce alla loro fede... D'altronde Vivaldi stesso (per quanto io lo detesti) era conosciuto come "il frate rosso".
\paragraph{L'intuizione lirica dei Druidi}Sono innumerevoli le storie di musicisti, compositori e cantanti "aiutati" da qualche... "supplemento" naturale nel loro lavoro. Dai druidi, non fate finta di non saperlo, sappiamo da dove viene la vostra "magia".
\paragraph{Il playback degli Warlock}Come ai Warlock non piace ricordare, il loro potere in realtà deriva da quello di altre entità, in un certo senso gli warlock nel migliore dei casi fanno una cover, nel peggiore cantano in playback...
\paragraph{I Bardi ve li spiegate da soli}La spiegazione del bardo è banale e lasciata come esercizio per il lettore.

\subsection{Desiderio}

Desiderio è un incantesimo potentissimo, il cui potenziale è compreso da pochissime creature, probabilmente solo un paio di esse di origine mortale. \\ Purtroppo a Eovras la conoscenza di come lanciare Desiderio è stata perduta, quindi gli incantatori che avrebbero accesso a questo leggendario incantesimo non possono sceglierlo o ottenerlo passivamente. \\ Ci sono tuttavia entità in grado di soddisfare desideri, e di sicuro un avventuriero sufficientemente determinato potrà ottenere in qualche modo almeno un utilizzo di questo incantesimo... forse...

\subsection{Resurrezione}

Resuscitare qualcuno è una faccenda relativamente semplice a livello magico, posto di avere il giusto catalizzatore. Purtroppo tale catalizzatore, ovvero il diamante, non esiste in natura. Girano voci tuttavia di un nano, un certo Hausdwarf, che nella sua dimora abbia trovato il modo di fabbricarli, ma sono solo dicerie... forse...

\section{Il pantheon di Eovras}

\DndDropCapLine{I}{}ndipendentemente dalla volontà di un certo Enefeles, a Eovras esistono in effetti diverse divinità, le quali regolano vari aspetti dell'esistenza insieme a moltissimi spiriti minori. Per mia mancanza di creatività, il Pantheon di Eovras ricalca quello ideato da J. R. R. Tolkien ne "Il Silmarillion". I Valar fungono da divinità maggiori, i Maiar da divinità minori e spiriti vari. Quindi sì, se ve lo steste chiedendo, Olorin (anche conosciuto come Mithrandir, Gandalf il Grigio, Gandalf il Bianco, Gandalf il Gandalf, Ganjalf e Gandalf il Rimbambito) esiste a Eovras.