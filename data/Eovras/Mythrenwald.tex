\chapter{Foresta di Mythrenwald}

\begin{DndReadAloud}
  \it
  "Minchia, quante foglie" disse Fillianore addentrandosi nella foresta con i suoi compagni. "Forse venire qui in autunno non è stata una buona ide-" La giovane tiefling non riuscì a finire la frase: appena vide l'albero al centro della radura dove erano appena entrati rimase a bocca aperta e si ammutolì.
\end{DndReadAloud}

\section{Luoghi}

\DndDropCapLine{A}{} tutti coloro che in un modo o nell'altro giungono alle porte della foresta di Mythrenwald basta uno sguardo per rendersi conto di non trovarsi di fronte al boschetto dietro il loro villaggio. Essa non è solo la più grande del continente, ma si dice che sia la culla della vita stessa fin dalla creazione di Eovras ed è pregna di un potere primordiale e sconfinato, ma ancora più grande è il numero di misteri che essa racchiude. Nel cuore della foresta torreggia il Grande Albero di Mythrenbaum, più alto e più antico di ogni struttura mai concepita da menti mortali.

\subsection{Il Grande Albero di Mythrenbaum}

Ben poco si sa della colossale pianta, le cui radici si estendono fino alle profondità della terra e la cui chioma sfonda il bianco soffitto delle nuvole. Oltre alle sue gargantuesche dimensioni, la caratteristica più sorprendente dell'Albero è la sua corteccia: un fittissimo reticolo di rune lo abbraccia completamente, che si muovono, cambiano, splendono di luce stellare. Nessuna mente, mortale o divina, può anche solo sperare di possedere tutta la conoscenza incisa sulla corteccia del Grande Albero, vi sono scritte storie di epoche dimenticate, dati astronomici dalla precisione sorprendente, profezie di eventi che avverranno tra millenni e millenni, incantesimi di potenza incalcolabile e la ricetta per un mix di spezie da usare quando si cucina il pollo arrosto (provare per credere).

\subsubsection{Dentro al tronco dell'Albero}

Il tronco del Grande Albero presenta una piccola cavità (piccola, più o meno) dove è stato rispettosamente scavato l'alloggio del Grande Guardiano. Al centro di questa cavità si trova il cuore dell'Albero, una fibra di pura energia stellare che si protrae lungo tutto l'albero, dalle radici più profonde alle foglioline più alte.

\subsubsection{Sulla chioma}

Nessuno è mai riuscito a scalare l'Albero fino alla cima della chioma o a volarci sopra, ma tra gli strati più bassi si trovano gli alloggi dei membri del consiglio, che si riuniscono sulla cima del tronco. Dagli alloggi scendono dei montacarichi (ovviamente a motore magico) che gli anziani usano per salirvi. Il Grande Guardiano non usa questi montacarichi per salire il tronco ma corre lungo la parete.

\subsection{Il borgo di Mythrenberg}

Nella radura intorno al Grande Albero sorge il borgo di Mythrenberg, sede del Circolo della Luna e nucleo abitato più grande della foresta. Mythrenberg sicuramente non è grande come una città volante dei forgiati, o nemmeno come il grande porto di Romboporto, ma c'è tutto quello che serve per condurre una vita pacifica.

\subsubsection{La sede del Circolo} Nulla di particolare, un edificio dalle dimensioni relativamente contenute dotato di un dormitorio, una mensa, uno scriptorium e una sala riunioni per il consiglio degli anziani.

\subsubsection{La biblioteca della corteccia} Visto che le rune del Grande Albero sono scritte in una miriade di lingue, molte di queste dimenticate, parte del compito dei druidi novizi è tradurle e ricopiarle su dei banali libri per la consultazione di chiunque lo richieda, per la gioia di Kur il bibliotecario.

\subsubsection{La falegnameria di Y'Keah} Una normale falegnameria, molto semplice, troppo semplice considerando la domanda di mobili in tutta la foresta.

\subsection{I villaggi di Mythrenwald}

Disseminati per la foresta esistono tutta una serie di piccoli villaggi, sarebbe impossibile elencarli tutti, abitati da varie razze in coesistenza pacifica: elfi di ogni discendenza, gnomi, halfling, umani, aaracockra, ma anche dragonidi, mezzorchi, tiefling, forgiati addirittura... c'è chi dice che da qualche parte si nasconano persino dei divorati elementali...

\section{Abitanti}

\subsection{Halimath Selevarum}

\subsubsection{Il cieco con gli occhi aperti}

Nessuno conosce davvero la storia di come il grande Guardiano di Mythrenwald abbia perso la vista, ma tutti coloro che ne abbiano mai sentito parlare sanno bene che non è saggio assumere che Halimath Selevarum non sia altro che un monaco cieco e indifeso. \\ La sua sconfinata saggezza è frutto dell'esperienza di quasi nove secoli, la sua è una storia di redenzione e di introspezione... ma non è detto che sia così propenso a raccontarvela.

\subsubsection{Un passato oscuro}

Sono in pochi coloro che sanno che in realtà Halimath è giunto a Eovras quando ormai era già un guerriero esperto. \\ Una notte d'estate fu trovato in una radura, nudo e privo di sensi, il suo corpo pieno di bruciature e cicatrici, ma con un sorriso sereno in volto. Quando i druidi di Mythrenwald riuscirono a fargli riprendere i sensi, si trovarono davanti un elfo completamente in pace con sè stesso. Non parlò mai a nessuno del suo passato.

\begin{DndMonster}[float*=b,width=\textwidth + 8pt]{Halimath Selevarum}
    \begin{multicols}{2}
      \DndMonsterType{Elfo dei boschi, buono neutrale}
  
      % If you want to use commas in the key values, enclose the values in braces.
      \DndMonsterBasics[
          armor-class = {20},
          hit-points  = {\DndDice{40d8 + 200}},
          speed       = {19.5 m},
        ]
  
      \DndMonsterAbilityScores[
          str = 12,
          dex = 20,
          con = 20,
          int = 20,
          wis = 20,
          cha = 16,
        ]
  
      \DndMonsterDetails[
          saving-throws = {For +13, Des +17, Int +17, Sag +17},
          skills = {Addestrare Animali +17, Arcano +17, Atletica +13, Intuizione +17, Percezione +17, Rapidità di Mano +17, Furtività +17},
          %damage-vulnerabilities = {cold},
          %damage-resistances = {bludgeoning, piercing, and slashing from nonmagical attacks},
          %damage-immunities = {poison},
          condition-immunities = {Avvelenato, ammalato},
          senses = {Vista cieca 36 m, vista delle auree 72 m. Percezione Passiva 27},
          languages = {Comune, Elfico, Silvano, Gnomesco, Druidico, Primordiale, Draconico},
          challenge = 16,
        ]
      % Traits

      \DndMonsterAction{Tratti di classe}
      Halimath è un Monaco dell'Ombra Redenta di 20° livello e un Monaco della via del Sè astrale di 20° livello. Possiede tutti i tratti garantitigli da queste due classi.

      \DndMonsterAction{Incantesimi}Halimath è un incantatore di 20° livello, possiede gli slot incantesimo appropriati e conosce tutti gli incantesimi da Druido. Recupera i suoi slot dopo ogni riposo lungo. La CD del tiro salvezza contro questi incantesimi è 25 e il bonus al tiro per colpire con questi incantesimi è di 17.
  
      \DndMonsterSection{Azioni}
      \DndMonsterAction{Multiattacco}
      Halimath effettua tre attacchi da mischia.
  
      %Default values are shown commented out
      \DndMonsterAttack[
        name=Pugni,
        distance=melee, % valid options are in the set {both,melee,ranged},
        %type=weapon, %valid options are in the set {weapon,spell}
        mod=+17,
        %reach=1.5,
        %range=20/60,
        %targets=bersaglio singolo,
        dmg=\DndDice{1d10+5},
        dmg-type=forza,
        %plus-dmg=,
        %plus-dmg-type=,
        %or-dmg=,
        %or-dmg-when=,
        %extra=,
      ]
  
      % Legendary Actions
      \DndMonsterSection{Punti ki}
      Halimath possiede 52 punti ki che può usare per i poteri di un Monaco dell'Ombra Redenta di 20° livello e di un Monaco della via del Sé Astrale di 20° livello.
    \end{multicols}
\end{DndMonster}

\subsection{I Druidi di Mythrenwald}

La foresta di Mythrenwald, per quanto antica e potente, non è invincibile di fronte a ogni minaccia, per questo secoli fa, dopo la prima terrificante Guerra di Eovras, un gruppo di rifugiati trovatisi al cospetto del grande Albero decise di votare la propria vita alla protezione della foresta... nacque il Circolo Druidico della Luna.

\subsubsection{I principi del Circolo}

\paragraph{Protezione della Foresta} Il più fondamentale dei compiti di un druido del Circolo della Luna è proteggere la foresta di Mythrenwald, sia contro le minacce immediate (ad esempio un dragonide con il raffreddore) sia contro quelle più remote. Non è raro trovare druidi di Mythrenwald in giro per il mondo per cercare di sfatare eventi o conflitti che possano mettere in pericolo la foresta.

\paragraph{Protezione della vita} Oltre a proteggere la foresta in particolare, i druidi di Mythrenwald nel corso dei loro viaggi devono sempre mantenere una condotta che minimizzi i danni nei confronti della vita che li circonda, qualunque sia la sua natura.

\paragraph{Rispetto della morte} Ogni druido sa che vita e morte sono due facce della stessa medaglia, da onorare con la stessa dignità. Un druido di Mythrenwald è tenuto anche ad amministrare i riti funebri di ogni creatura che incontri la morte lungo il suo stesso cammino.

\paragraph{Studio della corteccia e del cielo} La conoscenza incisa sul Grande Albero di Mythrenbaum e il suo riscontro astronomico sono le risorse più importanti per un drudio del Circolo della Luna, dalle quali egli trae consiglio e forza. Tra il ritorno da un viaggio e la partenza per un altro, un druido di Mythrenwald deve passare almeno una settimana a studiare le rune della corteccia e il cielo notturno.

\paragraph{Accoglienza e ospitalità} Il Circolo è nato da un gruppo di rifugiati in cerca di una casa, questo non deve essere mai dimenticato. Chiunque voglia unirsi è bene accetto, come ogni viaggiatore in cerca di ospitalità, indipendentemente dalla sua razza o dalle sue origni.

\begin{DndMonster}[float*=b,width=\textwidth + 8pt]{Fillianore Halamis}
  \begin{multicols}{2}
    \DndMonsterType{Tiefling, buono caotico}

    % If you want to use commas in the key values, enclose the values in braces.
    \DndMonsterBasics[
        armor-class = {16 (Armatura Magica)},
        hit-points  = {\DndDice{20d8 + 100}},
        speed       = {9 m},
      ]

    \DndMonsterAbilityScores[
        str = 16,
        dex = 16,
        con = 20,
        int = 16,
        wis = 16,
        cha = 20,
      ]

    \DndMonsterDetails[
        saving-throws = {Sag +9, Car +11},
        skills = {Acrobazia +9, Inganno +11, Intimidire +11, Intrattenere +11},
        %damage-vulnerabilities = {cold},
        damage-resistances = {Fuoco},
        %damage-immunities = {poison},
        %condition-immunities = {Avvelenato,},
        senses = {Vista Pura 36m. Percezione Passiva 13},
        languages = {Comune, Infernale, Silvano, Celestiale, Elfico},
        challenge = 10,
      ]
    % Traits

    \DndMonsterAction{Incantesimi innati}
    Fillianore può lanciare alcuni incantesimi grazie alle suppliche occulte che possiede e alla sua razza. La CD del tiro salvezza contro questi incantesimi è 19.
    \begin{DndMonsterSpells}
      \DndMonsterSpellLevel{Taumaturgia}
      \DndMonsterSpellLevel[1]{Armatura Magica, Camuffare Sè Stesso}
      \DndMonsterSpellLevel[2]{Invisibilità, Levitazione, Intimorire Infernale, Oscurità}
      \DndMonsterSpellLevel[3]{Inviare (su 6 persone prescelte)}
    \end{DndMonsterSpells}


    \DndMonsterAction{Incantesimi}
    Fillianore recupera i suoi slot incantesimo ogni riposo breve. La CD del tiro salvezza contro questi incantesimi è 19.
    \begin{DndMonsterSpells}
      \DndMonsterSpellLevel{Deflagrazione Occulta, Guida, Interdizione alle Lame, Mano Magica, Rintocco dei Morti, Riparare, Salvare i Morenti}
      \DndMonsterSpellLevel[5][5 slot]{Armatura di Agathys, Charme su Persone, Servitore Inosservato, Frantumare, Lama d'Ombra, Controincantesimo, Rimuovi Maledizione, Volare, Esilio, Porta Dimensionale, Terreno Illusorio, Contattare Altri Piani, Scrutare, Sogno}
    \end{DndMonsterSpells}

    \DndMonsterAction{Arcanum Mistico}
    Fillianore può lanciare alcuni incantesimi come Arcanum Mistico. Fillianore recupera i suoi utilizzi dell'Arcanum Mistico ogni riposo lungo. La CD del tiro salvezza contro questi incantesimi è 19.
    \begin{DndMonsterSpells}
      \DndMonsterSpellLevel[6][2]{Gabbia dell'Anima}
      \DndMonsterSpellLevel[7][2]{Spostamento Planare}
      \DndMonsterSpellLevel[8][2]{Parola del Potere Stordire}
      \DndMonsterSpellLevel[9][2]{Proiezione Astrale}
    \end{DndMonsterSpells}

    \DndMonsterAction{Maestro dell'Occulto}
    Una volta per riposo lungo, Fillianore può meditare per un minuto per riottenere i suoi slot spesi.

    \DndMonsterAction{Poteri del Signore dell'Assurdo}
    Il patrono di Fillianore è il Signore dell'Assurdo, quindi possiede tutti i poteri che ne derivano.

    \DndMonsterAction{Deflagrazione Aleatoria}
    6 volte per riposo breve, quando Fillianore colpisce un bersaglio con il trucchetto Deflagrazione Occulta, come azione bonus può imporre al bersaglio di tirare dalla tabella della Magia Selvaggia un numero di volte pari al numero di raggi da cui è stato colpito.

  \end{multicols}
\end{DndMonster}

\subsection{La struttura del Circolo}

Sebbene in origine il Circolo della Luna fosse un'associazione piuttosto disorganizzata, col tempo è andata costituendosi una struttura gerarchica basata sulla conoscenza delle rune della corteccia.

\paragraph{Il Grande Guardiano} Il capo del Circolo, il più saggio tra i Druidi con la più profonda coscienza dei segreti della foresta. Nella storia di Mythrenwald ci sono stati solo due Grandi Guardiani: Samalas Mythrenwachter, ormai deceduto da almeno un millennio, e Halimath Selevarum, l'attuale Guardiano.

\paragraph{Kur il bibliotecario} Nessuno sa molto su Kur, il misterioso bibliotecario di Mythrenberg. Gira voce che sia misteriosamente apparso offrendosi come volontario il giorno della fondazione della biblioteca (circa tre millenni fa) e che da allora abbia fatto un lavoro impeccabile. Sembra essere l'unico a conoscenza del significato dei nomi di Mythrenwald, Mythrenwachter, Mythrenberg e Mythrenbaum, ma ogni volta che glielo si chiede scoppia a ridere e non riesce a rispondere. \\ \textit{Altre informazioni su Kur possono essere reperite nella Biblioteca Omnicomprensiva di Ker.}

\paragraph{Il consiglio degli anziani} Sebbene i druidi di Mythrenwald abbiano una componente di maggioranza molto giovane, le decisioni che riguardano tutto il Circolo sono prese dal consiglio degli anziani, formato dai sette membri più vecchi del Circolo e presieduto dal Grande Guardiano, che si limita ad agire come organo esecutivo invece che legislativo.

\paragraph{L'Assemblea} L'assemblea di tutti i membri del Circolo funge sia da organo legittimante delle decisioni del Consiglio sia da Tribunale. Il suffragio è universale ed il voto viene espresso segretamente tramite l'incisione di una runa su un pezzo di corteccia di un albero "normale" della foresta e la sepoltura tra le sue radici. Una delle rune del Grande Albero provvede al conteggio e a fornire il responso.

\paragraph{L'ordine dei viaggiatori} Non è mai successo che tutto il Circolo si trovasse nella foresta dopo il giorno della sua fondazione, è sempre esistito un nutrito gruppo di druidi in viaggio all'esterno di Mythrenwald.

\subsubsection{Y'Keah il carpentiere}

Un nano dal marcato accento nordico e dall'inesauribile voglia di lavorare. I suoi mobili non sono di qualità particolarmente pregiata ma senza dubbio risultano molto semplici ed economici, che alla fine sono i fattori più importanti per una regione senza un'economia particolarmente fiorente.

\subsection{Fillianore Halamis}

\subsubsection{Figlia dell'Abisso}

Capita che i Tiefling come la giovane Fillianore Halamis nascano da famiglie umane, ma questo non è il caso di questo Warlock che abita a Mythrenberg. Nata da una madre mortale e un maiar successivamente corrotto da Melkor in un balrog, porta su di sè i segni della maledizione di suo padre. \\ Per quanto la riguarda in realtà, non le importa molto. Fin da piccola ha vissuto a Mythrenwald con sua madre, giocando con gli spiritelli della Foresta, facendo scherzi agli anziani del concilio, leggendo libri presi in prestito dalla Biblioteca della Corteccia (che puntualmente Kur doveva andare a recuperare di persona) e sforzandosi di costruire mobili che non sembrassero opere d'arte moderna nella falegnameria di Y'Keah. \\ La sua adolescenza procedeva bene, fino a quando dopo una storia d'amore finita male qualcosa cambiò in lei... non è detto per il meglio.

\subsubsection{L'incontro col Triangolo}

Frustrata, spaventata e delusa, Fillianore si allontanò da Mythrenberg e iniziò ad esplorare parti più profonde e misteriose della foresta... fino ad incontrare una strana radura triangolare, al cui centro si trovava un piccolo stagno. La radura e lo stagno erano popolate da piante e animali... strani è la parola giusta. Sembravano in tutto e per tutto identici a quelli tipici della foresta, ma avevano un indefinibile \textit{qualcosa} di sbagliato. \\ Non si sa esattamente come Bill si sia presentato a Fillianore, nè quali fossero i termini del loro patto... Ma una volta uscita dalla radura, era chiaro che ci fosse qualcosa di diverso in lei. Tornò a Mythrenberg, dove vive tutt'ora con sua madre, ma da quel giorno ogni settimana, a mezzanotte tra il giovedì e il venerdì, apre un portale e sparisce per il resto della nottata... Chissà dove va...