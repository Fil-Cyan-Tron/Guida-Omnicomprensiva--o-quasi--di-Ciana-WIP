\nchapter{Economia delle azioni}
\section{Il Round}
\DndDropCapLine{L}{}'unità di tempo fondamentale nel gioco è il round. Un round è un periodo di tempo di circa 6 secondi, durante il quale si svolgono "contemporaneamente" tutte le azioni dei personaggi.\\

\subsection{Iniziativa}
Contemporaneamente è un termine che va preso con le pinze. In realtà, ogni personaggio agisce in un ordine ben preciso, che viene determinato all'inizio di ogni combattimento tirando per l'iniziativa appunto, ovvero sommando il proprio modificatore di Destrezza al tiro di un d20. \\ Il risultato dell'iniziativa determina l'ordine di azione, chi ha ottenuto il risultato più alto agisce per primo, e così via. In caso di parità, si tira un nuovo d20 per determinare chi agisce prima.\\ Un personaggio può decidere di rinunciare alla propria azione per agire dopo un altro personaggio, ma non può decidere di agire prima di un altro personaggio che ha ottenuto un risultato più alto.\\ 

\subsection{Il turno}
\DndDropCapLine{D}{}urante il suo turno un personaggio può compiere un movimento, un'azione e un'azione bonus.\\

\subsubsection{Movimento}
Il movimento massimo che può compiere un personaggio durante il suo turno è pari alla sua velocità. Queste azioni possono essere compiute spendendo parte del proprio movimento.\\ Ci si può muovere in qualunque momento durante il proprio turno, l'importante è che il movimento totale sia pari alla propria velocità.\\
\paragraph{Movimento normale}
Il movimento normale è un movimento che non richiede alcun costo aggiuntivo oltre alla normale velocità di movimento.\\
\paragraph{Salto in lungo}
Un personaggio può saltare in lungo fino a un numero di metri pari al suo punteggio di Forza moltiplicato per 15 cm se parte da fermo, o al doppio di questo se parte con una rincorsa di almeno 3 m. La somma della distanza di rincorsa e della distanza saltata è comunque da considerarsi spesa ai fini del movimento.\\
\paragraph{Salto in alto}
Con una rincorsa di almeno 3 m, si può saltare fino ad un'altezza calcolata con la formula $(3 + F_{mod})*30cm$ dove $F_{mod}$ è il modificatore di Forza del personaggio. Saltando da fermi questa altezza è dimezzata. La distanza di rincorsa va comunque sottratta al movimento rimanente, ma quella di salto no.\\
\paragraph{Nuoto}
Un personaggio può nuotare fino a una distanza pari alla metà della sua velocità.\\
\paragraph{Scalata}
Un personaggio può scalare fino a una distanza verticale pari a metà della sua velocità.\\
\paragraph{Sdraiarsi prono}
Sdraiarsi a terra non ha nessun costo in termini di movimento.\\
\paragraph{Rialzarsi}
Rialzarsi da prono costa metà del movimento totale.\\
\paragraph{Strisciare}
Strisciare costa il doppio della distanza rispetto al movimento normale.\\

\subsubsection{Azioni}
Un personaggio può compiere un'azione durante il proprio turno.\\
\paragraph{Attaccare} Vedi la sezione specifica.\\
\paragraph{Lanciare un incantesimo} Un personaggio può lanciare un incantesimo dal tempo di lancio pari ad un'azione o usarne una per iniziare a lanciarne uno con un tempo di lancio più lungo. È importante specificare che se si lancia più di un incantesimo per turno, almeno uno di questi deve essere un trucchetto.\\
\paragraph{Privilegi di classe}
La maggior parte dei privilegi di classe richiedono un'azione per essere usati.\\
\paragraph{Scatto}
Un personaggio può usare un'azione per compiere un movimento aggiuntivo.\\
\paragraph{Usare un oggetto}
Un personaggio può usare un'azione per interagire con un oggetto, ad esempio bere una pozione.\\
\paragraph{Prepararsi}
Un personaggio può usare un'azione per prepararsi a compiere un'azione come reazione, specificando un evento che la farebbe scattare.\\
\paragraph{Schivare}
Un personaggio può usare un'azione per schivare, imponendo svantaggio ai tiri per colpire contro di lui da parte di nemici che è in grado di vedere e guadagnando vantaggio ai tiri salvezza su Destrezza fino all'inizio del suo prossimo turno.\\
\paragraph{Disimpegno}
Un personaggio può usare un'azione per disimpegnarsi, il che gli permette di muoversi senza provocare attacchi di opportunità da parte di nemici che sono in grado di vederlo e che non sono inabili.\\
\paragraph{Liberarsi}
Un personaggio può usare un'azione per liberarsi da una presa o da un vincolo con una prova contrapposta di Atletica o Acrobazia.\\
\paragraph{Aiutare}
Un personaggio può usare un'azione per aiutare un alleato in un'azione che richiede una prova di abilità, dandogli vantaggio.\\
\paragraph{Nascondersi}
Un personaggio può usare un'azione per nascondersi, provando una prova di Destrezza (Furtività) contro la Saggezza (Percezione) di chiunque sia in grado di vederlo.\\
\paragraph{Cercare}
Un personaggio può usare un'azione per cercare, provando una prova di Indagare o di Percezione contro la Furtività di chiunque sia in grado di nascondersi.\\

\subsubsection{Lottare}
Oltre ai classici attacchi, un personaggio può decidere di compiere un'azione di attacco speciale, ovvero la lotta. Un personaggio può lottare con un bersaglio se e solo se ha almeno una mano libera e questi è più grande di lui al massimo di una taglia.\\
\paragraph{Spingere a terra}
Un personaggio può usare un'azione per spingere a terra un nemico che si trova a meno di 1,5 m da lui. Il personaggio effettua una prova di Atletica contro una prova di Atletica o Acrobazia del nemico. Se il personaggio vince, il nemico è spinto a terra prono.\\
\paragraph{Allontanare}
Un personaggio può usare un'azione per allontanare un nemico che si trova a meno di 1,5 m da lui. Il personaggio effettua una prova di Atletica contro una prova di Atletica o Acrobazia del nemico. Se il personaggio vince, il nemico è spinto di 1,5 m più 1,5 m per ogni 5 punti di differenza tra le due prove.\\
\paragraph{Afferrare}
Un personaggio può usare un'azione per afferrare un nemico che si trova a meno di 1,5 m da lui. Il personaggio effettua una prova di Atletica contro una prova di Atletica o Acrobazia del nemico. Se il personaggio vince, il nemico è afferrato.\\

\subsubsection{Azioni bonus}
Un personaggio può compiere al massimo un'azione bonus durante il proprio turno, ma deve possedere un privilegio che gli dia effettivamente delle opzioni con cui usarla, ad esempio lanciare un incantesimo con tempo di lancio pari ad un'azione bonus.\\

\subsubsection{Reazioni}
Un personaggio può compiere una reazione durante il round ma fuori dal proprio turno, a meno che non abbia un privilegio che gli consenta di spenderla durante il suo turno.\\
\paragraph{Attacco di opportunità}
Un personaggio può compiere un attacco di opportunità contro un nemico che sta uscendo dalla sua area minacciata (un cerchio di raggio 1,5 m).\\
\paragraph{Azione preparata}
Un personaggio può compiere un'azione preparata come reazione, se l'evento che l'avrebbe fatta scattare si è verificato.\\