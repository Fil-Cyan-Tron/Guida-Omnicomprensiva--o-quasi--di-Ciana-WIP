\nchapter{Qualche regola}

\section{Condotta generale}

\DndDropCapLine{U}na campagna di D\&D è per sua natura un'esperienza da condividere con diverse persone, e sebbene esistano delle regole ufficiali o semiufficiali, è sempre una buona idea tracciare alcune linee guida per garantire lo svolgimento ottimale (in termini di gradevolezza del gioco) della campagna che ci si appresta a giocare.

\subsection{D\&D è un gioco di squadra}

Qualunque cosa succeda, è sempre il caso di ricordarsi che D\&D nasce come un gioco e quello deve restare, non ha senso prendersela o arrabbiarsi per cose che succedono nel gioco dandogli più importanza di quella che hanno. \\ La precondizione fondamentale di ogni gioiosa sessione o campagna di D\&D è la disposizione di ognuno alla comprensione e la tolleranza. \\ Tranne se si parla con un mago gnomo di nome Enefeles.

\subsection{Le regole vanno interpretate}

Spesso durante una sessione di D\&D capita che sorgano dubbi sull'interpretazione delle regole, che in quinta edizione sono spesso scritte in modo abbastanza vago. \\ In questi casi, i giocatori sono incoraggiati a proporre la loro interpretazione e accordarsi democraticamente, ma il giudizio ultimo è sempre quello del DM. \\ Se la discussione di una regola inizia a diventare troppo lunga, il DM può decidere immediatamente e chiudere la discussione.

\subsection{Segreti tra i personaggi, non tra i giocatori}

Un certo livello di metagaming è ammesso e necessario almeno secondo me, rende tutti i giocatori più partecipi e coinvolti nel roleplay. A meno che non ci sia un buon motivo di trama per tenere delle informazioni segrete tra i giocatori, è sempre una buona idea condividerle, purchè poi i personaggi non sfruttino o agiscano in virtù di informazioni che non hanno motivo di avere.

\subsection{I segreti del DM}

Lo giuro su tutti i miei dadi, se andate a cercare gli statblock delle creature che state affrontando vi mangio il cranio. Quella è la forma di metagaming più sbagliata di tutte, anche se conoscete già quelle creature se i vostri personaggi non le conoscono non avete alcun diritto di agire come se le conoscessero.\\ I segreti del DM sono assolutamente inviolabili, lo schermo è una barriera infrangibile. \\ I tiri del DM sono coperti per un buon motivo.

\subsection{Azioni e conseguenze}

Ogni giocatore è responsabile delle scelte compiute dal proprio personaggio. \\ \textit{L'unico e solo modo di conoscere gli effetti di un'azione è compierla, non chiederlo al DM}

\section{I personaggi e il party}

\DndDropCapLine{Q}{}ueste regole sono più per questioni di bilanciamento personali piuttosto che imperativi kantiani.

\subsection{Creazione del personaggio}

\paragraph{Statistiche al 1° livello}

Tirare 20d6 e distribuire i punteggi ottenuti come meglio si crede. Nessun punteggio (prima dei bonus razziali) può essere più alto di 15 o più basso di 5 per garantire una gestione tutto sommato pacifica del roleplay.

\paragraph{Talenti al 1° livello}

No, giocate l'umano variante se volete talenti al livello 1.

\paragraph{Personalità e background}

Cercate di evitare "lupi solitari", va bene un personaggio con delle difficoltà a livello relazionale ma provate a impersonare personaggi che vi diano comunque un sufficiente spazio di roleplay e relazione con gli altri membri del party, vi assicuro che rende il gioco più divertente per tutti.

\paragraph{Materiale sorgente}

Il \textit{Manuale del Giocatore}, la \textit{Guida Omnicomprensiva di Xanathar} e il \textit{Calderone Omnicomprensivo di Tasha} sono ammessi insieme ovviamente a questo libro(!), classi homebrew non sono ammesse in generale (salvo rarissime eccezioni), razze e sottoclassi homebrew sono generalmente bene accolte previa approvazione del DM. 

\paragraph{Equipaggiamento al 1° livello}

Oltre al normale equipaggiamento garantito dalle classi e dai background, ogni personaggio al 1° livello inizia con un oggetto magico che diventerà progressivamente più potente (ogni incremento del punteggio di caratteristica). Questo oggetto deve essere creato dai giocatori insieme al DM e deve essere parte della lore del personaggio. Se un personaggio subentra ad un livello più alto, sarà come se avesse avuto il suo oggetto fin dal 1° livello. Attenzione, perdere la sintonia con l'oggetto comporta la perdita di tutti i power up! Riacquistare la sintonia significa partire dal 1° livello dell'oggetto! Al 12° livello l'oggetto diventa senziente.

\paragraph{Roleplay sopra al powerplay}

Avere un personaggio forte è sicuramente divertente e incoraggiabile, ma piuttosto che ottimizzare l'utilità in combattimento a tutti i costi, ricordate sempre di non tralasciare il roleplay per ottenere quel d8 di danno in più.

\paragraph{Punti ferita ai livelli superiori}

Per quanto riguarda l'aumento dei punti ferita chiedo di usare sempre il valore atteso del dado vita e tratto come retroattivi gli aumenti del punteggio di Costituzione per il calcolo dei punti ferita.

\subsection{Il party}

\paragraph{Varietà}

Nella composizione del party chiedo di sforzarsi e mettersi d'accordo con gli altri giocatori per garantire una certa varietà in termini di classi, ruoli, razze e pesonalità, in quanto rende più interessante sia il combattimento che il roleplay.

\paragraph{Vincoli di composizione}

In particolare, chiederei di limitare i membri del party appartenenti ad una classe ad un singolo membro contemporaneamente, a meno che non si tratti di due sottoclassi che cambiano radicalmente il playstyle (ad esempio un Warlock melee e un Warlock cecchino). \\ Anche un singolo livello di multiclasse occupa lo "slot" della classe (sto guardando voi, che puntavate ad un singolo livello in \textit{Lama del Sortilegio}).

\paragraph{PVP}

Il PVP è ammesso purchè sia estremamente ben giustificato a livello di roleplay, è ammesso che i personaggi eventualmente si stacchino dal party e diventino NPC, non sono ammessi personaggi antagonisti per partito preso.

\subsection{Addio di un personaggio}

\paragraph{Morte}

Vi avverto fin dall'inizio: non sarò clemente con danni non letali e resurrezioni in generale, se un personaggio muore e non viene resuscitato prima che subentri il nuovo personaggio del giocatore, sarà da considerarsi morto. Se dovesse venire resuscitato, esisterà nella campagna come NPC controllato in condivisione dal DM e dal suo giocatore iniziale.

\paragraph{Allontanamento}

Può capitare che un giocatore perda il \textit{feeling} per un personaggio, che l'arco narrativo del personaggio si chiuda naturalmente o che per una serie di scelte gli obiettivi e i valori del personaggio si allontanino irreversibilmente da quelli del party. In questi casi, in accordo col DM, si può negoziare un'uscita di scena di questo personaggio e l'ingresso del nuovo personaggio del giocatore.\\ Si consiglia vivamente di non abusare di questa opzione.

\paragraph{Perdita del giocatore}

Infine, alle volte accade che non sia il personaggio ma il giocatore stesso ad abbandonare il party, per una serie di motivi. In base alla natura dell'assenza, il DM potrà concordare con il giocatore (se questi sarà disponibile a negoziare) il destino del suo personaggio, ma in qualsiasi caso è consigliabile optare per un allontanamento temporaneo dal party.

\subsection{Riunione col party dopo una morte o un'assenza}

\paragraph{Morte}

Ogni giocatore deve aver già pronto un personaggio da far subentrare in caso di morte del proprio per la sessione successiva. Alternativamente, in base al momento all'interno della trama, è straordinariamente possibile impersonare un personaggio "filler" per una sessione di divertimento e gaia trivialità.

\paragraph{Assenza prolungata}

Questo tipo di assenza andrebbe discussa con più attenzione e soprattutto anche con gli altri membri del party, soprattutto se dovessero essere subentrati nuovi giocatori. \\ In seguito alla riunione di un giocatore dopo un periodo di Erasmus, il suo personaggio perde temporaneamente la conoscenza della lingua Comune.

\section{Regole opzionali}

\DndDropCapLine{Q}{}ueste regole sono quelle che deviano o si aggiungono alle regole (o banalmente riporto da un manuale reinterpretate) di D\&D e che preferisco utilizzare nelle mie campagne.

\subsection{20 Naturale}

Trovo che sia estremamente divertente l'idea che il 20 naturale permetta di fare letteralmente qualsiasi cosa. Lo so, RAW sarebbe "il migliore esito \textit{possibile}", ma io preferisco dire che sì, fanculo, 20 nat è un successo automatico e pure con cose fuori di testa. Vuoi persuadere una porta ad aprirsi? Sicuro, occhio che se fallisci però potrebbe offendersi...

\subsection{Realismo stronzo}

La quinta edizione di D\&D non è bilanciata intorno all'idea di una battaglia per riposo lungo, i personaggi hanno semplicemente troppe opzioni e si finisce per trascurare l'importante fattore di gestione delle risorse che giustifica la minore versatilità dei combattenti rispetto agli incantatori, ma allo stesso tempo giocare qualcosa come 5 o 6 incontri al giorno può risultare estenuante. Per questo io preferisco implementare la regola del \textit{Realismo Crudo}: un riposo breve richiede 8 ore e un riposo lungo richiede una settimana. \\ A meno che non sia specificato il contrario dal DM, le regole dei dadi vita sono da rispettare.

\subsection{Dadi esplosivi}

Un dado esplosivo è un dado che viene ritirato ogni volta che il suo risultato è pari al massimo del dado e alla fine del processo restituisce la somma delle iterazioni. \\ Ad esempio, un d4 esplosivo potrebbe restituire come serie di risultati 4, 4 e 2, quindi il risultato del tiro sarebbe 4+4+2=10.

\subsection{Endomulticlasse}

Come ammetto la multiclasse, ammetto anche che un personaggio multiclassi nella sua stessa classe, magari per ricevere i privilegi di diverse sottoclassi! Al fine di conteggiare i suoi privilegi limitati tuttavia questi vanno conteggiati separatamente e secondo il livello della classe che glieli garantisce. \\ Ad esempio un Monaco/Monaco di livello 10/10 potrà ridurre al massimo di 50 punti ferita il suo danno da caduta usando una reazione, mentre un Warlock/Warlock di livello 11/11 avrà due utilizzi dell'Arcanum Mistico ma entrambi limitati al 6° livello. \\ Ogni ambiguità va sottoposta al giudizio del DM, ovviamente.

\subsection{Multiclasse alternativa}

Non amo molto il sistema di multiclasse secondo il quale finchè prendi livelli in altri incantatori puri puoi avere la normale progressione degli slot. Per quanto mi riguarda, multiclassando in diverse classi da incantatore si ottengono gli stessi slot che si otterrebbero con le altre classi e si sommano ai propri. Ad esempio un Mago/Chierico 6/5 non avrà slot di 6° livello, ma ne avrà 3+2 slot di 3° livello. Allo stesso modo, un combattente che multiclassa in un incantatore non riceve metà degli slot che riceverebbe normalmente ma li riceve tutti.

\subsection{Danni non letali}

Non mi importa molto della logica, ogni personaggio può decidere di infliggere danni non letali con qualsiasi sorgente... a meno di spiacevoli incidenti... \textit{risata malvagia}. \\ Ogni volta che un personaggio decide di colpire con danni non letali deve tirare un d100. Se fa meno dei danni che ha causato con quell'attacco, i suoi danni sono comunque considerati letali. \\ Se un personaggio non è in grado di infliggere danni letali o non letali per via di un'altra regola (ad esempio il Monaco dell'Ombra Redenta), quella bypassa il tiro.

\subsection{Punti Follia}

Non sempre nelle campagne di D\&D si bada alla salute mentale dei personaggi, ma non è questo il caso nel mio mondo! Dopo aver vissuto esperienze disturbanti, il DM può premiare i vostri personaggi con dei fantastici punti Follia, in base all'esperienza!

\begin{DndTable}[header=Effetti della Follia]{XX}
    Punteggio & Effetto \\
    1-4 & Svantaggio alle prove basate su caratteristiche mentali e ai tiri salvezza contro gli incantesimi\\
    5-9 & Paura di creature, luoghi e oggetti casuali o determinati dal DM\\
    10-19 & Svantaggio a tutti i tiri salvezza, tiri per colpire e prove di caratteristica\\
    20-34 & Ogni ora va superato un tiro salvezza su Saggezza con CD 12 o si è colpiti da pazzia a breve termine per 1d10 minuti\\
    35-49 & Paralisi mentale, se non viene curata la Follia dopo 1d4 giorni arriva a 50 punti automaticamente\\
    50+ & Pazzia permanente, il personaggio cade sotto il controllo del DM e il giocatore deve usare un nuovo personaggio\\
\end{DndTable}

\paragraph{Curare la Follia}
Un riposo breve o mangiare una bacca di \textit{Bacche Benefiche} (al massimo una per riposo breve) cura 1 punto Follia, un riposo lungo ne cura 3, \textit{Ristorare Inferiore} ne cura 5 e \textit{Ristorare Superiore} porta al limite superiore del livello precedente, \textit{Parola del Potere Guarire} cura tutti i punti Follia a meno che non siano 50 o più e l'effetto di \textit{Desiderio} può riportare alla normalità qualunque livello di Follia.

\subsection{Soglia di Danno}
Il sistema della CA in D\&D di quinta edizione è sicuramente una buona simulazione del combattimento, ma ci sono momenti in cui secondo me fallisce nelle sue premesse logiche. \\ Normalmente durante il combattimento l'interpretazione della CA è l'abilità di un personaggio di schivare o parare un attacco. Tuttavia nel caso di creature dalle proporzioni gargantuesche come il Terrasque ad esempio, non ha senso parlare di attacchi che le "manchino", perchè logicamente parlando mi sembrerebbe abbastanza difficile non colpire una creatura così grande. \\ Per questo ho deciso di stabilire una soglia di danno per alcune creature: la CA di queste armature sarà sicuramente molto bassa, ma se il danno (dopo il calcolo delle resistenze) di un singolo attacco (tre attacchi vanno conteggiati separatamente) è inferiore alla soglia di danno, non viene inflitto alcun danno. \\ Oltre a quella descritta sopra, che chiamerei soglia di danno "istanziale" si possono specificare altri tipi, ad esempio soglia di danno per azione o per turno. \\ Ho inoltre deciso di ribilanciare alcune abilità o opzioni dei personaggi intorno a questa regola, come avrete modo di vedere. 

\subsection{Divisioni e arrotondamenti}
A meno di specifiche contrarie, tutti gli arrotondamenti da fare a seguito di una divisione sono per difetto.

