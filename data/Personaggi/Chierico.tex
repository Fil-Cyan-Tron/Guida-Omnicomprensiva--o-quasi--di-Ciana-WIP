\chapter{Chierico}

\section{Privilegi di Classe Opzionali}

\section{Dominio Divino: Dominio di Integrità}

\begin{DndReadAloud}
  \it
  "E questo è interessante. Perchè è interessante? Perchè l'ho scritto, e io scrivo solo cose interessanti." \\ (W.A. De Graaf)
\end{DndReadAloud}

\subsection{Incantesimi di dominio}

\begin{DndTable}{XX}
  Livello da Chierico  & Incantesimi \\
  1° & Armatura di Agathys, Comando\\
  3° & Frantumare, Immagine Speculare\\
  5° & Aura di Vitalità, Controincantesimo\\
  7° & Inaridire, Occhio Arcano\\
  9° & Mano di Bigby, Reincarnazione\\
\end{DndTable}

\subsection{Competenze bonus}

Al 1° livello, il Chierico ottiene competenza nelle seguenti abilità: Arcano e Intuizione. \\ Quando effettua una prova di caratteristica su queste abilità raddoppia il suo modificatore di competenza.

\subsection{Forza bruta}

A partire dal 1° livello, il Chierico può scegliere di usare il suo modificatore di Intelligenza invece che di Saggezza per ogni prova di caratteristica, tiro per colpire o tiro salvezza che lo richieda.\\ Allo stesso modo, la sua caratteristica da incantatore diventa Intelligenza.

\subsection{Incanalare Divinità: Divisone con resto}

A partire dal 2° livello il Chierico può spendere un suo utilizzo di \textit{incanalare divinità} per effettuare la divisione con resto su dei danni subiti da un suo alleato. \\ I danni vengono divisi per tre e inflitti equamente al Chierico, all'alleato e alla creatura che ha inflitto i danni. \\ I tre poi devono effettuare una prova contrapposta di Intelligenza pura, e chi ottiene il risultato più basso riceve il resto dei danni.

\subsection{Incanalare Divinità: Permutazione}

A partire dal 6° livello il Chierico può spendere un suo utilizzo di \textit{incanalare divinità} per scegliere un numero di creature pari o inferiore al suo livello da Chierico (almeno 2) di cui conosca le posizioni e permutarle a suo piacimento. \\ Se una creatura viene permutata in uno spazio che non può occupare, viene automaticamente teletrasportata nello spazio sicuro che possa occupare più vicino.

\subsection{Buon ordinamento}

A partire dall'8° livello, quando il Chierico tira per iniziativa, può scegliere l'ultima creatura alleata nell'ordine di iniziativa, sè stesso incluso, e decidere quando agirà all'inizio di ogni round.

\subsection{Fattorizzazione unica}

A partire dal 17° livello, una volta per riposo lungo, il Chierico può usare la sua azione per scegliere una creatura che sia in grado di vedere e conoscerne i punti ferita attuali. \\ Quando lo fa, deve scegliere uno dei fattori primi del numero di punti ferita attuali della creatura: questa viene divisa in quel numero di copie più piccole di sè stessa e i suoi punti ferita vengono divisi equamente tra di esse. \\ Le copie possiedono le stesse statistiche e abilità della creatura di partenza, ma non possono effettuare azioni o azioni bonus, soltanto reazioni e movimenti.\\ All'inizio del turno successivo del Chierico, le copie vengono rifuse nella creatura iniziale, la quale riappare nello spazio in cui si trovava prima di essere divisa o nel più vicino spazio libero e subisce i danni subiti da ognuna delle singole copie. \\ Se il numero di punti ferita della creatura è un numero primo, il Chierico ottiene un utilizzo aggiuntivo di \textit{incanalare divinità}. \\ Il DM sceglie la taglia delle copie (minore o uguale alla taglia della creatura originale) e la loro posizione.
