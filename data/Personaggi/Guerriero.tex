\chapter{Guerriero}

\section{Privilegi di Classe Opzionali}

\section{Re dei Cavalieri}

\subsection{Acciaio e Vento}

A partire dal 3° livello, il Guerriero ottiene un'arma leggendaria, \textit{Acciaio e Vento}. \\ Si tratta di una spada lunga (versatile), il cui danno aumenta nel tempo in base al livello da Guerriero del suo possessore. Si presenta come una spada invisibile avvolta da delle forti correnti d'aria che condensando il vapore acqueo nell'atmosfera permettono di distinguerla. \\ Come azione gratuita, il Guerriero può congedare \textit{Acciaio e Vento} in un semipiano accessibile solo ad essa e richiamarla a sè, ovunque si trovi. \\ Una volta per riposo lungo, il Guerriero può decidere di usare la sua azione bonus per rivelare la vera forma di \textit{Acciaio e Vento} per 1 minuto, aggiungendo dei danni radiosi ai colpi effettuati con essa.
\paragraph{CD del tiro salvezza} Alcuni effetti di \textit{Acciaio e Vento} impongono ai bersagli un tiro salvezza. La CD di questo tiro è pari a 8 + il modificatore di Forza del Guerriero + il bonus di competenza del Guerriero.

\begin{DndTable}[header=Acciaio e vento]{XXXX}
    Livello da Guerriero & Danni & Danni radiosi extra & Velocità bonus \\
    3°-6° & 1d8 (1d10) & 1d6 & --\\
    7°-9° & 1d10 (1d12) & 2d6 & 1,5 m\\
    10°-14° & 2d6 (2d8) & 3d6 & 3 m\\
    15°-17° & 2d8 (2d10) & 4d6 & 4,5 m\\
    18°-20° & 2d10 (2d12) & 5d6 & 6 m\\
\end{DndTable}

\subsection{Parata impetuosa}
A partire dal 7° livello, se il Guerriero impugna \textit{Acciaio e Vento} può lanciare l'incantesimo \textit{Scudo} con essa un numero di volte pari al suo bonus di competenza per riposo breve. Se para l'attacco di un nemico usando questo privilegio, la velocità di movimento del Guerriero aumenta fino alla fine del suo prossimo turno come indicato sulla tabella.

\subsection{Aura regale}
A partire dal 10° livello, mentre \textit{Acciaio e Vento} è rivelata, tutte le creature alleate del Guerriero nel raggio di 9 m da essa hanno vantaggio nei tiri salvezza, un bonus pari al modificatore di Forza del Guerriero alle prove di caratteristica e un bonus pari al bonus di competenza del Guerriero ai danni effettuati con attacchi corpo a corpo. \\ Inoltre, i danni inflitti da \textit{Acciaio e Vento} e tutte le creature alleate nell'area sono considerati magici ai fini di ignorare le resistenze.

\subsection{Rocca della Tavola Rotonda}
A partire dal 15° livello, una volta per riposo lungo, il Guerriero può piantare a terra \textit{Acciaio e Vento} e lanciare l'incantesimo \textit{Reggia Meravigliosa di Mordenkainen} senza spendere componenti materiali. \\ L'incantesimo permane finchè la spada è piantata a terra, e il Guerriero è l'unico che la può estrarre.

\subsection{Miracolo del Re}
A partire dal 18° livello, alla fine del suo turno il Guerriero può usare la sua azione bonus per cominciare a caricare \textit{Acciaio e Vento} di energia (purchè sia già rivelata). Dall'inizio del suo turno successivo, può usare la sua azione per liberarla di fronte a sè. \\ Tutte le creature che si trovano in un'area larga 3 metri e lunga 27 metri di fronte al Guerriero devono effettuare un tiro salvezza su Costituzione. Se lo falliscono subiscono 12d8 danni radiosi e sono accecati fino all'inizio del turno successivo del Guerriero, altrimenti subiscono solo la metà dei danni. Le creature di allineamento malvagio hanno svantaggio al tiro, mentre le creature di allineamento buono hanno vantaggio. \\ Dopo aver usato questo privilegio, \textit{Acciaio e Vento} rimane rivelata fino al successivo riposo breve o lungo. Dopo aver usato questo privilegio, non può più essere usato fino al successivo riposo lungo.