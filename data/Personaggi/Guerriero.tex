\chapter{Guerriero}

\section{Privilegi di Classe Opzionali}

\begin{DndMonster}[float*=b,width=\textwidth + 8pt]{Brunilde, la Valchiria}
  \begin{multicols}{2}
    \DndMonsterType{Celestiale, legale neutrale}

    % If you want to use commas in the key values, enclose the values in braces.
    \DndMonsterBasics[
        armor-class = {20 (armatura completa e scudo)},
        hit-points  = {\DndDice{20d10 + 100}},
        speed       = {9 m, volo 12 m},
      ]

    \DndMonsterAbilityScores[
        str = 20,
        dex = 18,
        con = 20,
        int = 12,
        wis = 16,
        cha = 14,
      ]

    \DndMonsterDetails[
        saving-throws = {For +11, Des +10, Cos +11, Sag +9},
        %damage-vulnerabilities = {cold},
        damage-resistances = {Radiosi, danni da attacchi con armi non magiche},
        %damage-immunities = {poison},
        condition-immunities = {Avvelenata, ogni malattia},
        senses = {Vista Pura 36m, vista cieca 18 m. Percezione Passiva 19},
        languages = {Tutti i linguaggi},
        challenge = 10,
      ]
    % Traits

    \DndMonsterAction{Incantesimi innati}
    Brunilde può lanciare alcuni incantesimi innatamente. La CD del tiro salvezza contro questi incantesimi è 19.
    \begin{DndMonsterSpells}
      \DndMonsterSpellLevel{Colpo di Grazia, Guida, Resistenza, Salvare i Morenti}
      \DndMonsterSpellLevel[1]{Benedizione, Cerimonia, Duello Obbligato, Eroismo, Scudo della Fede}
      \DndMonsterSpellLevel[2]{Aiuto, Trova Cavalcatura}
      \DndMonsterSpellLevel[3]{Manto del Crociato}
      \DndMonsterSpellLevel[5]{Comunione, Santificare, Scrutare}
      \DndMonsterSpellLevel[7]{Forma Eterea}
    \end{DndMonsterSpells}

    \DndMonsterSection{Azioni}

    \DndMonsterAction{Multiattacco}
    Brunilde compie due \textit{attacchi con lancia} o un \textit{attacco con scudo}.

    \DndMonsterAction{Attacco con lancia}
    Attacco da mischia, +11 al tiro per colpire, \DndDice{1d10+5} danni perforanti (magici) sul colpo e \DndDice{1d8} danni radiosi aggiuntivi

    \DndMonsterAction{Attacco con scudo}
    Attacco da mischia, +11 al tiro per colpire, \DndDice{2d6+5} danni contundenti (magici) sul colpo e \DndDice{1d8} danni radiosi aggiuntivi. Sul colpo, tiro salvezza su Costituzione (CD 19): in caso di fallimento, il bersaglio è stordito fino alla fine del suo turno successivo.

    \DndMonsterSection{Azioni bonus}
    
    \DndMonsterAction{Carica}
    Brunilde si muove verso una creatura entro la sua velocità di movimento totale ed effettua un \textit{attacco con scudo} al suo arrivo.

    \DndMonsterSection{Reazioni}

    \DndMonsterAction{Soccorso divino}
    Brunilde può teletrasportarsi con una reazione davanti al guerriero che l'ha evocata e lanciare l'incantesimo \textit{Scudo} come parte della stessa reazione.

    \DndMonsterAction{Colpo di Grazia}
    Quando i punti ferita del guerriero che l'ha evocata scendono a 0, Brunilde si teletrasporta sulla sua posizione come reazione e all'inizio del suo prossimo turno lancia il trucchetto \textit{Colpo di Grazia} per poi sparire come azione bonus.
  \end{multicols}
\end{DndMonster}

\section{Campione Variante}

\subsection{Senza Paura (sostituisce Critico Migliorato)}
A partire dal 3° livello, il Guerriero è immune alla condizione di "Spaventato" e supera in automatico le prove di caratteristica contro le prove di Intimidazione (Carisma) delle altre creature.

\subsection{Angelo della Battaglia (sostituisce Critico Superiore)}
A partire dal 15° livello, una volta per riposo breve, il Guerriero può usare la sua azione per invocare l'aiuto di una Valchiria.\\
Per il prossimo minuto, può lanciare a volontà l'incantesimo \textit{Scudo} usando la sua reazione, le sue prove di caratteristica sono considerate influenzate dall'azione di \textit{Aiuto} da parte di una creatura invisibile e la sua velocità di movimento aumenta di 4,5 metri, inoltre non subisce gli effetti del terreno difficile.

\subsection{Ali della Vittoria (sostituisce Sopravvissuto)}
A partire dal 18° livello, quando tira per iniziativa, il Guerriero può usare la sua reazione per evocare Brunilde. \\ Essa viene considerata una creatura alleata e viene congedata fino al combattimento successivo quando il combattimento termina o i suoi punti ferita finiscono. Se i punti ferita del Guerriero scendono a 0, Brunilde userà il suo privilegio \textit{Colpo di Grazia} come reazione.

