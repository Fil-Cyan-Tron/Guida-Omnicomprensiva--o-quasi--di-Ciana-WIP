\chapter{Incantesimi}

\section{Incantesimi modificati}

\subsection{Esilio}
All'inizio di ogni suo turno, la creatura esiliata può ripetere il tiro salvezza su Carisma per liberarsi. \\ Inoltre, durante la durata dell'incantesimo, la creatura si trova ancora sul piano dell'incantatore, ma è considerata trattenuta. \\ Al termine "naturale" dell'incantesimo, la creatura torna nel suo piano d'origine. Se si trova già sul suo piano d'origine, viene esiliata in un semipiano innocuo (può fare ritorno soltanto con incantesimi, oggetti o privilegi che permettono di passare tra due piani). 
\paragraph{Ai livelli superiori} Quando l'incantatore lancia questo incantesimo usando uno slot incantesimo di 5° livello o superiore, la durata diminuisce di 12 secondi (2 turni) per ogni slot di livello superiore al 4°. Al 9° livello la durata è istantanea. \\ Indipendentemente dal livello dello slot usato, può essere esiliata solo una creatura alla volta.
\paragraph{Nota bene} Molti DM sono piuttosto \textit{laissez faire} nei confronti della componente materiale di \textit{Esilio}, ma non sarà questo il caso sotto la mia direzione. \\ \textit{Esilio} anche con questi nerf è comunque un incantesimo potentissimo, in grado di eliminare (almeno funzionalmente) creature ben al di fuori della portata di un incantatore di 7° livello, quindi è giusto richiedere una componente materiale molto specifica per alcune creature particolarmente potenti.

\subsection{Scudo}
Il bonus della CA garantito da \textit{Scudo} non è più un +5 ma pari al bonus di compentenza dell'incantatore che lo lancia.

\section{Lista degli incantesimi espansa}

\DndDropCapLine{L}{}a più grande fonte di flessibilità in D\&D è la gargantuesca lista di incantesimi fornita dal \textit{Manuale del Giocatore}, per non parlare delle innumerevoli espansioni presenti negli altri manuali ufficiali. \\ Detto questo, sembra che giocatori e dungeon master non facciano altro che aggiungere incantesimi su incantesimi, non ce la si fa più. Figuriamoci se io e i miei giocatori non ne abbiamo aggiunti a nostra volta.

\subsection{Trucchetti} 
Colpo di Grazia

\subsection{Incantesimi}
\paragraph{1° livello}
\paragraph{2° livello}
\paragraph{3° livello}
\paragraph{4° livello}
\paragraph{5° livello}
\paragraph{6° livello}
\paragraph{7° livello}
\paragraph{8° livello}
\paragraph{9° livello}
\paragraph{10° livello} \textit{Questi incantesimi sono paragonabili all'intervento di una divinità maggiore}
\paragraph{11° livello} \textit{Questi incantesimi sono paragonabili all'intervento di un demiurgo}
\paragraph{12° livello} \textit{Questi incantesimi sono paragonabili all'intervento di Eru Iluvatar}

\section{Descrizione degli incantesimi}

\DndSpellHeader%
  {Colpo di Grazia}
  {Trucchetto di Necromanzia}
  {1 azione}
  {Contatto}
  {V,S,M (Un'arma che non viene consumata)}
  {Istantanea}
L'incantatore esegue un rito su una creatura a 0 punti ferita, stabile o meno, e la uccide, mandando la sua anima nel Walhalla (o in un luogo ultraterreno equivalente).\\ Fino al prossimo riposo lungo, l'arma utilizzata viene considerata magica al fine di ignorare le resistenze ai danni da armi non magiche e in sintonia con l'incantatore, i suoi colpi infliggono 1d4 danni radiosi aggiuntivi contro i non morti. \\ Una creatura uccisa in questo modo non può essere resa un non morto e non può essere riportata in vita se non dall'intervento di una divinità o dall'incantesimo \textit{Desiderio}.

\DndSpellHeader%
  {Estensione di Dominio}
  {Matemagia di 7° livello}
  {1 ora (rituale)}
  {Contatto}
  {V,S,M (Diamanti del valore complessivo di 2500 mo che non vengono consumati)}
  {Istantanea}
  L'incantatore tocca una creatura consenziente. A partire dal termine dell'incantesimo, la creatura può sintonizzarsi con qualsiasi oggetto magico (indipendentemente dai requisiti) e qualsiasi numero di essi contemporaneamente.

\DndSpellHeader%
  {Limita Magie}
  {Abiurazione di 7° livello}
  {1 reazione, che l'incantatore effettua quando viene lanciato un incantesimo con bersaglio entro gittata}
  {9 metri}
  {S}
  {Concentrazione, finché non viene dissolto}
L'incantesimo crea una bolla di raggio a scelta dell'incantatore compreso tra 0,5 metri e 3 metri entro gittata. Ogni incantesimo che si trova all'interno della bolla viene bloccato nell'istante in cui compare la bolla. Quando la bolla scompare, gli incantesimi al suo interno riprendono il proprio corso.

\DndSpellHeader%
  {Parola del Potere: No}
  {Ammaliamento di 10° livello}
  {1 reazione}
  {Vista}
  {V}
  {Istantanea}
L'Incantatore pronuncia una parola del potere per negare le azioni e le intenzioni di un gruppo di creture in grado di sentirlo.
Le azioni compiute dalle creature influenzate dalla fine dell'ultimo turno dell'incantatore vengono annullate.
Tutte le creature influenzate sono spaventate fino alla fine del prossimo turno dell'Incantatore.

\DndSpellHeader{Più Profondo dell'Abisso, più Alto del Firmamento}
{Invocazione di 11° livello}
{1 minuto}
{Incantatore (raggio d'azione infinito)}
{V, S, M (quello che si può considerare il vero nucleo dell'universo, che viene consumato)}
{Istantanea}
L'incantatore proietta la sua coscienza lungo tutti i piani dell'universo, in ogni direzione, nel passato e nel futuro, portando in tensione tutta l'energia dell'universo. \\ Al termine del tempo di lancio, questa energia viene rilasciata ed il tessuto dello spazio-tempo stesso inizia a intonare un accordo, risuonando perfettamente con l'anima dell'incantatore. \\ La realtà stessa viene completamente riplasmata secondo questo accordo, ricreando tutto l'universo in una nuova forma.

\DndSpellHeader%
  {Punizione del Caos}
  {Invocazione di 3° livello}
  {1 azione bonus}
  {Incantatore}
  {V}
  {Concentrazione, fino a 1 minuto}
La prossima volta che l'incantatore colpisce una creatura con un attacco con un'arma da mischia entro la durata di questo incantesimo, l'arma in questione viene pervasa da un'aura di \textit{energia caotica} e l'attacco infligge 3d8 danni necrotici extra al bersaglio. 
\subparagraph{Ai livelli superiori}Quando questo incantesimo viene lanciato con uno slot di livello superiore al 3°, i danni inflitti aumentano di 1d8 ogni due livelli.

\DndSpellHeader{Scarica Elementale}
{Invocazione di 3° livello}
{1 azione}
{Incantatore (linea di 30 metri)}
{V, S}
{Istantanea}
Un raggio di energia elementale parte dall'incantatore in una direzione a sua scelta, formando una linea lunga 30 metri a larga 1,5 metri. Ogni creatura situata entro una sfera del raggio di 6 metri centrata su quel punto deve effettuare un tiro salvezza su Destrezza. Se lo fallisce, subisce 8d6 danni da  energia elementale del tipo scelto dall'incantatore (acido, freddo, fulmine, fuoco o tuono), mentre se lo supera, subisce soltanto la metà di quei danni. 
L'esplosione si diffonde oltre gli angoli e danneggia ogni oggetto vulnerabile al tipo di danno scelto nell'area che non sia indossato o trasportato.

\DndSpellHeader{Verità Assoluta}
{Divinazione di 9° livello}
{1 azione}
{Incantatore (Raggio 30 metri)}
{V,S}
{Concentrazione, 1 ora}
L'incantatore emana un'aura di pura verità che avvolge la sua persona e si espande in un raggio di 30 metri. \\ Durante la durata dell'incantesimo, ogni creatura all'interno dell'area di effetto deve superare un tiro salvezza su Saggezza o essere obbligata a dire la verità in risposta a qualsiasi domanda posta da chiunque nell'area, incluso l'incantatore. \\ Le creature che falliscono il tiro salvezza non possono rifiutarsi di rispondere in alcun modo durante la durata dell'incantesimo.\\ Le creature che superano il tiro salvezza non possono comunque mentire, ma al massimo possono rifiutarsi di rispondere. \\ Una volta terminata la durata dell'incantesimo o quando l'incantatore interrompe la concentrazione, l'aura di verità si dissipa e le creature precedentemente soggette all'incantesimo possono tornare a mentire o ingannare normalmente.
