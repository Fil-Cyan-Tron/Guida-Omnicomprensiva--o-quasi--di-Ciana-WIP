\chapter{Monaco}

\begin{DndReadAloud}
  \it
  "Li vedi questi?" il Monaco cieco mostrò i suoi pugni, splendenti di energia verde, al povero borseggiatore colto in flagrante. \\ "Questo è Porta e questo è Fogli."
\end{DndReadAloud}

\section{Privilegi di Classe Opzionali}

\DndDropCapLine{A}{}h, il Monaco, una delle mie classi preferite da giocare, purtroppo a mio parere un po' troppo underpowered in termini di disponibilità di risorse e dipendenza dalle statistiche, quindi visto il mio grande amore per questa classe ho deciso di buffarla perchè sì.

\subsection{Riserva di ki aumentata}

Il numero massimo di punti ki di un Monaco è pari al suo livello + il suo bonus di competenza.

\subsection{Arti Marziali ampliate}

Un Monaco competente può usare un punto ki per aggiungere il suo modificatore di Destrezza alle prove di caratteristica di Atletica per afferrare una creatura.

\section{Via dell'Ombra Redenta}

\subsection{Pentimento e Redenzione}

\DndDropCapLine{S}{}ebbene tutti i monaci condividano la saggezza degli antichi maestri, ci sono tradizioni che si allontanano dalla loro illuminazione: è questo il caso della Via dell'Ombra. \\ Talvolta però accade che un Monaco, dopo aver sfiorato la morte ad esempio, un Monaco può decidere di allontanarsi da queste pratiche e riscoprire la dimensione più meditativa, pietosa e naturale delle arti del ki.

\subsection{Magia del ki}

Tra i molti privilegi concessi dalla via dell'Ombra, vi è quello di lanciare alcuni incantesimi, ma si tratta di un potere oppresso e incompleto. Una volta che un Monaco si libera dall'influenza dell'oscurità, può finalmente imparare a incanalare il suo ki per lanciare incantesimi più raffinati.

\paragraph{Trucchetti} Un Monaco Redento conosce due trucchetti a sua scelta tratti dalla lista degli incantesimi del druido. Apprende un trucchetto da druido aggiuntivo a sua scelta al 10° livello.

\paragraph{Slot incantesimo}La tabella indica quanti slot incantesimo possiede un Monaco Redento per lanciare i suoi incantesimi di 1° livello e di livello superiore. \\ Per lanciare uno di questi incantesimi, il Monaco deve spendere uno slot incantesimo di livello pari o superiore al livello dell'incantesimo. Per esempio, se un Monaco ha preparato l'incantesimo di 1° livello \textit{Scudo} e possiede uno slot incantesimo di 1° livello e uno slot incantesimo di 2° livello, può lanciare \textit{Scudo} usando uno qualsiasi dei due slot. \\ Il Monaco recupera tutti gli slot incantesimo spesi quando completa un riposo lungo.

\paragraph{Incantesimi preparati di 1° livello e superiore}Un Monaco Redento prepara tre incantesimi di 1° livello a sua scelta dalla lista di incantesimi del druido. La colonna "Incantesimi Preparati" nella tabella indica quanti incantesimi di 1° livello o di livello superiore un Monaco può preparare. Ognuno di questi incantesimi deve appartenere a un livello di cui il Monaco possiede degli slot incantesimo.\\ Ogni volta che il Monaco completa un riposo lungo, può preparare incantesimi diversi della lista degli incantesimi del druido. Tutti gli incantesimi preparati devono essere di un livello di cui il Monaco possiede almeno uno slot incantesimo.

\paragraph{Caratteristica da incantatore}Saggezza è la caratteristica da incantatore usata per gli incantesimi da Monaco Redento. Il potere dei suoi incantesimi deriva dalla sua sintonia con il Ki intorno a lui. Un Monaco usa Saggezza ogni volta che un incantesimo fa riferimento alla sua caratteristica da incantatore. Usa inoltre il suo modificatore di Saggezza per definire la CD del tiro salvezza di un incantesimo da Monaco da lui lanciato e quando effettua un tiro per colpire con un incantesimo.

\paragraph{CD del tiro salvezza degli incantesimi}= 8 + il bonus di competenza del Monaco + il modificatore di Saggezza del Monaco. 

\paragraph{Modificatore di attacco dell'incantesimo}= il bonus di competenza del Monaco + il modificatore di Saggezza del Monaco

\begin{DndTable}[header= Slot Incantesimo per livello]{XXXXXXXXXX}
  Livello da Monaco & & Trucchetti conosciuti & & Incantesimi preparati & & 1° Lv. & 2° Lv. & 3° Lv. & 4° Lv.\\
  3° & & 2 & & 3 & & 2 & - & - & - \\
  4° & & 2 & & 4 & & 3 & - & - & - \\
  5° & & 2 & & 4 & & 3 & - & - & - \\
  6° & & 2 & & 4 & & 3 & - & - & - \\
  7° & & 2 & & 5 & & 4 & 2 & - & - \\
  8° & & 2 & & 6 & & 4 & 2 & - & - \\
  9° & & 2 & & 6 & & 4 & 2 & - & - \\
  10° & & 3 & & 7 & & 4 & 3 & - & - \\
  11° & & 3 & & 8 & & 4 & 3 & - & - \\
  12° & & 3 & & 8 & & 4 & 3 & - & - \\
  13° & & 3 & & 9 & & 4 & 3 & 2 & - \\
  14° & & 3 & & 10 & & 4 & 3 & 2 & - \\
  15° & & 3 & & 10 & & 4 & 3 & 2 & - \\
  16° & & 3 & & 11 & & 4 & 3 & 3 & - \\
  17° & & 3 & & 11 & & 4 & 3 & 3 & - \\
  18° & & 3 & & 11 & & 4 & 3 & 3 & - \\
  19° & & 3 & & 12 & & 4 & 3 & 3 & 1 \\
  20° & & 3 & & 13 & & 4 & 3 & 3 & 1 \\
\end{DndTable}

\subsection{Aura viva}

A partire dal 3° livello, il ki di un Monaco Redento si manifesta visualmente come un'aura intorno a lui, che sia cosciente o incosciente. Questa funge da fonte di luce fioca per tutte le creature non ostili. \\ Tutti i danni inferti dal Monaco a creature viventi sono danni non letali.

\subsection{Auree nell'oscurità}

A partire dal 6° livello, il Monaco perde la sua visione. Guadagna la capacità di vedere le auree delle creature viventi con la sua mente e può ricavarne diverse informazioni, come l'allineamento, i punti ferita, la capacità di lanciare incantesimi e la forza magica. Tuttavia grazie alla sua sintonia con il mondo circostante, guadagna vista cieca nel raggio di 18m. \\ A causa della sua cecità, quando usa il privilegio "Deviare i proiettili" il Monaco non può più rilanciare indietro i proiettili neutralizzati.
\paragraph{Consigli per il DM}È consigliabile comunicare al giocatore queste informazioni in modo implicito, tramite ad esempio una descrizione dell'aura della creatura, piuttosto che dargliele direttamente.

\subsection{Nulla si crea, nulla si distrugge, tutto si trasforma}

A partire dall'11° livello, una volta per riposo lungo, quando il Monaco vede l'aura di una creatura vivente scendere a 0 punti ferita, può spendere un turno in concentrazione per entrare in sintonia con il suo corpo e assorbire da essa un numero di punti ki pari al suo bonus di competenza, ma per ogni punto ki recuperato in questo modo perde lui stesso due punti ferita. \\ In qualunque momento, il Monaco può convertire i suoi punti ki in slot incantesimo temporanei e viceversa, con un tasso di un punto ki per livello (purchè possieda già uno slot di quel livello).

\subsection{Sacrificio assoluto}

A partire dal 16° livello, con un atto altruistico definitivo, il Monaco può attingere a tutto il suo ki per poi farlo esplodere con un'emanazione del raggio di 15 metri. Tutti gli alleati morti che siano all'interno dell'emanazione sono riportati in vita, come se fossero soggetti ad un incantesimo \textit{Resurrezione Pura}. \\ Il Monaco viene completamente distrutto. Un Monaco distrutto in tal modo non può più tornare in vita, nemmeno tramite un incantesimo \textit{Desiderio} o \textit{Miracolo} o grazie al potere di una divinità. \\ Inoltre, il nome del Monaco può essere pronunciato ma non potrà più essere scritto. Tutti i riferimenti scritti del suo nome divengono nient'altro che spazi bianchi e tutti gli oggetti magici con cui era in sintonia perdono ogni proprietà magica. \\ Tutto ciò che rimane del Monaco dopo aver usato questo privilegio è un sacchetto di semi di \textit{Principessa Serena} con su cucita l'unica testimonianza scritta del suo nome.

