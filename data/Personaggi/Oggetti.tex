
\chapter{Oggetti magici}

\section{Oggetti legati ai personaggi dei giocatori}

%\section{Anello protettore di Sophos il Savio}

%\textit{Artefatto meraviglioso leggendario, richiede sintonia con un Incantatore che stia venendo divorato dalla magia. \\ Questo oggetto è un estratto dalla Biblioteca Omnicomprensiva di Ker.}

% \DndDropCapLine{D}{}urante le sue ricerche per scoprire quanto più possibile sulla magia, il mago Ælinor Könungru si è imbattuto in un incantesimo estremamente antico e potente, che appena lanciato ha cominciato a consumarne il corpo. Per fortuna è riuscito a contenerlo e ad isolarlo nel proprio braccio grazie al potere dell'Anello Protettore di Sophos il Savio. L'incantesimo non è tuttavia sparito, e i suoi segni sono rimasti impressi sulla pelle del mago.

% \begin{DndReadAloud}
%   \it
%   L'Anello Protettore di Sophos il Savio permette a chi lo indossa di attingere ad una fonte di antica energia racchiusa nella sua gemma, che reagisce con alcuni tipi di incantesimi protettivi alimentandoli senza richiedere energia all'incantatore.
% \end{DndReadAloud}

% \subsection{Potere passivo}
% L'anello può sostenere la concentrazione necessaria al mantenimento dell'incantesimo \textit{Limita Magie}. 

% \subsection{Potere Attivo}
% \textit{Requisito: 4 livelli in sintonia con l'anello.}\\
% Un numero di volte pari al proprio bonus di competenza per riposo lungo, l'incantatore che usa l'anello può lanciare l'incantesimo \textit{scudo} senza consumare slot.

% \subsection{Coscienza}
% \textit{Requisito: 12 livelli in sintonia con l'anello.}\\
% L'anello è considerato senziente, la sua personalità è simile a quella di Sophos il Savio ma non ne possiede i ricordi o le conoscenze.

% \subsection{Corruzione mortale}
% \textit{Requisito: 12 livelli in sintonia con l'anello.}\\
% Una volta per riposo lungo, l'Incantatore può usare la sua reazione per spostare l'incantesimo \textit{Limita Magia} dal suo corpo ad un punto esterno. Fare questo tuttavia libera l'incantesimo nel suo corpo, il quale finché la bolla non viene ripristinata subisce 1d6+2 danni necrotici ogni volta che lancia un incantesimo e 1d4 danni necrotici ogni turno. Se l'incantatore mantiene un'altra concentrazione, ogni turno deve passare un TS su costituzione con CD pari alla propria CD del tiro salvezza contro gli incantesimi, altrimenti la perde. Se l'incantatore muore per questi danni, si trasforma in un divorato.

%\section{STOCCO DI SARZEE}

%\textit{Artefatto meraviglioso leggendario, richiede sintonia con un Bardo}

%\section{Prigione Fluida di PATRONO DI SIMONE}

%\textit{Artefatto meraviglioso leggendario, richiede sintonia con un Warlock}

\section{Oggetti non legati ai personaggi dei giocatori}

\subsection{Guanti del CleptoMana}
\textit{Richiede sintonia con un Warlock} 

\subsubsection{Un regalo da molto lontano... ma in che direzione?}
Osservando questi guanti, è normale chiedersi da dove vengano: la loro fattura è inusuale, precisa, come se fossero stati costruiti con attrezzi perduti... o non ancora inventati? \\
Un oggetto bizzarro, pensando al suo potere sembra strano che non ci siano leggende al riguardo, ma di certo l'incantatore in suo possesso ha tutte le carte giuste per scriversene una sua…

\subsubsection{Ladro di Magia}
Un numero di volte pari al suo bonus di competenza per riposo lungo, l'incantatore può scegliere una creatura che egli sia in grado di vedere e usare la sua azione per tentare utilizzare uno slot incantesimo del bersaglio di 5° livello o inferiore. \\
Quando lo fa, come parte della stessa azione lancia un trucchetto a partire dalla posizione del bersaglio. L'incantatore deve essere in grado di vedere il bersaglio del suo trucchetto. Il trucchetto è considerato lanciato dall'incantatore con la sua caratteristica da incantatore. \\ Se il bersaglio non possiede slot di 5° livello o inferiore, i guanti non hanno alcun effetto e sia l'azione che l'utilizzo sono considerati utilizzati.

\subsubsection{Controincantesimo}
Un numero di volte pari a metà del suo bonus di competenza per riposo lungo, l'incantatore può lanciare l'incantesimo \textit{Controincantesimo} come incantesimo di 5° livello senza spendere slot incantesimo. \\ Come parte della stessa reazione, può lanciare un trucchetto a partire dalla posizione del bersaglio (vedi \textit{Ladro di Magia}).

\subsubsection{Pura Cleptoman(i)a}
Una volta per riposo lungo, l'incantatore può scegliere un altro incantatore che sia in grado di vedere e impadronirsi di un suo slot di 6° livello o superiore, convertendolo in uno slot di 5° livello. \\ Se il bersaglio non possiede slot di 6° livello o superiore, i guanti non hanno alcun effetto e sia l'azione che l'utilizzo sono considerati utilizzati.

\subsection{Occhi di Omatara}
\textit{Coppia di anelli che richiedono sintonia con due incantatori diversi contemporaneamente}

\subsubsection{Sotto lo sguardo della morte}
Anelli magici leggendari, la cui favola narra siano stati concessi da Omatara, dea della morte, a due incantatori fratelli che riuscirono a ingannarla. \\ Sempre secondo la leggenda, se un incantatore solo dovesse indossarli entrambi, Omatara in persona si preoccuperebbe di provvedere a dargli una morte rapida e fatidica, ma si tratta solo di una leggenda… vero? 

\subsubsection{L'occhio sinistro}
Come azione bonus, l'indossatore può scegliere una creatura senziente che sia in grado di vedere. \\ Per i prossimi 6 secondi, l'indossatore può percepire il mondo dagli occhi di quella creatura e controllarne lo sguardo. La creatura deve effettuare un TS su saggezza con CD 16; se lo fallisce, è accecata durante l'effetto di questo oggetto; L'incantatore può decidere di usare questo effetto senza accecare. \\ Questo privilegio può essere usato un numero di volte pari al bonus di competenza dell'indossatore per riposo breve. La creatura influenzata non può sapere da chi sta venendo influenzata.

\subsubsection{L'occhio destro}
Come azione, l'indossatore può toccare due volte l'anello e teletrasportarsi nell'ombra di una creatura senziente che possa vedere, potendo effettuare un attacco a mani nude. \\ Questo privilegio può essere usato un numero di volte pari al bonus di competenza dell'indossatore per riposo breve.

\subsubsection{Stereoscopia}
Se i due anelli sono indossati da persone diverse, come azione l'indossatore dell'occhio destro può girare l'anello sul dito e percepire il mondo dalla prospettiva dell'indossatore dell'occhio sinistro, non importa dove esso si trovi. A sua volta, l'indossatore dell'occhio sinistro può usare la sua azione per teletrasportarsi nell'ombra dell'indossatore dell'occhio destro. \\ Usare questo privilegio conta come un utilizzo. Se uno dei due anelli sta venendo utilizzato, l'altro indossatore ne è a conoscenza.

\subsection{Frattura}
\textit{Pugnale magico (+0, 1d6 danni magici taglienti) che richiede sintonia con un Ladro, la sua potenza dipende dal livello del Ladro, oggetto meraviglioso, leggendario}

\subsubsection{Squarcio dimensionale}
A partire dal 1° livello, l'utilizzatore può usare un'azione bonus per tagliare lo spazio e aprire un portale di circa 30 cm di raggio per un punto entro 9m che è sia grado di vedere. Può attaccare attraverso questo portale, che si richiude alla fine del suo turno.

\subsubsection{Fenditura silenziosa}
A partire dal 4° livello, l'utilizzatore quando usa \textit{Squarcio dimensionale} può scegliere una creatura che sia in grado di vedere ed effettuare una prova di caratteristica di Destrezza (Furtività) contro la sua percezione passiva. Se la supera, il portale si apre dietro la creatura (che ne è inconsapevole) e il prossimo attacco effettuato attraverso di esso ha vantaggio.

\subsubsection{Strappo duraturo}
A partire dall'8° livello, i portali aperti con \textit{Squarcio dimensionale} permangono per un'ora, o fino a che l'utilizzatore non li chiude con un'azione bonus o usa di nuovo \textit{Squarcio dimensionale} per aprirne un altro.

\subsubsection{Gringott}
A partire dal 10° livello, l'utilizzatore può usare \textit{Squarcio dimensionale} per lanciare l'incantesimo \textit{Semipiano}. Il portale si apre ogni volta su un semipiano di nome "Gringott", legato al pugnale. Solo creature in sintonia con Frattura possono accedere a Gringott lanciando l'incantesimo \textit{Semipiano}.

\subsubsection{Espansione immobiliare}
A partire dal 12° livello, quando l'utilizzatore scopre un oggetto a cui è legato un semipiano, può usare la sua azione per effettuare una prova di caratteristica di Destrezza (Rapidità di mano) contro la CD per il TS contro gli incantesimi dell'incantatore che ha creato il semipiano.\\
Se ha successo, il semipiano diventa parte di Gringott. Una porta per esso appare su uno dei muri di Gringott scelto dall'incantatore. Se Gringott è già composta da diversi semipiani, la porta può apparire in uno qualsiasi di essi. L'utilizzatore può riarrangiare la struttura dei semipiani di Gringott a suo piacimento.\\
L'oggetto da cui è stato sottratto il semipiano mantiene le sue proprietà di porta per esso per un ultimo viaggio, dopodichè perde queste proprietà magiche, ma non viene distrutto e mantiene le altre, se ne aveva.

\subsubsection{Fendispazio}
A partire dal 16° livello, i danni di Frattura aumentano a 2d6 e \textit{Squarcio dimensionale} può essere usato per lanciare l'incantesimo \textit{Portale} verso una destinazione sullo stesso piano di esistenza.

\subsubsection{Fendispazio interplanare}
A partire dal 19° livello, \textit{Squarcio dimensionale} può essere usato per lanciare l'incantesimo \textit{Portale} senza limitazioni.

\subsection{Errori di battitura}
Esiste tutta una classe di oggetti magici che nascono quando il DM fa qualche errore di battitura. Ne incontrerete molti, e verranno tutti messi qui.

\subsubsection{Alabarba}
\textit{Arma magica comune}\\
Quando un personaggio impugna l'Alabarba, gli cresce istantaneamente una barba. Se questi possiede già una barba, a questa crescerà un'altra barba. Se il personaggio smette di impugnare l'Alabarba, la barba sparisce istantaneamente. Finchè impugna l'Alabarba, il personaggio ha un bonus di +1 alle prove di Carisma.\\
L'Alabarba è assolutamente indistinguibile da un'alabarda normale. Qualsiasi alabarda quando viene fabbricata ha una piccola possibilità (1\%) di diventare un'Alabarba.