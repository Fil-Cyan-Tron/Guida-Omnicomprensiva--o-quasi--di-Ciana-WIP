
\chapter{Oggetti magici}

\section{Oggetti legati ai personaggi dei giocatori}

Questi sono oggetti particolarmente studiati che sono fatti per essere legati ai personaggi creati dai miei giocatori (e a Fillianore e Cianar), diventano più forti man mano che il loro livello aumenta e sono parte della loro storia.

\subsection{Tavoletta di Minera Ioun}
\textit{Tavoletta interattiva (artefatto) meravigliosa leggendaria, richiede sintonia con un Sapiente specializzato in Strategia}

\subsubsection{Funzioni elementari}
A partire dal 1° livello, la tavoletta può essere usata per accedere ad una mappa accurata (ma non omnicomprensiva) della regione di mondo in cui si trova, un cannocchiale, una bussola, un astrolabio e un telescopio. Inoltre permette di scattare delle fotografie e di prendere appunti. Quando viene utilizzata, galleggia a mezz'aria e si muove seguendo l'utilizzatore.\\ Con un'azione, l'utilizzatore può lanciare l'incantesimo \textit{Scurovisione} sulla tavoletta, e guardare attraverso il suo schermo per condividerne gli effetti.\\ La tavoletta ha infine un punteggio di Intelligenza di 14, e ha competenza in tutte le abilità su intelligenza, condividendo il bonus di competenza dell'utilizzatore. Con un'azione bonus, essa può effettuare prove di caratteristica su Intelligenza.\\ Utilizzare la tavoletta richiede almeno una mano libera.

\subsubsection{Amo il fashion, infatti sono un fascista}
A partire dal 1° livello, con un'azione bonus, la tavoletta permette di cambiare l'aspetto estetico dei vestiti dell'utilizzatore. Le modifiche all'abbigliamento sono meramente estetiche non hanno alcun effetto sulle statistiche, tuttavia alla vista e al tatto le modifiche all'abbigliamento sono indistinguibili da un abito reale.

\subsubsection{Stasys}
A partire dal 4° livello, quando una creatura nel raggio di 6 metri dall'utilizzatore è bersaglio di un attacco con arma a distanza (con un proiettile di velocità comparabile ai riflessi umani), l'utilizzatore può usare la sua reazione per aggiungere 1d6 alla sua CA.\\
Se il colpo manca il bersaglio, esso rimane fermo a mezz'aria a 1,5 metri da esso fino alla fine del prossimo turno del bersaglio, dopodichè riparte con la stessa velocità e traiettoria.

\subsubsection{Golem Gufo}
A partire dal 4° livello, con un'azione l'utilizzatore può evocare un Golem Gufo (vedi appendice) che è in grado di controllare attraverso la tavoletta con un'azione bonus fintanto che esso si trovi entro 30m da lui.\\ Il Golem Gufo trasmette alla tavoletta suoni e immagini, agisce nello stesso turno dell'utilizzatore, ma a meno che l'utilizzatore non gli dia degli ordini usando la sua azione bonus, può compiere solo l'azione di Schivata.\\ Se l'utilizzatore è incapacitato, il Golem Gufo agisce autonomamente secondo quello che può ritenere essere il migliore corso d'azione.\\ Se il Golem Gufo viene abbattuto, l'utilizzatore deve passare un'ora di tempo a ripararlo per poterlo evocare di nuovo. Può essere evocato un solo Golem Gufo alla volta.

\subsubsection{Golem aiutanti}
A partire dall'8° livello l'utilizzatore può usare un'azione per lanciare l'incantesimo \textit{Servitore Inosservito} per evocare fino a tre Golem Aiutanti con le seguenti caratteristiche:
\paragraph{Aspetto} I Golem Aiutanti non sono invisibili
\paragraph{Classe armatura} la CA dei Golem Aiutanti è pari a 14
\paragraph{Punteggi di caratteristica} I Golem Aiutanti hanno un punteggio di 8 in ogni statistica
\paragraph{Punti ferita} I Golem Aiutanti hanno 30 punti ferita ciascuno.
Se un Golem viene distrutto, prima di poterlo rievocare l'utilizzatore deve passare un'ora a ripararlo.

\subsubsection{Analisi avanzata}
A partire dall'8° livello, con un'azione, l'utilizzatore può lanciare l'incantesimo \textit{Visione del Vero} sulla tavoletta, e guardare attraverso il suo schermo per condividerne gli effetti. Quando l'incantesimo termina, la tavoletta è disabilitata fino all'alba successiva per l'eccesso di informazioni.\\ A partire dal 16° livello, può farlo senza ripercussioni.

\subsubsection{Stasys+}
A partire dal 12° livello, con un'azione l'utilizzatore può lanciare una versione limitata dell'incantesimo \textit{Fermare il Tempo}.\\
Sceglie un bersaglio entro 9 metri e gli impone un tiro salvezza su Costituzione con CD pari a 8 + il suo bonus di competenza + il suo modificatore di Intelligenza. Se questo fallisce o sceglie di fallire, è congelato nel tempo fino alla fine del turno successivo dell'utilizzatore o finchè esso non vi pone fine con una reazione.\\
Quando l'effetto termina, il bersaglio subisce tutti i danni accumulati durante il blocco temporale.

\subsubsection{Controllare costrutti}
A partire dal 12° livello, con un'azione l'utilizzatore può tentare di far prendere il controllo di un costrutto entro 18 metri all'intelligenza artificiale della tavoletta.\\ Per farlo deve superare una prova di Intelligenza con CD determinata dal DM in base alle dimensioni e alla complessità del costrutto.\\
Se ha successo, l'intelligenza artificiale della tavoletta si trasferisce nel costrutto e può controllarlo come se fosse un suo corpo, tornando automaticamente nella tavoletta se questo viene distrutto.\\
Questo privilegio può essere usato una volta per riposo breve.

\subsubsection{Costrutti migliorati}
A partire dal 16° livello, i punti ferita del Golem Gufo aumentano a 52 (8d8+10), i suoi punteggi di Forza e Destrezza aumentano a 20 e questo può portare in volo fino a due creature di taglia media.\\
Uno dei Golem Aiutanti può essere sostituito da un Golem Guerriero (vedi appendice).

\subsubsection{Macrocostrutto}
A partire dal 19° livello, la tavoletta può prendere il controllo di cinque costrutti contemporaneamente.\\ Se lo fa, può assemblarli in un cazzo di Megazord (vedi appendice).

\subsubsection{Stasys++}
A partire dal 19° livello, con un'azione l'utilizzatore può lanciare l'incantesimo \textit{Fermare il Tempo}.\\
Questo privilegio può essere usato una volta per riposo lungo.

\subsection{Quaderno del Demiurgo}
\textit{Quaderno (artefatto) meraviglioso leggendario, richiede sintonia con un Bardo del Collegio della Creazione o un Warlock del Benefattore Temporale}\\
Un quaderno che pare avere un numero illimitato di pagine, di qualsiasi colore.
Staccandone una pagina e realizzando un origami esso acquisisce certe caratteristiche magiche.

\subsubsection{Origami}
A partire dal 1° livello, l'utilizzatore può staccare una pagina per realizzare un origami magico. La realizzazione di un origami, in base alla sua complessità, può richiedere tra un minuto e una mezzora.\\
Quando l'utilizzatore realizza l'origami di una creatura con Grado Sfida minore o uguale al suo bonus di competenza, può lanciare l'incantesimo \textit{Trova Famiglio} e animarlo in modo da usarlo come famiglio.\\
Il famiglio ha punti ferita massimi pari al livello dell'utilizzatore. Quando scende a 0 punti ferita, l'origami viene distrutto.\\
Quando l'origami viene distrutto o l'utilizzatore decide di farlo tornare un normale origami, può riutilizzare questo privilegio.

\subsubsection{Segreti e segrete}
Moltissime strutture a Eovras hanno un leggio di pietra davanti alla loro entrata.\\
A partire dal 1° livello, se l'utilizzatore appoggia il quaderno aperto su questo leggio, sulla pagina aperta verrà disegnata magicamente una mappa della struttura, che segnerà in tempo reale la posizione del quaderno fintanto che questo si troverà al suo interno.

\subsubsection{Fogli a millemila}
A partire dal 4° livello, come parte della creazione di un origami, l'utilizzatore può alterarne la taglia da minuscolo a enorme. Successivamente, può usare un'azione per toccarlo e alterarla ulteriormente.

\subsubsection{L'inchiostro è più denso del sangue}
A partire dall'8° livello, quando l'utilizzatore crea un origami come famiglio può trasformarlo in una creatura in carne ed ossa.\\
Quando questa scende a 0 punti ferita o viene congedata, torna a essere un origami.

\subsubsection{Cartapentiere}
A partire dal 12° livello, l'utilizzatore può usare un origami per usare una seconda volta il privilegio del Collegio della Creazione \textit{Compimento della Creazione}.

\subsubsection{Animato a mano}
A partire dal 16° livello, l'utilizzatore può usare il quaderno per usare su due oggetti il privilegio del Collegio della Creazione \textit{Creazione Animata} e usare la stessa azione bonus per impartire ordini diversi a ciascuno.

\subsubsection{Iperrealismo}
A partire dal 19° livello, quando l'utilizzatore realizza un origami che rappresenta un oggetto inanimato, può usarlo per lanciare l'incantesimo \textit{Immagine Maggiore}. In questo caso, l'oggetto può essere di dimensione fino a Gargantuesca.\\
Se nessun essere senziente svela l'illusione e il Bardo mantiene concentrazione per una durata pari a un'ora per ogni grado di dimensione, (minuscolo o meno: un'ora, piccolo: due ore, e così via) l'oggetto smette di essere un'illusione e diventa reale.

\subsection{Prigione fluida di Absynthe}
\textit{Arma simbiotica (artefatto) leggendaria, richiede sintonia con un Warlock della Lama del Sortilegio o un Paladino del Giuramento di Vendetta}

\subsection{Borsa degli arnesi stupefacenti di Skardy}
\textit{Borsa di oggetti (artefatto) meravigliosi leggendari, richiede sintonia con un Druido del Circolo della Luna}

\subsection{Sintetizzatore d'Ebano di Tauriel Halamis}
\textit{Anello (artefatto) leggendario, richiede sintonia con un Warlock del Signore dell'Assurdo}

\subsubsection{Sintesi assoluta}
A partire dal 1° livello, l'utilizzatore può far prendere all'anello la forma di un qualsiasi strumento musicale (di dimensioni contenute) in Ebano, paragonabile a uno strumento "vero" di ottima fattura.\\ Quando è in questa forma, l'utilizzatore può usarlo come focus per i suoi incantesimi da Warlock.\\ Finchè è in sintonia con questo oggetto, l'utilizzatore è competente nell'uso di qualsiasi strumento musicale, anche quelli in cui l'anello non è in grado di trasformarsi (ad esempio, un organo a canne).

\subsection{Scudo della Lealtà di Artorias}
\textit{Amuleto (artefatto) leggendario, richiede sintonia con un Guerriero in grado di trasformarsi in un Lupo}

\section{Oggetti non legati ai personaggi dei giocatori}

\subsection{Errori di battitura}
Esiste tutta una classe di oggetti magici che nascono quando il DM fa qualche errore di battitura. Ne incontrerete molti, e verranno tutti messi qui.

\subsubsection{Alabarba}
\textit{Arma magica comune}\\
Quando un personaggio impugna l'Alabarba, gli cresce istantaneamente una barba. Se questi possiede già una barba, a questa crescerà un'altra barba. Se il personaggio smette di impugnare l'Alabarba, la barba sparisce istantaneamente. Finchè impugna l'Alabarba, il personaggio ha un bonus di +1 alle prove di Carisma.\\
L'Alabarba è assolutamente indistinguibile da un'alabarda normale. Qualsiasi alabarda quando viene fabbricata ha una piccola possibilità (1\%) di diventare un'Alabarba.

\subsection{Frattura}
\textit{Pugnale magico (+0, 1d6 danni magici taglienti) che richiede sintonia con un Ladro, la sua potenza dipende dal livello del Ladro, oggetto meraviglioso, leggendario}

\subsubsection{Squarcio dimensionale}
A partire dal 1° livello, l'utilizzatore può usare un'azione bonus per tagliare lo spazio e aprire un portale di circa 30 cm di raggio per un punto entro 9m che è sia grado di vedere. Può attaccare attraverso questo portale, che si richiude alla fine del suo turno.

\subsubsection{Fenditura silenziosa}
A partire dal 4° livello, l'utilizzatore quando usa \textit{Squarcio dimensionale} può scegliere una creatura che sia in grado di vedere ed effettuare una prova di caratteristica di Destrezza (Furtività) contro la sua percezione passiva. Se la supera, il portale si apre dietro la creatura (che ne è inconsapevole) e il prossimo attacco effettuato attraverso di esso ha vantaggio.

\subsubsection{Strappo duraturo}
A partire dall'8° livello, i portali aperti con \textit{Squarcio dimensionale} permangono per un'ora, o fino a che l'utilizzatore non li chiude con un'azione bonus o usa di nuovo \textit{Squarcio dimensionale} per aprirne un altro.

\subsubsection{Gringott}
A partire dal 10° livello, l'utilizzatore può usare \textit{Squarcio dimensionale} per lanciare l'incantesimo \textit{Semipiano}. Il portale si apre ogni volta su un semipiano di nome "Gringott", legato al pugnale. Solo creature in sintonia con Frattura possono accedere a Gringott lanciando l'incantesimo \textit{Semipiano}.

\subsubsection{Espansione immobiliare}
A partire dal 12° livello, quando l'utilizzatore scopre un oggetto a cui è legato un semipiano, può usare la sua azione per effettuare una prova di caratteristica di Destrezza (Rapidità di mano) contro la CD per il TS contro gli incantesimi dell'incantatore che ha creato il semipiano.\\
Se ha successo, il semipiano diventa parte di Gringott. Una porta per esso appare su uno dei muri di Gringott scelto dall'incantatore. Se Gringott è già composta da diversi semipiani, la porta può apparire in uno qualsiasi di essi. L'utilizzatore può riarrangiare la struttura dei semipiani di Gringott a suo piacimento.\\
L'oggetto da cui è stato sottratto il semipiano mantiene le sue proprietà di porta per esso per un ultimo viaggio, dopodichè perde queste proprietà magiche, ma non viene distrutto e mantiene le altre, se ne aveva.

\subsubsection{Fendispazio}
A partire dal 16° livello, i danni di Frattura aumentano a 2d6 e \textit{Squarcio dimensionale} può essere usato per lanciare l'incantesimo \textit{Portale} verso una destinazione sullo stesso piano di esistenza.

\subsubsection{Fendispazio interplanare}
A partire dal 19° livello, \textit{Squarcio dimensionale} può essere usato per lanciare l'incantesimo \textit{Portale} senza limitazioni.

\subsection{Guanti del CleptoMana}
\textit{Richiede sintonia con un Warlock} 

\subsubsection{Un regalo da molto lontano... ma in che direzione?}
Osservando questi guanti, è normale chiedersi da dove vengano: la loro fattura è inusuale, precisa, come se fossero stati costruiti con attrezzi perduti... o non ancora inventati? \\
Un oggetto bizzarro, pensando al suo potere sembra strano che non ci siano leggende al riguardo, ma di certo l'incantatore in suo possesso ha tutte le carte giuste per scriversene una sua…

\subsubsection{Ladro di Magia}
Un numero di volte pari al suo bonus di competenza per riposo lungo, l'incantatore può scegliere una creatura che egli sia in grado di vedere e usare la sua azione per tentare utilizzare uno slot incantesimo del bersaglio di 5° livello o inferiore. \\
Quando lo fa, come parte della stessa azione lancia un trucchetto a partire dalla posizione del bersaglio. L'incantatore deve essere in grado di vedere il bersaglio del suo trucchetto. Il trucchetto è considerato lanciato dall'incantatore con la sua caratteristica da incantatore. \\ Se il bersaglio non possiede slot di 5° livello o inferiore, i guanti non hanno alcun effetto e sia l'azione che l'utilizzo sono considerati utilizzati.

\subsubsection{Controincantesimo}
Un numero di volte pari a metà del suo bonus di competenza per riposo lungo, l'incantatore può lanciare l'incantesimo \textit{Controincantesimo} come incantesimo di 5° livello senza spendere slot incantesimo. \\ Come parte della stessa reazione, può lanciare un trucchetto a partire dalla posizione del bersaglio (vedi \textit{Ladro di Magia}).

\subsubsection{Pura Cleptoman(i)a}
Una volta per riposo lungo, l'incantatore può scegliere un altro incantatore che sia in grado di vedere e impadronirsi di un suo slot di 6° livello o superiore, convertendolo in uno slot di 5° livello. \\ Se il bersaglio non possiede slot di 6° livello o superiore, i guanti non hanno alcun effetto e sia l'azione che l'utilizzo sono considerati utilizzati.

\subsection{Occhi di Omatara}
\textit{Coppia di anelli leggendari, richiedono sintonia con due incantatori diversi contemporaneamente}

\subsubsection{Sotto lo sguardo della morte}
Anelli magici la cui favola narra siano stati concessi da Omatara, dea della morte, a due incantatori fratelli che riuscirono a ingannarla. \\ Sempre secondo la leggenda, se un incantatore solo dovesse indossarli entrambi, Omatara in persona si preoccuperebbe di provvedere a dargli una morte rapida e fatidica, ma si tratta solo di una leggenda… vero? 

\subsubsection{L'occhio sinistro}
Come azione bonus, l'indossatore può scegliere una creatura senziente che sia in grado di vedere. \\ Per i prossimi 6 secondi, l'indossatore può percepire il mondo dagli occhi di quella creatura e controllarne lo sguardo. La creatura deve effettuare un TS su saggezza con CD 16; se lo fallisce, è accecata durante l'effetto di questo oggetto; L'incantatore può decidere di usare questo effetto senza accecare. \\ Questo privilegio può essere usato un numero di volte pari al bonus di competenza dell'indossatore per riposo breve. La creatura influenzata non può sapere da chi sta venendo influenzata.

\subsubsection{L'occhio destro}
Come azione, l'indossatore può toccare due volte l'anello e teletrasportarsi nell'ombra di una creatura senziente che possa vedere, potendo effettuare un attacco a mani nude. \\ Questo privilegio può essere usato un numero di volte pari al bonus di competenza dell'indossatore per riposo breve.

\subsubsection{Stereoscopia}
Se i due anelli sono indossati da persone diverse, come azione l'indossatore dell'occhio destro può girare l'anello sul dito e percepire il mondo dalla prospettiva dell'indossatore dell'occhio sinistro, non importa dove esso si trovi. A sua volta, l'indossatore dell'occhio sinistro può usare la sua azione per teletrasportarsi nell'ombra dell'indossatore dell'occhio destro. \\ Usare questo privilegio conta come un utilizzo. Se uno dei due anelli sta venendo utilizzato, l'altro indossatore ne è a conoscenza.