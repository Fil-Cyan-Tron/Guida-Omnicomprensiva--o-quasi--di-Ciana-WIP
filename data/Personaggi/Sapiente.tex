\nchapter{Sapiente}
\textit{Creato da u/LaserLlama, tradotto da Davide Borra e ribilanciato dal sottoscritto}

\subsection{Menti magnifiche}
Ci sono molte persone magnificamente intelligenti al mondo, ma poche sono veri Sapienti. Nati con l'innato desiderio di conoscere tutto ciò che possono, e il potenziale per un intelletto geniale, i Sapienti spendono la loro intera vita imparando tutto ciò che chi li circonda è disposto ad insegnargli. Spesso riconoscibili ad una giovane età, la loro insaziabile sete di conoscenza li conduce nelle più grandi librerie, università e nei principali luoghi di sapere. Desiderano viaggiare per scoprire i più reconditi segreti del mondo, spesso dedicando la propria vita all'avventura. Per un Sapiente, nessun prezzo è troppo alto per la promessa della scoperta.

\subsection{Intenso interesse}
I Sapienti sono molto concentrati nell'area di studi che anno scelto, e spesso diventano ossessionati dal conoscere tutto cià che possono sulla loro specialità. Nella loro missione per la scoperta, i Sapienti sono disposti a rinunciare ad ogni loro convinzione, politica o religiosa che sia, per acquisire le informazioni che anelano. Per essi, il loro deriderio di conoscenza è più importante della lealtà verso un gruppo o un'ideologia. \\
Spesso pagando un grande prezzo, i Sapienti non si fermeranno davanti a nulla finché non avranno fatto una scoperta rivoluzionaria nella loro ariea di studio. non è raro incontrare uno di questi studiosi lontano dalla sicurezza di un'universita e le sue librerie.

\subsection{Creare un Sapiente}
Durante la creazione di un Sapiente, bisognerebbe considerare la loro istruzione e il loro livello di educazione formale. Era la promessa della migliore università il denaro può comprare? Oppure, era un figlio della strada, disposto a combattere per ogni frammento di conoscenza su cui potrebbe mettere le mani? Forse la loro mente è stata allenata fin dalla nascita per essere il perfetto strumenti di analisi per una nobile famiglia.\\
Considera anche perché il tuo Sapiente ha scelto di affidarsi al proprio intelletto anziché usare i propri doni nello studio dell'arcano o al servizio di un potere più alto. Sono condannati a non poter lanciare neanche la più semplice magia? O hanno giurato di non affidarsi mai al potere che ha distrutto la loro famiglia?\\
Infine, perchè ha scelto di diventare un avventuriero invece che vivere una vita accademica? Ha scelto di andare oltre il normale studio e dedicarsi alle infinite scoperte dell'avventura? Oppure ha sempre avuto una passione per lo studio del mondo?

\begin{DndSidebar}{Sapiente e Multiclasse}
\paragraph{Punteggio di caratteristica minimo} Il personaggio multiclasse deve avere un punteggio di Intelligenza pari o superiore a 13 per acquisire un livello in questa classe o per acquisire un livello in un'altra classe se é gia un Sapiente,
\paragraph{Competenze ottenute} Se il Sapiente non é la classe iniziale del personaggio, ecco quali competenze ottiene quando acquisisce il suo primo livello da Sapiente: armature leggere, un'abilità a scelta dalla lista delle abilità da Sapiente e un set di strumenti da artigiano a scelta.
\end{DndSidebar}

\section{Creazione}

\subsection{Punti ferita}
\paragraph{Dadi Vita} 1d8 per livello da Sapiente
\paragraph{Punti ferita al 1° livello} 8 + il modificatore di Costituzione del Sapiente
\paragraph{Punti ferita ai livelli successivi} 5 (1d8) + il modificatore di Costituzione del Sapiente

\subsection{Competenze}
\paragraph{Armature} Armature leggere
\paragraph{Armi} Armi semplici, spade corte
\paragraph{Strumenti} Un set di strumenti da artigiano a scelta
\paragraph{Tiri Salvezza} Intelligenza, Saggezza
\paragraph{Abilità} Tre a scelta tra Arcano, Storia, Indagare, Intuizione,  Medicina, Natura, Persuasione o Religione

\subsection{Equipaggiamento iniziale}
Inizia con l'equipaggiamento seguente, in aggiunta all'equipaggiamento conferito dal suo background:
\begin{itemize}
    \item {\it(a)} un'arma semplice a scelta o {\it(b)} una spada corta.
    \item {\it(a)} una balestra leggera e 20 frecce o {\it(b)} due pugnali.
    \item un set di strumenti da artigiano a scelta
    \item un'armatura di cuoio, e una borsa da studioso
    \item 5d4 x 10 mo
\end{itemize}

\subsection{Creazione Rapida}
Puoi creare rapidamente un Sapiente usando le seguenti linee guida. Prima, metti il tuo punteggio di Intelligenza al punteggio più alto, seguito da Destrezza. Secondo, scegli il background nobile.

\subsection{Privilegi del Sapiente}
I privilegi ottenuti dal Sapiente a ogni livello sono riassunti in questa tabella.

\begin{DndReadAloud}
    \begin{DndTable}[header=Sapiente]{cccc}
        Lv & PB& Privilegi di Classe & \begin{tabular}{cc}\ Ambiti di\ \\Ricerca\end{tabular}\\
        1° & +2 & \begin{tabular}{c}Attenta Analisi,\\Difesa predittiva                     \end{tabular}&--\\
        2° & +2 & \begin{tabular}{c}Ambiti di Ricerca,\\ Intelletto Maestoso (d4)   \end{tabular}& 2\\
        3° & +2 & \begin{tabular}{c}Disciplina Accademica                           \end{tabular}& 2\\
        4° & +2 & \begin{tabular}{c}Aumento dei punteggi\\ di caratteristica          \end{tabular}& 3\\
        5° & +3 & \begin{tabular}{c}Riflessi Accelerati (2)\\Studio Rapido\\Intelletto maestoso (d6)                         \end{tabular}& 3\\
        6° & +3 & \begin{tabular}{c}Privilegio della\\Disciplina Accademica                           \end{tabular}& 3\\
        7° & +3 & \begin{tabular}{c}Percezione Affinata                              \end{tabular} & 3\\
        8° & +3 & \begin{tabular}{c}Aumento dei punteggi\\ di caratteristica          \end{tabular}& 4\\
        9° & +4 & \begin{tabular}{c}Flash di Genialità                        \end{tabular} & 4\\
        10° & +4 & \begin{tabular}{c}Esperto di Previsioni\\Intelletto maestoso (d8)                          \end{tabular}& 4\\
        11° & +4 & \begin{tabular}{c}Potente Osservazione                             \end{tabular}& 4\\
        12° & +4 & \begin{tabular}{c}Aumento dei punteggi\\ di caratteristica         \end{tabular}& 5\\
        13° & +5 & \begin{tabular}{c}Privilegio della\\Disciplina Accademica                       \end{tabular}& 5\\
        14° & +5 & \begin{tabular}{c}Volontà di Ferro                        \end{tabular}& 5\\
        15° & +5 & \begin{tabular}{c}Intelletto Maestoso (d10)                             \end{tabular}& 5\\
        16° & +5 & \begin{tabular}{c}Aumento dei punteggi\\ di caratteristica         \end{tabular}& 6\\
        17° & +6 & \begin{tabular}{c}Privilegio della\\Disciplina Accademica,\\Riflessi Accelerati (3)                       \end{tabular}& 6\\
        18° & +6 & \begin{tabular}{c}Profonda Comprensione                          \end{tabular}& 6\\
        19° & +6 & \begin{tabular}{c}Aumento dei punteggi\\ di caratteristica         \end{tabular}& 6\\
        20° & +6 & \begin{tabular}{c}Genio Indiscusso                             \end{tabular}& 6\\
    \end{DndTable}
\end{DndReadAloud}

\subsection{Attenta Analisi}
Il Sapiente può rapidamente analizzare un nemico. A partire dal 1° livello, un numero di volte pari al proprio bonus di di competenza più il suo modificatore di intelligenza ogni riposo breve, può usare un'azione bonus per analizzare una creatura che è in grado di vedere entro 18 metri, designandola come suo Mark. Finché si concentra sul suo Mark, come se si stesse concentrando su un incantesimo, il Sapiente ottiene i seguenti benefici:
\begin{itemize}
\item quando colpisce il Mark con un attacco o lo osserva per 1 minuto, può apprendere una delle seguenti caratteristiche a sua scelta: punteggio di caratteristica più alto, punteggio di caratteristica più basso, Classe Armatura, velocità, punti ferita massimi o tipo di creatura.
\item può usare Intelligenza, al posto di Forza o Destrezza, per i tiri per colpire e per i danni contro il Mark.
\item ha vantaggio su prove di caratteristica su Intelligenza e Saggezza per ottenere informazioni sul Mark.
\item può fare un tiro salvezza su Intelligenza, anziché su Costituzione per mantenere la concentrazione sul Mark.
\end{itemize}
La creatura rimane indefinitamente il Mark del Sapiente, almeno finché il Sapiente non perde la concentrazione, la creatura non esce dal suo campo visivo, o il Sapiente usa questo privilegio su un'altra creatura.

\subsection{Difesa predittiva}
Lo stile di combattimento di un Sapiente, basato sull'osservazione e lo studio, gli permette di anticipare e schivare gli attacchi. A partire dal 1° livello, nel calcolo della Classe Armatura, il Sapiente può usare il proprio modificatore di Intelligenza al posto di quello di Destrezza.

\subsection{Ambiti di Ricerca}
Mai soddisfatto della propria conoscenza corrente, un Sapiente è sempre alla ricerca di un modo per espandere i propri orizzonti. Al 2° livello, il Sapiente apprende due Ambiti di Ricerca a sua scelta dalla lista al termine della descrizione della classe.\\
Ne apprende un altro al 4°, 8°, 12° e 16° livello.

\subsection{Intelletto Maestoso}
La mente di un Sapiente è capace di maestose intuizini. A partire dal 2° livello, quando compie una prova di caratteristica che richiede o un tiro salvezza su Saggezza o Intelligenza, o quando tira per i danni contro il Mark, aggiunge il proprio Dado Intelletto, che è un d4, al tiro. \\
Inoltre, quando una creatura che una creatura che il Sapiente vede colpisce il Mark con un attacco, il Sapiente può usare la sua reazione per aggiungere il proprio Dado Intelletto al danno dell'attacco. \\
Il Dado Intelletto aumenta al 5° (d6), 10° (d8) e 15° (d10) livello.

\subsection{Disciplina Accademica}
Il Sapiente ha scelto di specializzarsi in un campo di studio. A partire dal 3° livello, sceglie una Disciplina Accademica: Stratega (e basta).\\ La Disciplina Accademica garantisce privilegi al 3° livello e poi di nuovo al 6°, 13° e al 17° livello.

\subsection{Riflessi Accelerati}
L'abilità di un Sapiente di reagire a ciò che lo circonda è quasi soprannaturale. A partire dal 5° livello, ottiene un bonus ai tiri di iniziativa pari al sio modificatore di Intelligenza, ed ottiene una seconda reazione ogni round. Un singolo effetto può scatenare una sola reazione. \\
Quando raggiunge l'11° livello in questa classe può sacrificare la sua azione bonus per una reazione in più.\\
Quando raggiunge il 17° livello in questa classe, ottiene una terza reazione ogni round, per un massimo di 4 reazioni.

\subsection{Studio rapido}
A partire dal 5° livello, il Sapiente è in grado di apprendere nuove capacità con estrema velocità. Ottiene competenza in un'abilità, oggetto o arma, o impara una nuova lingua a sua scelta. \\
Inoltre, il Sapiente può spendere 1 ora (durante un riposo breve o lungo), può cambiare questa competenza o lingua con un'altra competenza o lingua a sua scelta, purché abbia accesso ad un esempio da cui imparare. Con esempio si intende un insegnante consenziente, un libro in un altra lingua o il manuale di un oggetto.

\subsection{Percezione Affinata}
A partire dal 7° livello, il Sapiente non può essere sorpreso a meno che non sia incapacitato.\\
Inoltre, quando tira per iniziativa, può compiere una delle seguenti azioni prima che le altre creature agiscano:
\begin{itemize}
    \item Usare Attenta Analisi per scegliere come Mark una creatura che può vedere.
    \item Fare una prova di caratteristica di Intelligenza per ottenere informazioni.
    \item Fare una tra le azioni di Aiuto,  Prepararsi o Cercare.
\end{itemize}

\subsection{Lampo di Genialità}
A partire dal 9° livello, un numero di volte pari al proprio bonus di competenza ogni riposo breve, come reazione quando una creatura entro 9 metri compie un tiro salvezza, il Sapiente può urlargli un consiglio, garantendogli un bonus al tiro pari al suo Dado Intelletto\\
La creatura deve essere in grado di udire il Sapiente per ottenere questo bonus.

\subsection{Esperto di Previsioni}
Il Sapiente sembra sempre un passo avanti ai suoi nemici. A partire dal 10° livello, finché non è incapacitato, ha vantaggio su ogni tiro salvezza o prova di caratteristica che il Mark gli impone.

\subsection{Potente Osservazione}
Il Sapiente è in grado di identificare anche la più piccola debolezza e sfruttarla con spietata efficienza. A partire dall'11° livello, un numero di volte pari al proprio bonus di competenza ogni riposo breve può usare una reazione per aggiungere un tiro del proprio Dado Intelletto ad ogni tiro per i danni degli alleati, finché può vedere il bersaglio e l'attaccante lo può sentire.\\
Inoltre, quando usala propria reazione per aggiungere il proprio Dado Intelletto ai danni di un attacco contro il proprio Mark, può tirare il Dado Intelletto due volte e usare il risultato più alto.

\subsection{Volontà di Ferro}
A partire dal 14° livello, il Sapiente ottiene competenza nei tiri salvezza su Carisma. Quando è costretto a fare un tiro salvezza su Carisma, ottiene un bonus al tiro pari al proprio Dado Intelletto.\\
Inoltre, quando un effetto gli permette di fare un tiro salvezza su Intelligenza, Saggezza o Carisma per ridurre o annullare i danni, non prende danno in caso di successo e solo metà danno in caso di fallimento. Tuttavia subisce comunque gli effetti al di là del danno come specificato dalla sorgente.

\subsection{Profonda Comprensione}
A partire dal 18° livello, con un'azione, un Sapiente può prevedere la prossima mossa del suo Mark e avvertire gli alleati dei suoi piani. Fino all'inizio del prossimo turno del Sapiente, il Mark ha svantaggio in tutte le prove di caratteristica, tiri per colpire e tiri salvezza, e delle creature a scelta del Sapiente hanno vantaggio nei tiri salvezza che il Mark impone loro.\\
Una volta utilizzato questo privilegio, non può essere usato di nuovo finché non si completa un riposo breve o lungo.

\subsection{Genio Indiscusso}
Il Sapiente realizza il proprio vero potenziale di genio. Al 20° livello, il suo punteggio di intelligenza aumenta di 4, fino ad un massimo di 24. Inoltre, quando tira il proprio Dado Intelletto e ottiene un risultato minore del proprio modificatore di Intelligenza, può sostituire al valore ottenuto il proprio modificatore di Intelligenza.

\subsection{Ambiti di Ricerca}

\subsubsection{Agricoltura}
Il Sapiente ha un grande interesse in tutto ciò che cresce. Ogni volta che fa una prova di Intelligenza (Natura) relativamente a piante, fattorie, agricoltura, o giardinaggio, ha vantaggio in quel tiro. \\
Inoltre, se spende 8 ore a prendersi cura di colture, piante o campi, può lanciare la versione di 8 ore dell'incantesimo \textit{Crescita Vegetale} su ogni coltura egli decida di accudire in quel periodo.

\subsubsection{Astrologia}
Il Sapiente è un esperto studioso delle mappe stellari e delle costellazioni, e può usare i movimenti dei corpi celesti per prevedere il futuro. Durante un riposo lungo in cui può vedere il cielo stellato, può tirare un d20 e annotarne il valore. Può rimpiazzare un suo tiro per colpire, un suo tiro salvezza o una sua prova di abilità a sua scelta con questo valore, prima di aver effettuato tale tiro.\\
La sostituzione può essere effettuata al massimo una volta, e se non viene utilizzata entro la fine del riposo lungo successivo, è persa.

\subsubsection{Enigmi}
Il Sapiente ha speso molto tempo a riflettere e ragionare su enigmi e rime. Quando parla, può scegliere di farlo in Enigmi. Quando lo fa, sembra di star parlando normalmente, ma ci sono dei messaggi nascosti nelle sue parole. \\
In un'ora, che può trascorrere in un riposo breve o lungo, il Sapiente può insegnare ad una creatura a comprendere gli Enigmi. Una creatura con intelligenza pari o superiore a 13 che comprende gli Enigmi, può rispondere per Enigmi.

\subsubsection{Equitazione}
Il Sapiente ha sviluppato un grande amore per la cura dei cavalli. Il Sapiente ottiene competenza in Addestrare Animali. Ogni volta egli compie una prova di Saggezza (Addestrare Animali), può fare invece un tiro su Intelligenza (Addestrare Animali).\\
Inoltre, quando fa una prova su Addestrare Animali relativamente a prendersi cura di un cavallo, cavalcarlo o far riprodurre dei cavalli, ha vantaggio su quel tiro.\\
Infine, quando sta cavalcando un cavallo, aggiunge il proprio modificatore di Intelligenza ad ogni prova di abilità o tiro salvezza compiuto dalla cavalcatura.

\subsubsection{Falconeria}
Il Sapiente ha speso molti mesi ad imparare ad addestrare i rapaci. Ottiene un compagno Falco, che usa le statistiche del'Aquila Gigante del \textit{Manuale dei Mostri}, ma ha una statistica di Intelligenza di 14. Il Sapiente e il Falco possono comunicare usando gesti e suoni. Il falco è cecamente fedele al Sapiente e obbedisce ai suoi comandi.\\
In combattimento, il Falco condivide l'iniziativa del Sapiente e agisce nel suo turno. Può muoversi e usare la sua reazione in modo indipendente, ma può compiere solo l'azione Schivata a meno che il Sapiente non usi una propria azione bonus per ordinargli di compiere un'azione dal proprio stat block, o un'altra azione. Se il Sapiente è incapacitato, il Falco agisce autonomamente.\\
Se il Falco scende a 0 punti ferita, compie i tiri salvezza contro morte come farebbe un giocatore. In caso morisse, le abilità del Sapiente gli permettono di catturare e addomesticare un altro falco impiegandoci 8 ore di lavoro e spendendo 5 mo in esche.

\subsubsection{Gioielliere}
Il Sapiente è un apprendista orafo. Ottiene competenza in due strumenti da gioielliere, e il suo bonus di competenza raddoppia ai fini di ogni tiro che li utilizzi.\\
Inoltre, quando è alla luce del sole, può spendere un''ora usando strumenti da gioielliere per modificare  una gemma che egli possa toccare in modo che assorba la luce solare.\\ 
Come azione, può rilasciare la luce contenuta nella gemma, facendole emettere luce intensa in un raggio di 9 metri e luce fioca in un raggio di 18 metri. Dopo 24 ore, una gemma modificata in questo modo, perde le sue proprietà e ritorna normale.

\subsubsection{Istruzione}
Il Sapiente ha dedicato una significativa parte dei suoi studi alla didattica. Alla fine di un riposo lungo può insegnare a una creatura in grado di sentirlo una competenza o un linguaggio che egli conosce. Fino alla fine di un nuovo riposo lungo del Sapiente, la creatura ottiene quella competenza.

\subsubsection{Linguistica}
Il Sapiente impara a parlare, leggere e scrivere un numero di lingue pari al proprio modificatore di intelligenza.\\
Quando compie un tiro su Carisma (Persuasione) parlando a una creatura nella sua lingua madre (a patto che sia diversa dal Comune), ha vantaggio nel tiro.

\subsubsection{Maestria in abilità}
Il Sapiente ha speso un significativo ammontare di tempo, e un importante sforzo nella ricerca su una singola abilità. Sceglie un'abilità in cui ha competenza, e ogni volta che fa una prova di caratteristica che richieda l'abilità o l'oggetto scelti può trattare un tiro di 7 o inferiore sul d20 come un 8.\\
È possibile ottenere questo Ambito di Ricerca più di una volta ma scegliendo ogni volta una competenza differente.

\subsubsection{Meditazione}
Il Sapiente spende ogni giorno del tempo a riflettere per schiarirsi la mente e mantenerla lucida. Ogni volta lancia il proprio Dado Intelletto e ottiene un 1, può ritirarlo finché non ottiene un risultato più alto di 1.\\
Può ottenere questo Ambito di Ricerca al più tre volte, aumentando ogni volta il numero che può ritirare: la seconda volta ritira 1 e 2 mentre la terza 1, 2 e 3.

\subsubsection{Memoria Infallibile}
Il Sapiente può ricordare alla perfezione i dettagli di ciò che decide di memorizzare. Se spende almeno un minuto ad osservare un oggetto o una creatura, potrà raddoppiare il suo bonus di competenza sui tiri di Intelligenza (Storia) per ricordare un'informazione che ha osservato su di esso.

\subsubsection{Musica}
Il Sapiente ha un fine orecchio musicale. Ottiene competenza in Intrattenere e in tre strumenti musicali a scelta. Ogni volta compie una prova di Carisma (Intrattenere) o una prova di caratteristica che usa uno strumento in cui il Sapiente ha competenza, ottiene un bonus al tiro pari al proprio modificatore di intelligenza.

\subsubsection{Navigazione}
Il Sapiente ottiene competenza negli strumenti da cartografo e da navigatore e  nei veicoli (acqua). Ogni volta egli compia un prova di abilità con questi strumenti, usa sempre il proprio punteggio di intelligenza e può aggiungere il doppio del suo bonus competenza al proprio tiro.\\
Inoltre, finché ha una mappa (purchè essa sia accurata) o può vedere il cielo stellato (purchè esso sia sensato e accurato rispetto alla sua esperienza), non può perdersi neanche attraverso la magia.

\subsubsection{Scherma}
Il Sapiente ottiene competenza in spade lunghe, stocchi e scimitarre.\\
Quando una creatura che può vedere compie un attacco con un'arma da mischia contro il Sapiente, questi può usare la sua reazione per lanciare il proprio Dado Intelletto e aggiungere il suo risultato alla propria Classe Armatura contro quell'attacco. Se l'attacco manca, il Sapiente può contrattaccare con una spada lunga, uno stocco o una scimitarra come parte della stessa reazione.

\subsubsection{Teologia}
Il Sapiente è un devoto studente del divino. Impara a parlare, leggere e scrivere il Celestiale e ottiene competenza in Religione.\\
Ogni volta che compie una prova di Intelligenza (Religione) per ricordarsi di devozioni religiose, rituali o costumi, o un tiro su Carisma per interagire con un Celestiale o un altro servo del Divino, ha vantaggio nel tiro.

\subsubsection{Tiratore Scelto}
Il Sapiente ottiene competenza in ogni arma marziale a distanza. Quando compie un attacco a distanza con un'arma, può usare il proprio Dado Intelletto al posto del dado utilizzato dall'arma per i danni.\\
Inoltre, se l'ambientazione include armi da fuoco, e il Sapiente è stato esposto alla tecnologia che le fa funzionare, è considerato competente in tutte le armi da fuoco semplici e marziali.

\subsubsection{Tradizioni}
Il Sapiente è uno studente di cultura, politica e tradizioni. Quando fa una prova di  Intelligenza (Storia) relativamente a costumi locali, cultura o tradizione, ha vantaggio nel tiro. \\
Questa eclettica conoscenza dei costumi e delle tradizioni locali lo rende inoltre un eccellente ambasciatore: quando incontra un leader o un'importante figura locale, ha vantaggio in ogni tiro su Carisma che gli viene richiesto.