\chapter{Sapiente}
\textit{Creato da u/LaserLlama, tradotto da Davide Borra e ribilanciato dal sottoscritto}

\subsection{Menti magnifiche}
Ci sono molte persone magnificamente intelligenti al mondo, ma poche sono veri Sapienti. Nati con l'innato desiderio di conoscere tutto ciò che possono, e il potenziale per un intelletto geniale, i Sapienti spendono la loro intera vita imparando tutto ciò che chi li circonda è disposto ad insegnargli. Spesso riconoscibili ad una giovane età, la loro insaziabile sete di conoscenza li conduce nelle più grandi librerie, università e nei principali luoghi di sapere. Desiderano viaggiare per scoprire i più reconditi segreti del mondo, spesso dedicando la propria vita all'avventura. Per un Sapiente, nessun prezzo è troppo alto per la promessa della scoperta.

\subsection{Intenso interesse}
I Sapienti sono molto concentrati nell'area di studi che anno scelto, e spesso diventano ossessionati dal conoscere tutto cià che possono sulla loro specialità. Nella loro missione per la scoperta, i Sapienti sono disposti a rinunciare ad ogni loro convinzione, politica o religiosa che sia, per acquisire le informazioni che anelano. Per essi, il loro deriderio di conoscenza è più importante della lealtà verso un gruppo o un'ideologia. \\
Spesso pagando un grande prezzo, i Sapienti non si fermeranno davanti a nulla finché non avranno fatto una scoperta rivoluzionaria nella loro ariea di studio. non è raro incontrare uno di questi studiosi lontano dalla sicurezza di un'universita e le sue librerie.

\subsection{Creare un Sapiente}
Durante la creazione di un Sapiente, bisognerebbe considerare la loro istruzione e il loro livello di educazione formale. Era la promessa della migliore università il denaro può comprare? Oppure, era un figlio della strada, disposto a combattere per ogni frammento di conoscenza su cui potrebbe mettere le mani? Forse la loro mente è stata allenata fin dalla nascita per essere il perfetto strumenti di analisi per una nobile famiglia.\\
Considera anche perché il tuo Sapiente ha scelto di affidarsi al proprio intelletto anziché usare i propri doni nello studio dell'arcano o al servizio di un potere più alto. Sono condannati a non poter lanciare neanche la più semplice magia? O hanno giurato di non affidarsi mai al potere che ha distrutto la loro famiglia?\\
Infine, perchè ha scelto di diventare un avventuriero invece che vivere una vita accademica? Ha scelto di andare oltre il normale studio e dedicarsi alle infinite scoperte dell'avventura? Oppure ha sempre avuto una passione per lo studio del mondo?

\begin{DndSidebar}{Sapiente e Multiclasse}
\paragraph{Punteggio di caratteristica minimo} Il personaggio multiclasse deve avere un punteggio di Intelligenza pari o superiore a 13 per acquisire un livello in questa classe o per acquisire un livello in un'altra classe se é gia un Sapiente,
\paragraph{Competenze ottenute} Se il Sapiente non é la classe iniziale del personaggio, ecco quali competenze ottiene quando acquisisce il suo primo livello da Sapiente: armature leggere, un'abilità a scelta dalla lista delle abilità da Sapiente e un set di strumenti da artigiano a scelta.
\end{DndSidebar}

\section{Creazione}

\subsection{Punti ferita}
\paragraph{Dadi Vita} 1d8 per livello da Sapiente
\paragraph{Punti ferita al 1° livello} 8 + il modificatore di Costituzione del Sapiente
\paragraph{Punti ferita ai livelli successivi} 5 (1d8) + il modificatore di Costituzione del Sapiente

\subsection{Competenze}
\paragraph{Armature} Armature leggere
\paragraph{Armi} Armi semplici, spade corte
\paragraph{Strumenti} Un set di strumenti da artigiano a scelta
\paragraph{Tiri Salvezza} Intelligenza, Saggezza
\paragraph{Abilità} Tre a scelta tra Arcano, Storia, Indagare, Intuizione,  Medicina, Natura, Persuasione o Religione

\subsection{Equipaggiamento iniziale}
Inizia con l'equipaggiamento seguente, in aggiunta all'equipaggiamento conferito dal suo background:
\begin{itemize}
    \item {\it(a)} un'arma semplice a scelta o {\it(b)} una spada corta.
    \item {\it(a)} una balestra leggera e 20 frecce o {\it(b)} due pugnali.
    \item un set di strumenti da artigiano a scelta
    \item un'armatura di cuoio, e una borsa da studioso
    \item 5d4 x 10 mo
\end{itemize}

\subsection{Creazione Rapida}
Puoi creare rapidamente un Sapiente usando le seguenti linee guida. Prima, metti il tuo punteggio di Intelligenza al punteggio più alto, seguito da Destrezza. Secondo, scegli il background nobile.

\subsection{Privilegi del Sapiente}
I privilegi ottenuti dal Sapiente a ogni livello sono riassunti in questa tabella.

\begin{DndReadAloud}
    \begin{DndTable}[header=Sapiente]{cccc}
        Lv & PB& Privilegi di Classe & \begin{tabular}{cc}\ Abilità\ \\Accad.\end{tabular}\\
        1° & +2 & \begin{tabular}{c}Attenta Analisi,\\Difesa predittiva                     \end{tabular}&--\\
        2° & +2 & \begin{tabular}{c}Ambiti di Ricerca,\\ Intelletto Maestoso (d4)   \end{tabular}& 2\\
        3° & +2 & \begin{tabular}{c}Disciplina Accademica                           \end{tabular}& 2\\
        4° & +2 & \begin{tabular}{c}Aumento dei punteggi\\ di caratteristica          \end{tabular}& 3\\
        5° & +3 & \begin{tabular}{c}Riflessi Accelerati (2)\\Studio Rapido\\Intelletto maestoso (d6)                         \end{tabular}& 3\\
        6° & +3 & \begin{tabular}{c}Privilegio della\\Disciplina Accademica                           \end{tabular}& 3\\
        7° & +3 & \begin{tabular}{c}Percezione Affinata                              \end{tabular} & 3\\
        8° & +3 & \begin{tabular}{c}Aumento dei punteggi\\ di caratteristica          \end{tabular}& 4\\
        9° & +4 & \begin{tabular}{c}Flash di Genialità                        \end{tabular} & 4\\
        10° & +4 & \begin{tabular}{c}Esperto di Previsioni\\Intelletto maestoso (d8)                          \end{tabular}& 4\\
        11° & +4 & \begin{tabular}{c}Potente Osservazione                             \end{tabular}& 4\\
        12° & +4 & \begin{tabular}{c}Aumento dei punteggi\\ di caratteristica         \end{tabular}& 5\\
        13° & +5 & \begin{tabular}{c}Privilegio della\\Disciplina Accademica                       \end{tabular}& 5\\
        14° & +5 & \begin{tabular}{c}Volontà di Ferro                        \end{tabular}& 5\\
        15° & +5 & \begin{tabular}{c}Intelletto Maestoso (d10)                             \end{tabular}& 5\\
        16° & +5 & \begin{tabular}{c}Aumento dei punteggi\\ di caratteristica         \end{tabular}& 6\\
        17° & +6 & \begin{tabular}{c}Privilegio della\\Disciplina Accademica,\\Riflessi Accelerati (3)                       \end{tabular}& 6\\
        18° & +6 & \begin{tabular}{c}Profonda Comprensione                          \end{tabular}& 6\\
        19° & +6 & \begin{tabular}{c}Aumento dei punteggi\\ di caratteristica         \end{tabular}& 6\\
        20° & +6 & \begin{tabular}{c}Genio Indiscusso                             \end{tabular}& 6\\
    \end{DndTable}
\end{DndReadAloud}

\subsection{Attenta Analisi}