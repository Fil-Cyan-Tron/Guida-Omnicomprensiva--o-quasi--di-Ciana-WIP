\nchapter{Warlock}

\section{Privilegi di Classe Opzionali}

\subsection{Suppliche Occulte}

\subsubsection{Occhi Aperti}
\textit{Prerequisito: 18° livello} \\ Il potere del patrono, apre gli occhi al Warlock, che potrebbe non essere più in grado di richiuderli: ottiene \textit{vista pura}, non può essere limitata o disattivata se non rimuovendo questa supplica o accecandosi volontariamente. \\
La costante esposizione al Piano Etereo è estremamente impegnativa anche per la mente di incantatori molto esperti, quindi il Warlock ha bisogno di tempo ed esperienza per abituarsi.
\paragraph{Dopo nessun aumento di livello}Il Warlock ha costanti emicranee di intensità variabile in base alla presenza di auree molto potenti; in presenza di creature la cui forma nel Piano Etereo risulterebbe sconvolgente o disturbante (a discrezione del DM), il Warlock ha svantaggio ai TS per mantenere la concentrazione e ogni attacco inflitto da tali creature provoca 1d4-1 danni psichici aggiuntivi.
\paragraph{Dopo un aumento di livello}Il Warlock si sta abituando alla forma del mondo sul Piano Etereo. Non subisce più danni aggiuntivi e non ha più svantaggio ai TS per mantenere la concentrazione, a meno che questa non sia disturbata direttamente dalle creature dall'aspetto più disturbante.
\paragraph{Dopo due aumenti di livello o al 20° livello} Ormai la \textit{vista pura} è l'unica vista che il Warlock conosca, il Piano Etereo e il Piano Materiale per lui sono assolutamente inseparabili. La \textit{vista pura} non gli causa più nessun problema. Da questo momento in poi, rimuovere questa supplica gli causerebbe cecità permanente, incurabile se non da altre sorgenti di \textit{vista pura} (come \textit{Visione del Vero}). \\
Riottenere questa supplica dopo averla rimossa richiede di riabituarsi a essa.

\subsubsection{Deflagrazione Aleatoria}
\textit{Prerequisito: trucchetto Deflagrazione Occulta} \\
Un numero di volte pari al suo bonus di competenza per riposo breve, quando il Warlock colpisce un bersaglio con il trucchetto Deflagrazione Occulta, come azione bonus può imporre al bersaglio di tirare dalla tabella della Magia Selvaggia un numero di volte pari al numero di raggi da cui è stato colpito.

\section{Patrono Ultraterreno: Spazio Proiettivo}

\begin{DndReadAloud}
  \it
  "E quando l'ho visto, ragazzi, ho esclamato 'Ma questa è Geometria Pura!'" \\ (M. Andreatta)
\end{DndReadAloud}

\DndDropCapLine{G}{}irano voci su una certa entità... Un oggetto legato alla stessa natura dei piani dove ci muoviamo... coloro che ne hanno ricevuto l'intuizione la descrivono come "una ganzata pazzesca" o "roba da Harry Potter", perdono l'uso della ragione e vanno in giro a importunare i passanti con frasi terrificanti come "Ragazzi, avete della Geometria?"... roba da brividi...

\subsection{Lista degli incantesimi ampliata}

\begin{DndTable}{XX}
  Livello dell'incantesimo  & Incantesimi \\
  1°  &  Caduta Morbida, Colpo Intrappolante\\
  2°  &  Levitazione, Zona di Verità\\
  3°  &  Glifo di Interdizione, Lentezza\\
  4°  &  Divinazione, Sfera Elastica di Otiluke\\
  5°  &  Cerchio di Teletrasporto, Legame Planare\\
\end{DndTable}

\subsection{Forma lineare associata}

A partire dal 1° livello, una volta per riposo lungo, il Warlock può indicare una creatura che è in grado di vedere. Con un'azione può azzerare il bonus del tiro per colpire di quella creatura fino all'inizio del proprio prossimo turno.

\subsection{Omogeneizzazione}

A partire dal 6° livello, un numero di volte pari al suo bonus di competenza per riposo lungo, quando una creatura è a terra con 0 punti ferita, il Warlock può usare la sua azione e toccarla per trasferirla in un semipiano temporaneo. \\ Finchè la creatura si trova nel semipiano, il Warlock ha uno slot incantesimo aggiuntivo. Quando il Warlock inizia un riposo breve o usa questo privilegio su un'altra creatura, la creatura intrappolata viene liberata e diventa stabile.

\subsection{Proiezione}

A partire dal 10° livello, un numero di volte pari al suo bonus di competenza per riposo lungo, il Warlock può usare la sua azione per proiettare la sua immagine su una superficie solida che è in grado di vedere. \\ Per non più di 1 minuto, la posizione del Warlock diventa quella della sua immagine. Fintanto che si trova su una superficie, il Warlock può muoversi liberamentein ogni direzione lungo la stessa, al doppio della sua velocità di movimento, ma non può effettuare azioni che coinvolgano il mondo esterno ad eccezione del parlare. \\ Dopo 1 minuto, il Warlock esce dalla superficie e si trova nello spazio disponibile più vicino a dove si trovava sulla superficie. Questo effetto termina in anticipo se la superficie dove si trova il Warlock viene distrutta o se usa la sua azione per uscirne.

\subsection{Chiusura Proiettiva di Bézout}

A partire dal 14° livello, una volta per riposo breve, il Warlock può usare la sua azione bonus per generare un'aura di Geometria Pura nel raggio di 36m intorno a sè. \\ Fino alla fine del prossimo turno del Warlock, tutti gli attacchi a distanza che richiedono un tiro per colpire effettuati da creature dentro l'aura contro altre creature dentro l'aura vanno automaticamente a segno senza effettuare il tiro per colpire.

\section{Patrono Ultraterreno: Signore dell'Assurdo}

\begin{DndReadAloud}
  \it
  Dopo che il fantasmino l'ebbe spaventata con un "Ciao!" alle spalle, Fillianore con un sobbalzo materializzò il suo violino e lanciò un acuto stridentissimo. Quando il raggio di energia colpì Amarilli, un triangolo ben vestito si materializzò con un "Wahoo!" per una ventina di volte prima che questa esplodesse in un crepitare di scintille arcobaleno.
\end{DndReadAloud}

\DndDropCapLine{N}{}el multiverso sono poche le creature in grado di esistere contemporaneamente in ogni realtà. È questo il caso dell'entità conosciuta nel nostro mondo come il Signore dell'Assurdo. A voi potrebbe essere familiare sotto altri nomi, come Il Triangolo, Fancy Dorito, Alex, o il suo preferito... Bill. \\ Il suo vero obiettivo non è chiaro, ma una cosa è certa: ciò che pretende dai suoi Warlock è una sana dose di divertimento.

\subsection{Lista degli incantesimi ampliata}

\begin{DndTable}{XX}
  Livello dell'incantesimo  & Incantesimi \\
  1°  & Risata Incontenibile di Tasha, Dardo Tracciante\\
  2°  & Alterare Sè Stesso, Trucco della Corda \\
  3°  & Palla di Fuoco, Parola Guaritrice di Massa \\
  4° & Santuario Privato di Mordenkainen, Tentacoli Neri di Evard \\
  5° & Dominare Persone, Ristorare Superiore \\
\end{DndTable}

\subsection{Squarcio della fortuna}

A partire dal 1° livello, dopo aver lanciato un incantesimo il Warlock deve tirare il d20. Se il risultato del tiro è uguale o inferiore a 2 + il livello dello slot utilizzato, 2+0 nel caso di un trucchetto, il Warlock deve tirare il d100 e subire il relativo effetto dalla tabella della Magia Selvaggia. \\ Se il risultato dalla tabella è il lancio di un incantesimo, subisce anch'esso gli effetti di Squarcio della fortuna.\\ In compenso, il Warlock ha uno slot incantesimo per riposo breve in più, una supplica occulta in più e un utilizzo dell'arcanum mistico di ogni livello per riposo lungo in più.

\begin{DndTable}{XX}
  Livello da Warlock & Dado esplosivo \\
  1°-4°  & d4\\
  5°-10°  & d6 \\
  11°-16°  & d8 \\
  17°-20° & d10 \\
\end{DndTable}

\subsection{Deflagrazione deflagrante}

A partire dal 1° livello quando il Warlock colpisce un avversario con il trucchetto "Deflagrazione Occulta", come azione bonus può aggiungere il suo dado esplosivo ai danni inflitti da ogni raggio. \\ Se il Warlock è sotto un qualunque effetto di massimizzazione del risultato dei tiri per i danni, non può aggiungere il dado esplosivo. \\ Questo privilegio può essere usato un numero di volte pari al bonus di competenza del Warlock per riposo breve.

\subsection{(S)fortunato}

A partire dal 6° livello, il Warlock può appellarsi al suo patrono per alterare il fato in suo favore. \\ Ottiene il talento "Fortunato", ma ogni volta che lo usa deve tirare dalla tabella della Magia Selvaggia.

\subsection{Divertimento Extraplanare}

L'influenza del patrono si espande alle creature intorno al Warlock. \\ Al 10° livello il Warlock ottiene il privilegio "Ispirazione bardica", il dado di ispirazione è il suo dado esplosivo (che se usato in questo modo non esplode). \\ Se il dado di ispirazione risulta in un 1, l'utilizzatore deve tirare dalla tabella della Magia Selvaggia.

\subsection{Pandemonio}

Una volta per riposo lungo, il Warlock può imporre con un'azione a tutte le creature coscienti (compreso sè stesso) nel raggio di 18m di tirare dalla tabella della Magia Selvaggia e tirare il dado esplosivo del Warlock. Ogni creatura riceve il totale dei danni da lei tirati come danni psichici.

\section{Patrono Ultraterreno: Benefattore Temporale}
\textit{Questo patrono è largamente ispirato dai due "Sè del Futuro" di u/EinksArcana e u/Uroul su Reddit}

\begin{DndReadAloud}
  \it
  "Uh" il giovane Aiden osservò i tre Galeb Duhr che lo circondavano e sparì con un sonoro "Pop". \\ Prima che i tre elementali potessero capire cosa fosse successo, il braccio di uno dei Guardiani di Pietra esplose in un lampo di energia arancione brillante.
\end{DndReadAloud}

\DndDropCapLine{L}{}a figura di Aiden Agathys è indubbiamente una delle più misteriose che i nostri eroi abbiano avuto modo di incontrare. In particolare, Lucian ha potuto sperimentare qualche... ragionevole grado di stranezza. Non si sa cosa voglia ottenere Aiden, ma l'identità che ha assunto è quella del Benefattore Temporale, un anonimo gentiluomo che contatta un giovane per offrirgli poteri e privilegi in cambio di... occasionale collaborazione.

\subsection{Lista degli incantesimi ampliata}

\begin{DndTable}{XX}
  Livello dell'incantesimo  & Incantesimi \\
  1°  & Dardo Tracciante, Parola Guaritrice\\
  2°  & Localizza Oggetto, Zona di Verità\\
  3°  & Chiaroveggenza, Velocità\\
  4° & Divinazione, Localizza Creatura\\
  5° & Conoscenza delle Leggende, Modificare Memoria\\
\end{DndTable}

\subsection{Biglietto da Visita}
Al 1° livello, il Benefattore dona al suo Warlock il suo biglietto da visita. Le sue caratteristiche estetiche sono determinate dal DM in base alla vera identità del Benefattore (NB, non è strettamente necessario che si tratti di un biglietto da visita vero e proprio).\\
Attraverso di esso, il Benefattore può mandare brevi messaggi al suo Warlock, il quale può a sua volta porgli attraverso di esso delle brevi domande alle quali ricevere altrettanto brevi risposte.\\
Il Warlock apprende il trucchetto \textit{Guida}.

\subsection{Aiutino}
Occasionalmente, il Benefattore interviene per manipolare la sorte in favore del suo protetto.\\
Al 6° livello, il Warlock ottiene il talento \textit{Fortunato}.

\subsection{Progressi accelerati}
Man mano che il potere del protetto cresce, anche i privilegi concessi dal Benefattore si fanno più forti.\\
A partire dall'11° livello, il Warlock ottiene una supplica occulta aggiuntiva e ne ignora i requisiti di livello. Gli altri requisiti della supplica devono comunque essere soddisfatti.

\subsection{Giro di Orologio}
A partire dal 14° livello, il Warlock inizia a comprendere davvero una parte delle capacità del suo Benefattore.\\
Un numero di volte pari al suo bonus di competenza per riposo breve, all'inizio del suo turno il Warlock può evocare una sua ombra temporale. Essa agisce al suo posto. Alla fine del suo turno, il Warlock sceglie se saltare al posto dell'ombra (e ufficializzare temporalmente il suo turno) o ripetere il turno con un corso d'azione diverso.

%\section{Patrono Ultraterreno: Sè del Futuro superato}

\DndDropCapLine{I}{}l tempo è una di quelle cose con cui si dovrebbe stare sempre molto attenti a giocare, non si sa mai che genere di pasticci potrebbero verificarsi. Quando da piccolo Aiden fu contattato dal suo Benefattore Temporale, non poteva immaginare che questi non fosse altro che il sè del futuro e che sarebbe stata solo una questione di tempo prima di arrivare a raggiungerlo... Ma soprattutto nemmeno l'Aiden adulto poteva prevederne le conseguenze.\\
Incatenato agli orrori dell'unica linea temporale che non può cambiare, impossibilitato a tornare più indietro di quel tragico contatto, la sua sanità mentale è stata logorata dal supplizio di Sisifo... Ormai non rimane più traccia del ragazzo sorridente che fingeva di detestare gli orfanelli a cui insegnava nella scuola di sua sorella, rimane solo un'ombra di rancore e odio, determinata a bruciare tutto con le fiamme della follia.

\subsection{Prerequisiti}
\paragraph{Livello} Il Warlock deve raggiungere il 20° livello da Warlock del Benefattore Temporale
\paragraph{Identità} Il Benefattore Temporale deve essere il Warlock stesso

\subsection{Multitasking}
Il Warlock ottiene un'azione, due azioni bonus e due reazioni aggiuntive per turno.

\subsection{Arcanum mistico migliorato}
Il Warlock perde i suoi utilizzi dell'Arcanum Mistico e ottiene tre slot di Magia del Patto di 6° livello, tre di 7°, due di 8° e due di 9°.

\subsection{Magia del patto migliorata}
Il Warlock può usare due azioni, tre azioni bonus e tre reazioni per riottenere i suoi slot spesi di Magia del Patto.

\subsection{Lista degli incantesimi espansa}
Il Warlock può imparare incantesimi da qualsiasi lista degli incantesimi.

\subsection{Dono del patto migliorato}
Il Warlock ottiene i doni del patto che non possiede (tra patto del Tomo, della Lama, della Catena e del Talismano).

\subsection{Suppliche Occulte aggiuntive}
Il Warlock ottiene quattro suppliche occulte aggiuntive.

\subsection{Instabilità temporale}
Il Warlock ha svantaggio ai tiri salvezza in cui non ha competenza.\\
La sua CA diventa pari a 10 + il suo bonus di competenza + il suo modificatore di Carisma.\\
Se entra in contatto col Warlock, una creatura vivente deve fare un tiro salvezza su Costituzione contro la CD del Warlock. Se lo fallisce subisce 1d4 danni necrotici, 1d4 danni radiosi e ottiene almeno 1d8 punti di avvelenamento da radiazione. Se lo supera, subisce la metà di questi danni e non ottiene nessun punto di avvelenamento da radiazione.

\subsection{Spostamento Temporale}
Il Warlock può spostarsi tra una linea temporale e l'altra, viaggiando nel passato e nel futuro ma mai non può mai tornare più indietro del momento in cui si è raggiunto.\\
Se prova a effettuare alterazioni troppo drammatiche, viene ricatapultato al momento del superamento.